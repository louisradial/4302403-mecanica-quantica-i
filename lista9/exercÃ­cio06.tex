\begin{exercício}{Partícula confinada em uma caixa esférica}{exercício06}
    Uma partícula de massa \(m\) confinada numa caixa esférica de raio \(a\) é descrita pelo hamiltoniano
    \begin{equation*}
        H = \frac{\vetor{p}^2}{2m} + V(r),
    \end{equation*}
    onde o potencial central \(V(r)\) é dado por
    \begin{equation*}
        V(r) = \begin{cases}
            0, &\text{se } r \in (0,a)\\
            \infty, &\text{se }r \notin (0,a).
        \end{cases}
    \end{equation*}
    \begin{enumerate}[label=(\alph*)]
        \item Qual a condição de contorno a ser adotada na solução do problema de autovalores deste hamiltoniano?
        \item Qual a energia do estado fundamental? Qual a sua degenerescência?
    \end{enumerate}
\end{exercício}
\begin{proof}[Resolução]

\end{proof}
