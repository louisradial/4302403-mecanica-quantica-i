\begin{exercício}{O grupo \(\mathrm{SU}(2)\)}{exercício2}
    O grupo \(\mathrm{SU}(2)\) é o grupo de matrizes \(2\times 2\) unitárias e de determinante unitário.
    \begin{enumerate}[label=(\alph*)]
        \item Mostre que
            \begin{equation*}
                \mathrm{SU}(2) = \setc*{\begin{pmatrix}
                    a && b\\
                    -\conj{b} && \conj{a}
                \end{pmatrix}}{a,b \in \mathbb{C} : \abs{a}^2 + \abs{b}^2 = 1}.
            \end{equation*}
        \item Mostre que se \(U \in \mathrm{SU}(2)\) pertence a uma vizinhança da identidade podemos escrever \(U = \unity - i \tau\), com \(\tau\) auto-adjunto, e que \(\tau\) é combinação linear real das matrizes de Pauli, dada por \(\tau = \frac12 \theta_i \sigma_i\), com \(\theta_i \to 0\).
        \item Seja \(\theta^2 = \theta_i \theta_j \delta_{ij}\) e \(\theta_i = \theta n_i\), onde \(\vetor{n}\) é um vetor unitário. Supondo agora que os \(\theta_i\) sejam finitos, definimos
            \begin{equation*}
                U_{\vetor{n}}(\theta) = \lim_{N\to \infty} \left[U\left(\frac{\theta}{N}\right)\right]^N.
            \end{equation*}
            Mostre que \(U_{\vetor{n}}(\theta) = \exp\left(-\frac12i\theta \vetor{n} \cdot \vetor{\sigma}\right)\) e mostre que toda matriz de \(\mathrm{SU}(2)\) é dessa forma.
    \end{enumerate}
\end{exercício}
\begin{proof}[Resolução]
    Consideremos a aplicação
    \begin{align*}
        A : \mathbb{C} \times \mathbb{C} &\to \mathrm{Mat}(2, \mathbb{C})\\
                                   (x,y) &\mapsto
        \begin{pmatrix}
            x && y\\
            -\conj{y} && \conj{x}
        \end{pmatrix}.
    \end{align*}
    Sejam \(a, b \in \mathbb{C}\) tais que \(\abs{a}^2 + \abs{b}^2 = 1\), então \(\det{A(a,b)} = 1\) e
    \begin{equation*}
        \herm{A(a,b)}A(a,b) = \begin{pmatrix}
            \conj{a} && -b\\
            \conj{b} && a
        \end{pmatrix}
        \begin{pmatrix}
            a && b\\
            -\conj{b}&&\conj{a}
        \end{pmatrix} =
        \begin{pmatrix}
            \abs{a}^2 + \abs{b}^2 && 0\\
            0 && \abs{a}^2 + \abs{b}^2
        \end{pmatrix}
         = \unity,
    \end{equation*}
    portanto \(A(a,b) \in \mathrm{SU}(2)\), isto é, \(\mathrm{SU}(2) \supset \setc{A(a,b)}{a,b\in \mathbb{C} : \abs{a}^2 + \abs{b}^2 = 1}\). Seja \(U \in \mathrm{SU}(2)\), então existem \(\alpha_{0}, \alpha_{1}, \alpha_{2}, \alpha_{3} \in \mathbb{C}\) tais que
    \begin{equation*}
        U = \begin{pmatrix}
            \alpha_0 && \alpha_1\\
            \alpha_2 && \alpha_3
        \end{pmatrix}.
    \end{equation*}
    Como \(\det(U) = 1\), temos \(\alpha_0 \alpha_3 - \alpha_1 \alpha_2 = 1\) e da unitariedade de \(U\), segue que
    \begin{equation*}
        \begin{pmatrix}
            \conj{\alpha_0} && \conj{\alpha_2}\\
            \conj{\alpha_1} && \conj{\alpha_3}
        \end{pmatrix}
        =
        \begin{pmatrix}
            \alpha_3 && -\alpha_1\\
            -\alpha_2 && \alpha_0
        \end{pmatrix},
    \end{equation*}
    portanto \(\conj{\alpha_3} = \alpha_0\) e \(\alpha_2 = -\conj{\alpha_1}\). Assim, temos \(\abs{\alpha_0}^2 + \abs{\alpha_1}^2 = 1\) e
    \begin{equation*}
        U = \begin{pmatrix}
            \alpha_0 && \alpha_1\\
            -\conj{\alpha_1} && \conj{\alpha_0}
        \end{pmatrix} = A(\alpha_0, \alpha_1),
    \end{equation*}
    e concluímos que \(\mathrm{SU}(2) =\setc{A(a,b)}{a,b\in \mathbb{C} : \abs{a}^2 + \abs{b}^2 = 1}. \)

    É claro que podemos tomar a aplicação \(A\) como uma função de quatro números reais, a saber as partes real e imaginária dos argumentos, portanto
    \begin{equation*}
        A(a_1, a_2, b_1,b_2) = \begin{pmatrix}
            a_1 + i a_2 && b_1 + i b_2\\
            -b_1 + i b_2 && a_1 - i a_2
        \end{pmatrix} = a_1 \unity + i a_2 \sigma_1 + i b_1 \sigma_2 + i b_2 \sigma_3,
    \end{equation*}
    e podemos escrever \(\mathrm{SU}(2) = \setc{u_0 \unity + i u_1 \sigma_1 + i u_2 \sigma_2 + i u_3 \sigma_3}{(u_0, u_1, u_2, u_3) \in S^3}\), onde \(S^3\) é a 3-esfera. Notemos que para todo \(U \in \mathrm{SU}(2) \setminus \set{\unity, -\unity}\), podemos escrever univocamente \(u_0 = \cos\varphi\) para algum \(\varphi \in (-\pi, \pi)\), de forma que podemos definir \(\eta_j = \frac{u_j}{\sin\varphi}\), o que implica que o vetor \(\vetor{\eta} = \eta_j \vetor{e}_j\) é unitário e podemos escrever
    \begin{equation*}
        U = \cos\varphi \unity + i \sin\varphi \vetor{\eta}\cdot \vetor{\sigma}.
    \end{equation*}
    Resumindo, concluímos que
    \begin{equation*}
        \mathrm{SU}(2) = \setc{\cos\varphi \unity + i \sin\varphi \vetor{\eta}\cdot \vetor{\sigma}}{\varphi \in [-\pi,\pi], \vetor{\eta} \in \mathbb{R}^3 : \norm{\vetor{\eta}} = 1},
    \end{equation*}
    já que tomando \(\varphi = \pm\pi\), temos \(\pm \unity\) e a escolha de \(\vetor{\eta}\) é irrelevante. No \cref{ex:exercício3} mostraremos ainda que
    \begin{equation*}
        \mathrm{SU}(2) = \setc*{\exp\left(-\frac12 i \theta \vetor{n}\cdot\vetor{\sigma}\right)}{\theta \in [-2\pi,2\pi], \vetor{n} \in \mathbb{R}^3 : \norm{\vetor{n}} = 1},
    \end{equation*}
    portanto para uma elemento \(U\) na vizinhança da identidade temos
    \begin{equation*}
        U = \exp\left(-\frac12 i \theta \vetor{n}\cdot\vetor{\sigma}\right) \simeq \unity - \frac12 i \theta \vetor{n} \cdot \vetor{\sigma},
    \end{equation*}
    onde \(\abs{\theta} \ll 1\) e \(\frac12 \theta \vetor{n} \cdot \vetor{\sigma}\) é auto-adjunto.

    Definimos então a aplicação
    \begin{align*}
        U : S^2 \times (-\epsilon, \epsilon) &\to \mathrm{SU}(2)\\
                          (\vetor{n},\theta) &\mapsto \unity - \frac12 i \theta \vetor{n} \cdot \vetor{\sigma}
    \end{align*}
    onde \(\epsilon \ll 1\). Fixando uma direção \(\vetor{n} \in S^2\), temos
    \begin{equation*}
        U(\theta_1, \vetor{n}) U(\theta_2, \vetor{n}) = \unity - \frac12 i(\theta_1 + \theta_2)\vetor{n}\cdot\vetor{\sigma} + O(\theta_1 \theta_2) = U(\theta_1 + \theta_2, \vetor{n})
    \end{equation*}
    sempre que \(\theta_1, \theta_2 \in (-\epsilon, \epsilon)\). Dessa forma, obtemos
    \begin{equation*}
        U(\theta,\vetor{n}) = \lim_{N \to \infty} U\left(\frac{\theta}{N}, \vetor{n}\right)^N = U(\theta,\vetor{n}) = \lim_{N\to \infty} \left[\unity - \frac{i \theta \vetor{n}\cdot \vetor{\sigma}}{2N}\right]^N = \exp\left(-\frac12 i \theta \vetor{n}\cdot\vetor{\sigma}\right),
    \end{equation*}
    como desejado.
\end{proof}
