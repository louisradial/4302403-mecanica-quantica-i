\begin{exercício}{Momento angular orbital de uma partícula}{exercício04}
    O momento angular orbital de uma partícula é descrito pelo estado
    \begin{equation*}
        \Psi(\theta,\varphi) = N\left[\frac{1}{3}Y_0^0(\theta,\varphi) + \frac16 Y_{1}^0 (\theta, \varphi)\right].
    \end{equation*}
    \begin{enumerate}[label=(\alph*)]
        \item Encontre \(N\) para que este estado esteja propriamente normalizado.
        \item Este estado é autovetor de \(\vetor{L}^2\)?
        \item Este estado é autovetor de \(L_z\)?
        \item Qual a probabilidade de uma medida obter \(\ell = 1\)?
        \item Qual a probabilidade de uma medida obter \(m = 0\)?
    \end{enumerate}
\end{exercício}
\begin{proof}[Resolução]
    Para que o estado seja normalizado, temos
    \begin{equation*}
        \abs{N}^{-2} = \int_0^\pi \dli{\theta}\int_0^{2\pi} \sin\theta \dli{\varphi} \left[\frac{1}{3}\conj{Y_0^0}(\theta,\varphi) + \frac16 \conj{Y_{1}^0}(\theta, \varphi)\right]\left[\frac{1}{3}Y_0^0(\theta,\varphi) + \frac16 Y_{1}^0 (\theta, \varphi)\right]= \frac19 + \frac1{36} = \frac{5}{36},
    \end{equation*}
    já que \(\int \dli{\Omega} \conj{Y_{\ell'}^{m'}} Y_\ell^m = \delta_{\ell \ell'} \delta_{m m'}\), portanto podemos tomar \(N = \frac{6}{\sqrt{5}}\). Na base \(\ket{\ell m}\) com \(\vetor{L}^2 \ket{\ell m} = \hbar^2 \ell (\ell + 1) \ket{\ell m}\) e \(L_z \ket{\ell m} \hbar m \ket{\ell m}\), temos
    \begin{equation*}
        \ket{\Psi} = \frac{2}{\sqrt{5}}\ket{00} + \frac1{\sqrt{5}} \ket{10},
    \end{equation*}
    portanto vemos que o estado é autovetor de \(L_z\), com autovalor \(0\), mas que não é autovetor de \(\vetor{L}^2\), e tem probabilidade \(\frac15\) de se medir \(\ell = 1\).
\end{proof}
