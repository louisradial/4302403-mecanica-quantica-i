\begin{exercício}{Potencial delta de Dirac}{exercício3}
    Considere uma partícula cujo hamiltoniano se escreve
    \begin{equation*}
        H = - \frac{\hbar^2}{2m} \diff*[2]{}{x} - \alpha \delta(x),
    \end{equation*}
    onde \(\alpha\) é uma constante positiva.
    \begin{enumerate}[label=(\alph*)]
        \item Integre a equação de autovalores de \(H\) em \([-\varepsilon, \varepsilon]\). Fazendo \(\varepsilon \to 0\), mostre que a derivada da autofunção \(\varphi(x)\) sofre em \(x = 0\) uma descontinuidade que deve ser calculada em função de \(\alpha, m\) e \(\varphi(0)\).
        \item Suponha que a energia \(E\) da partícula é negativa, então
            \begin{equation*}
                \varphi(x) = \begin{cases}
                    A_1 e^{\rho x} + A_1' e^{-\rho x},&\text{se }x < 0\\
                    A_2 e^{\rho x} + A_2' e^{-\rho x},&\text{se }x > 0,
                \end{cases}
            \end{equation*}
            onde \(\rho\) é uma constante que exprimiremos em função de \(E\) e \(m\). Utilizando os resultados do item anterior, determine a matriz \(M\) definida por
            \begin{equation*}
                \begin{pmatrix}
                    A_2\\
                    A_2'
                \end{pmatrix}=
                M
                \begin{pmatrix}
                    A_1\\
                    A_1'
                \end{pmatrix}.
            \end{equation*}
            Use o fato de que \(\varphi(x)\) é uma função de quadrado integrável para deduzir os autovalores da energia. Calcule as funções de onda normalizadas correspondentes.

        \item Represente graficamente essas funções de onda. Forneça a ordem de grandeza de sua largura \(\Delta x\).
        \item Qual a probabilidade \(\overline{\dl{\mathcal{P}(p)}}\) de medir o momento da partícula entre \(p\) e \(p + \dl{p}\) em um dos estados estacionários normalizados calculados acima? Para que valor de \(p\) essa probabilidade é máxima? Em que domínio \(\Delta p\) é importante? Calcule a ordem de grandeza de \(\Delta x \Delta p\).
    \end{enumerate}
\end{exercício}
\begin{proof}[Resolução do item (a)]
    Para um estado estacionário com função de onda \(\braket{x}{\varphi(t)} = \varphi(x)\exp\left(-i \frac{E}{\hbar} t\right)\) temos
    \begin{equation*}
        -\frac{\hbar^2}{2m}\diff[2]{\varphi}{x} - \alpha \delta(x) \varphi(x) = E\varphi(x) \implies \diff[2]{\varphi}{x} + \frac{2m \alpha}{\hbar^2} \delta(x)\varphi(x) + \frac{2m E}{\hbar^2}\varphi(x) = 0
    \end{equation*}
    para todo \(x \in \mathbb{R}\). Integrando em \([-\varepsilon, \varepsilon]\) obtemos
    \begin{equation*}
        \diff{\varphi}{x}[x = \varepsilon] - \diff{\varphi}{x}[x = -\varepsilon] = - \frac{2 m \alpha}{\hbar^2} \varphi(0) - \frac{2m E}{\hbar^2}\int_{-\varepsilon}^\varepsilon \dli{x} \varphi(x).
    \end{equation*}
    Tomando o limite em que \(\varepsilon \to 0\) e utilizando a continuidade da função de onda, segue que
    \begin{equation*}
        \diff{\varphi}{x}[x = 0^+] - \diff{\varphi}{x}[x = 0^-] = -\frac{2m \alpha}{\hbar^2}\varphi(0),
    \end{equation*}
    isto é, a derivada da função de onda sofre uma descontinuidade em \(x = 0\).
\end{proof}

\begin{proof}[Resolução do item (b)]
    Para \(x \in (-\infty, 0)\) e para \(x \in (0, \infty)\), temos
    \begin{equation*}
        \diff[2]{\varphi}{x} + \frac{2m E}{\hbar^2}\varphi(x) = 0,
    \end{equation*}
    donde segue que, para \(E < 0\), temos
    \begin{equation*}
        \varphi(x) = \begin{cases}
            \displaystyle A_1 e^{\rho x} + A_1' e^{-\rho x},&\text{se }x < 0\\
            \displaystyle A_2 e^{\rho x} + A_2' e^{-\rho x},&\text{se }x > 0,
        \end{cases}
    \end{equation*}
    com \(A_1, A_1', A_2, A_2' \in \mathbb{C}\) e \(\rho = \sqrt{\frac{2m\abs{E}}{\hbar^2}}\).

    Como \(\varphi\) é contínua em \(x = 0\), temos
    \begin{equation*}
        A_1 + A_1' = A_2 + A_2' = \varphi(0).
    \end{equation*}
    Da descontinuidade da derivada da função de onda em \(x = 0\), temos
    \begin{equation*}
        \rho A_1 - \rho A_1' - \rho A_2 + \rho A_2' = \frac{2m \alpha}{\hbar^2} \varphi(0).
    \end{equation*}
    Com isso, podemos escrever
    \begin{equation*}
        A_1 - A_1' - A_2 + A_2' = \frac{2m \alpha}{\hbar^2 \rho} (A_1 + A_1') \implies A_2' - A_2 = \left(\frac{2m \alpha}{\hbar^2 \rho} - 1\right)A_1 + \left(1 + \frac{2m \alpha}{\hbar^2 \rho}\right)A_1',
    \end{equation*}
    donde segue que
    \begin{equation*}
        A_2' = \frac{m \alpha}{\hbar \rho} A_1 + \left(1 + \frac{m \alpha}{\hbar^2 \rho}\right)A_1'
    \end{equation*}
    e
    \begin{equation*}
        A_2 = \left(1 - \frac{m \alpha}{\hbar^2 \rho}\right) A_1 - \frac{m \alpha}{\hbar^2 \rho} A_1'.
    \end{equation*}
    Escrevendo em forma de equação matricial, temos
    \begin{equation*}
        \begin{pmatrix}
            A_2\\A_2'
        \end{pmatrix} =
        \begin{pmatrix}
            1 - \frac{m \alpha}{h^2 \rho} & - \frac{m \alpha}{\hbar^2 \rho}\\
            \frac{m \alpha}{\hbar^2 \rho} & 1 + \frac{m \alpha}{\hbar^2 \rho}
        \end{pmatrix}
        \begin{pmatrix}
            A_1\\A_1'
        \end{pmatrix}.
    \end{equation*}
    Como a função de onda deve ser normalizável, ela deve ser limitada, portanto sabemos que as constantes \(A_1'\) e \(A_2\) se anulam. Assim, devemos ter \(\frac{m \alpha}{\hbar^2 \rho} = 1\) e concluímos que há apenas um autoestado ligado, de energia
    \begin{equation*}
        E = - \frac{m \alpha^2}{2\hbar^2}
    \end{equation*}
    e função de onda
    \begin{equation*}
        \varphi(x, t) = A_1 \exp\left(-\frac{m \alpha}{\hbar^2}\abs{x}\right)\exp\left(i\frac{m \alpha^2}{2\hbar^3}t\right).
    \end{equation*}
    Impondo a condição de normalização, temos
    \begin{equation*}
        1 = \int_{\mathbb{R}}\dli{x}\conj{\varphi(x,t)}\varphi(x,t) = 2\abs{A_1}^2 \int_{0}^\infty \dli{x} \exp\left(-\frac{2m \alpha}{\hbar^2}x\right) \implies \abs{A_1} = \sqrt{\frac{m \alpha}{\hbar^2}},
    \end{equation*}
    isto é, a função de onda é dada por
    \begin{equation*}
        \varphi(x, t) = \sqrt{\frac{m \alpha}{\hbar^2}}\exp\left(-\frac{m \alpha}{\hbar^2}\abs{x}\right)\exp\left(i\frac{m \alpha^2}{2\hbar^3}t\right)
    \end{equation*}
    para todo \(x \in \mathbb{R}\).
\end{proof}

\begin{figure}[!ht]
    \centering
    \begin{tikzpicture}
        \begin{axis}[
            width=0.8\linewidth,
            height=0.25\textheight,
            xmin=-5.15, xmax=5.15,
            ymin=0,ymax=1.25,
            domain=-5:5,
            samples=500,
            axis lines=middle,
            xlabel={\(x\)},
            % xlabel near ticks,
            % ylabel near ticks,
            ylabel={\(\varphi(x)\)},
            % legend pos=north east,
            ytick={1},
            xtick={-{3*sqrt(2)},-{2*sqrt(2)}, -{sqrt(2)}, 0, {sqrt(2)}, {2*sqrt(2)}, {3*sqrt(2)}},
            xticklabels={\(-\frac{3\sqrt{2}\hbar^2}{m \alpha}\), \(-\frac{2\sqrt{2}\hbar^2}{m \alpha}\), \(-\frac{\sqrt{2}\hbar^2}{m \alpha}\), \(0\), \(\frac{\sqrt{2}\hbar^2}{m \alpha}\), \(\frac{2\sqrt{2}\hbar^2}{m \alpha}\), \(\frac{3\sqrt{2}\hbar^2}{m \alpha}\)},
            yticklabels={\(\sqrt{\frac{m \alpha}{\hbar^2}}\)},
            ]
            \addplot[thick, Mauve] {exp(-abs(x))};
            \draw[thick, dotted, Pink] ({-sqrt(2)},0) -- ({-sqrt(2)}, {exp(-sqrt(2))});
            \draw[thick, dotted, Pink] ({sqrt(2)}, 0) -- ({sqrt(2)}, {exp(-sqrt(2))});
        \end{axis}
        \end{tikzpicture}
    \caption{Função de onda do autoestado no instante inicial.}
\end{figure}

\begin{proof}[Resolução do item (c)]
    Por paridade temos
    \begin{equation*}
        \mean{x}_{\varphi} = \int_{\mathbb{R}}\dli{x} \conj{\varphi(x,t)} x \varphi(x, t) = \frac{m \alpha}{\hbar^2}\int_{\mathbb{R}} \dli{x} x \exp{\left(-\frac{2m \alpha}{\hbar^2}\abs{x}\right)} = 0.
    \end{equation*}
    Assim, o valor esperado de \(x^2\) é
    \begin{align*}
        \mean{x^2}_{\varphi} = \int_{\mathbb{R}} \dli{x} \conj{\varphi(x,t)} x^2\varphi(x,t)
        &= \frac{m \alpha}{\hbar^2} \int_{\mathbb{R}} \dli{x} x^2 \exp{\left(-\frac{2m \alpha}{\hbar^2}\abs{x}\right)}\\
        &= \left(\frac{\hbar^2}{2m\alpha}\right)^2 \int_0^\infty \dli{\xi} \xi^2e^{-\xi}\\
        &= \left(\frac{\hbar^2}{2m\alpha}\right)^2 \diffp*[2]{\int_0^\infty\dli{\xi} e^{-\beta\xi}}{\beta}[\beta = 1]\\
        &= \left(\frac{\hbar^2}{2m\alpha}\right)^2 \diffp*[2]{\beta^{-1}}{\beta}[\beta = 1]\\
        &= \frac12\left(\frac{\hbar^2}{m\alpha}\right)^2.
    \end{align*}
    Desta forma, o desvio padrão de \(x\) é \(\Delta_\varphi x = \frac{\hbar^2}{m \alpha\sqrt{2}}\).
\end{proof}

\begin{proof}[Resolução do item (d)]
    A distribuição de momento é dada pela transformada de Fourier inversa
    \begin{align*}
        b(k) = \frac{1}{\sqrt{2\pi}} \int_{\mathbb{R}} \dli{x} e^{i k x} \varphi(x)
        &= \sqrt{\frac{m \alpha}{2\pi \hbar^2}}\int_{\mathbb{R}} \dli{x} e^{ikx - \rho\abs{x}}\\
        &= \sqrt{\frac{m \alpha}{2\pi \hbar^2}} \left(\frac{1}{ik + \rho} - \frac{1}{ik - \rho}\right)\\
        &= \sqrt{\frac{m \alpha}{2\pi \hbar^2}}\frac{2\rho}{k^2 + \rho^2}\\
        &= \sqrt{\frac{m \alpha}{2\pi \hbar^2}} \frac{2m \alpha}{(\hbar k)^2 + \left(\frac{m \alpha}{\hbar}\right)^2}.
    \end{align*}
    Assim, a probabilidade de medir o momento da partícula entre \(p\) e \(p + \dli{p}\) é dada por
    \begin{equation*}
        \dl{\mathcal{P}(p)} = \abs{b(k)}^2\dl{k} = \frac{2m^3 \alpha^3}{\pi \hbar^3\left[p^2 + \left(\frac{m \alpha}{\hbar}\right)^2\right]^2} \dl{p} = \frac{2 \hbar}{\pi m \alpha \left[\left(\frac{\hbar p}{m \alpha}\right)^2 + 1\right]^2}\dl{p}.
    \end{equation*}
    Disso, vemos que \(p = 0\) maximiza a densidade probabilidade, já que é uma função par e decresce para todo \(p > 0\). Ainda, como é uma função par, segue que \(\mean{p}_{\varphi} = 0\) e temos
    \begin{align*}
        \mean{p^2}_{\varphi} = \int_{\mathbb{R}} p^2 \dli{\mathcal{P}(p)}
        &= \int_{\mathbb{R}} \dli{p} \frac{\hbar}{m \alpha \pi} \frac{2p^2}{\left[1 + \left(\frac{\hbar p}{m \alpha}\right)^2\right]^2}\\
        &= \frac{m^2 \alpha^2}{\hbar^2\pi} \int_{\mathbb{R}} \dli{\xi} \frac{2\xi^2}{(1 + \xi^2)^2}\\
        &= -\frac{m^2 \alpha^2}{\hbar^2 \pi} \int_{\mathbb{R}} \dli{\xi}\xi \diff[2]{\arctan\xi}{\xi}\\
        &= \frac{m^2 \alpha^2}{\hbar^2},
    \end{align*}
    portanto o desvio padrão do momento é \(\Delta_\varphi p = \frac{m \alpha}{\hbar}\). Concluímos que a relação de incerteza de posição e momento é dada por \(\Delta_\varphi x \Delta_\varphi p = \frac{\hbar}{\sqrt{2}}\), e que o desvio padrão do momento é significativo no domínio de campo forte.
\end{proof}

\begin{figure}[!ht]
    \centering
    \begin{tikzpicture}
        \begin{axis}[
            width=0.8\linewidth,
            height=0.25\textheight,
            xmin=-5.15, xmax=5.15,
            ymin=0,ymax=1.25,
            domain=-5:5,
            samples=500,
            axis lines=middle,
            xlabel={\(p\)},
            % xlabel near ticks,
            % ylabel near ticks,
            ylabel={\(\mathcal{P}(p)\)},
            % legend pos=north east,
            ytick={1},
            xtick={-3,-2,-1,0,1,2,3},
            xticklabels={\(-\frac{3m \alpha}{\hbar}\), \(-\frac{2m \alpha}{\hbar}\), \(-\frac{m \alpha}{\hbar}\), \(0\), \(\frac{m \alpha}{\hbar}\), \(\frac{2m \alpha}{\hbar}\), \(\frac{3m \alpha}{\hbar}\)},
            yticklabels={\(\frac{2\hbar}{\pi m \alpha}\)},
            ]
            \addplot[thick, Mauve] {1/((1 + x^2)^2)};
        \end{axis}
        \end{tikzpicture}
    \caption{Densidade de probabilidade para o momento.}
\end{figure}
