\begin{exercício}{}{exercício2}
    Mil partículas estão confinadas em uma caixa unidimensional, com paredes em \(x = 0\) e \(x = a\). Em \(t = 0\) o estado de cada partícula é representado por
    \begin{equation*}
        \braket{x}{\psi(0)} = \psi(x,0) = A x(x-a).
    \end{equation*}
    \begin{enumerate}[label=(\alph*)]
        \item Normalize \(\ket{\psi}\) encontrando o valor da constante \(A\).
        \item Quais os níveis de energia em que pode ser encontrada uma dessas partículas em \(t = 0\)?
        \item Calcule \(\mean{x}\), \(\mean{p}\) e \(\mean{H}\) em \(t = 0\).
        \item Quantas partículas estão em \(x < \frac{a}2\) em \(t = 0?\).
        \item Quantas partículas têm energia \(E_5\) em \(t = 0?\)
    \end{enumerate}
\end{exercício}
\begin{proof}[Resolução]

\end{proof}
