\begin{exercício}{}{exercício3}
    Considere uma partícula cujo hamiltoniano se escreve
    \begin{equation*}
        H = - \frac{\hbar^2}{2m} \diff*[2]{}{x} - \alpha \delta(x),
    \end{equation*}
    onde \(\alpha\) é uma constante positiva.
    \begin{enumerate}[label=(\alph*)]
        \item Integre a equação de autovalores de \(H\) em \([-\varepsilon, \varepsilon]\). Fazendo \(\varepsilon \to 0\), mostre que a derivada da autofunção \(\varphi(x)\) sofre em \(x = 0\) uma descontinuidade que deve ser calculada em função de \(\alpha, m\) e \(\varphi(0)\).
        \item Suponha que a energia \(E\) da partícula é negativa, então
            \begin{equation*}
                \varphi(x) = \begin{cases}
                    A_1 e^{\rho x} + A_1' e^{-\rho x},&\text{se }x < 0\\
                    A_2 e^{\rho x} + A_2' e^{-\rho x},&\text{se }x > 0,
                \end{cases}
            \end{equation*}
            onde \(\rho\) é uma constante que exprimiremos em função de \(E\) e \(m\). Utilizando os resultados do item anterior, determine a matriz \(M\) definida por
            \begin{equation*}
                \begin{pmatrix}
                    A_2\\
                    A_2'
                \end{pmatrix}=
                M
                \begin{pmatrix}
                    A_1\\
                    A_1'
                \end{pmatrix}.
            \end{equation*}
            Use o fato de que \(\varphi(x)\) é uma função de quadrado integrável para deduzir os autovalores da energia. Calcule as funções de onda normalizadas correspondentes.

        \item Represente graficamente essas funções de onda. Forneça a ordem de grandeza de sua largura \(\Delta x\).
        \item Qual a probabilidade \(\overline{\dl{\mathcal{P}(p)}}\) de medir o momento da partícula entre \(p\) e \(p + \dl{p}\) em um dos estados estacionários normalizados calculados acima? Para que valor de \(p\) essa probabilidade é máxima? Em que domínio \(\Delta p\) é importante? Calcule a ordem de grandeza de \(\Delta x \Delta p\).
    \end{enumerate}
\end{exercício}
\begin{proof}[Resolução]

\end{proof}
