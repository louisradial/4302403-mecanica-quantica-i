\begin{exercício}{Evolução temporal de um sistema de dois níveis}{exercício4}
    Suponha que um estado físico em \(t = 0\) possa ser descrito como
    \begin{equation*}
        \ket{\psi(t = 0)} = \frac{1}{\sqrt{2}}\left(\ket{+} + i \ket{-}\right),
    \end{equation*}
    em que \(\ket+\) e \(\ket{-}\) são os autovetores de um certo operador \(S_z\) associados aos autovalores \(\alpha\) e \(-\alpha\). O hamiltoniano desse sistema tem a forma \(H = - \omega S_z\). Calcule o vetor de estado \(\ket{\psi(t)}\). Qual a probabilidade de medir esses estado físico com o valor de \(S_z\) igual a \(\alpha\)? Qual a probabilidade de que uma medida consecutiva da energia do sistema forneça o valor \(\omega \alpha\)?
\end{exercício}
\begin{proof}[Resolução]

\end{proof}
