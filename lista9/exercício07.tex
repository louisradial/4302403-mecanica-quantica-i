\begin{exercício}{Medidas de um estado do átomo de hidrogênio}{exercício07}
    Um elétron no campo coulombiano de um próton encontra-se num estado descrito pela função de onda
    \begin{equation*}
        \Psi(\vetor{r}) = \frac1{\sqrt{35}}\left[\psi_{311}(\vetor{r}) + 5 \psi_{21-1}(\vetor{r}) + 3\psi_{210}(\vetor{r})\right],
    \end{equation*}
    onde \(\psi_{n\ell m}(\vetor{r}) = R_{n\ell}(r) Y_{\ell m}(\theta, \phi)\) são as autofunções normalizadas da hamiltoniana e \(E_n = - \frac{m e^4}{2\hbar^2 n^2}\) as energias.
    \begin{enumerate}[label=(\alph*)]
        \item \(\Psi(\vetor{r})\) é uma autofunção de \(H\)? Se é, qual é o autovalor correspondente de \(H\)? Se não é, qual é o valor esperado de \(H\)?
        \item \(\Psi(\vetor{r})\) é uma autofunção de \(\vetor{L}^2\)? Se é, qual é o autovalor correspondente de \(\vetor{L}^2\)? Se não é, qual é o valor esperado de \(\vetor{L}^2\)?
        \item \(\Psi(\vetor{r})\) é uma autofunção de \(L_z\)? Se é, qual é o autovalor correspondente de \(L_z\)? Se não é, qual é o valor esperado de \(L_z\)?
        \item Calcule as probabilidades do elétron se encontrar num estado com \(\ell = 1\) e projeções \(m = 1, 0, -1\), respectivamente?
    \end{enumerate}
\end{exercício}
\begin{proof}[Resolução]

\end{proof}
