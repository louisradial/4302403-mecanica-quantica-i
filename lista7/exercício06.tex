
\begin{exercício}{Potencial periódico}{exercício6}
    Uma partícula de massa \(m\) move-se num potencial unidimensional periódico de extensão infinita. O potencial é zero em quase todos os lugares, mas em regiões estreitas de comprimento \(b\) separadas por espaços de comprimento \(a \gg b\) o potencial é \(V_0 > 0\). Podemos escrever
    \begin{equation*}
        V(x) = \sum_{n \in \mathbb{Z}} V_0 b \delta(x - na).
    \end{equation*}
    \begin{enumerate}[label=(\alph*)]
        \item Quais as condições de contorno apropriadas que devemos aplicar às funções de onda? Por quê?
        \item Suponha que a energia mais baixa da onda que pode propagar-se através do potencial seja \(E_0 = \frac{\hbar^2 k_0^2}{2m}\). Escreva a equação transcendental que pode ser resolvida para obter \(k_0\) e, logo, \(E_0\).
        \item Escreva a função de onda para a energia \(E_0\) válida na região \([0,a]\). Escolhemos normalização e fase tal que \(\psi(0) = 1\). O que acontece com a função de onda entre \(x = a\) e \(x = a+b\)?
        \item Mostre que existem intervalos de energia para os quais não existe auto-função. Encontre o valor da energia do início do primeiro desses intervalos.
    \end{enumerate}
\end{exercício}
\begin{proof}[Resolução do item (a)]
    Consideremos o operador
    \begin{equation*}
        D_a \ket{x} = \ket{x + a},
    \end{equation*}
    que é unitário já que é um operador de translação. Dessa forma, seu espectro é contido na circunferência unitária, isto é, é da forma \(e^{i\theta}\), com \(\theta \in \mathbb{R}\). Como \(\delta(x - y)f(x) = \delta(x - y)f(y)\), temos \([D_a, V] = 0\), pois
    \begin{align*}
        [D_a, V]\psi(x) &= V_0 bD_a \sum_{n \in \mathbb{Z}} \psi(na)\delta(x - na) - V_0 b\sum_{n \in \mathbb{Z}} \delta(x - na)\psi(x - a)\\
                        &= V_0 b\sum_{n \in \mathbb{Z}}\psi(na) \delta(x - na - a) - V_0 b\sum_{n \in \mathbb{Z}} \psi(na - a) \delta(x - na)\\
                        &= V_0 b \sum_{n' \in \mathbb{Z}}\psi(n'a - a) \delta(x - n'a) - V_0 b\sum_{n \in \mathbb{Z}} \psi(na - a) \delta(x - na)\\
                        &= 0.
    \end{align*}
    Temos também \([D_a, p] = 0\), pois
    \begin{equation*}
        [D_a, p]\psi(x) = -i\hbar \diff*{\psi(x - a)}{x}  + i\hbar D_a \psi'(x) = i\hbar \psi'(x - a) - i\hbar \psi'(x - a) = 0,
    \end{equation*}
    portanto concluímos que \(\set{D_a, H}\) é um conjunto completo de operadores compatíveis.

    Consideremos os vetores \(\ket{n}\) que representam a partícula completamente localizada em uma das regiões estreitas de comprimento \(b\), de posição \(x = na,\) com \(n \in \mathbb{Z}\). Desta definição segue que \(D_a \ket{n} = \ket{n+1}\), portanto
    \begin{equation*}
        \ket{\theta} = \sum_{n \in \mathbb{Z}} e^{i n \theta} \ket{n}
    \end{equation*}
    é autovetor de \(D_a\), pois
    \begin{equation*}
        D_a \ket{\theta} = \sum_{n \in \mathbb{Z}} e^{i n \theta} \ket{n+1} = e^{-i \theta} \sum_{n' \in \mathbb{Z}} e^{i n'\theta} \ket{n'} = e^{-i\theta} \ket{\theta}.
    \end{equation*}
    Tornemos agora nossa atenção para a função de onda \(\braket{x}{\theta}\), que satisfaz
    \begin{equation*}
        \bra{x}D_a\ket{\theta} = e^{-i\theta} \braket{x}{\theta}\quad\text{e}\quad
        \bra{x}D_a\ket{\theta} = \bra{x}\herm{(D_a^{-1})}\ket{\theta} = \braket{x - a}{\theta},
    \end{equation*}
    então concluímos que \(\braket{x - a}\theta = e^{-i\theta}\braket{x}{\theta}\). Notemos que \(\braket{x}\theta = e^{iqx} u_q(x)\), satisfaz esta relação para uma função \(a\)-periódica \(u_q(x)\) e \(qa = \theta\), pois
    \begin{equation*}
        e^{iq(x - a)} u_q(x - a) = e^{iqx}u_q(x) e^{-iqa}.
    \end{equation*}
    Notemos ainda que se aproximarmos o hamiltoniano para um modelo em que há apenas tunelamento de posições vizinhas com uma energia \(\delta \epsilon\) e energia \(\epsilon\) para permanecer em uma posição, isto é,
    \begin{equation*}
        H\ket{n} = \epsilon \ket{n} - \delta \epsilon \ket{n - 1} - \delta \epsilon \ket{n + 1},
    \end{equation*}
    temos
    \begin{align*}
        H \ket{\theta} &= \sum_{n \in \mathbb{Z}} e^{in\theta} \left[\epsilon \ket{n} - \delta \epsilon \ket{n - 1} - \delta \epsilon \ket{n + 1}\right]\\
                       &= \epsilon \sum_{n \in \mathbb{Z}} e^{in\theta}\ket{n} - \delta \epsilon \left[\sum_{n' \in \mathbb{Z}} e^{i (n' - 1) \theta} \ket{n'} + \sum_{n'' \in \mathbb{Z}} e^{i (n'' + 1)\theta} \ket{n''}\right]\\
                       &= \left[\epsilon  - \delta \epsilon\left(e^{-i \theta} + e^{i\theta}\right)\right]\ket{\theta}\\
                       &= \left(\epsilon - 2\delta \epsilon \cos\theta\right) \ket{\theta},
    \end{align*}
    ou seja, \(\ket{\theta}\) é autovetor simultâneo de \(H\) e de \(D_a\). Baseado na simetria de translações e neste último resultado, buscamos funções de onda que são autoestados simultâneos de \(H\) e de \(D_a\), isto é, devem ser funções \(a\)-períodicas do tipo \(\psi(x) = e^{iqx}u_q(x)\), com \(e^{-iqa}\) autovalor de \(D_a\), satisfazendo \(\psi(x - a) = e^{-iaq}\psi(x)\), resultado conhecido como o teorema de Bloch.
\end{proof}
\begin{lemma}{Espalhamento por potencial delta}{espalhamento_delta}
    A matriz de transmissão para o potencial
    \begin{equation*}
        V(x) = \frac{\hbar^2 g}{2m} \delta(x)
    \end{equation*}
    é dada por
    \begin{equation*}
        M(k) =
        \begin{pmatrix}
            1 + \frac{g}{2ik} && \frac{g}{2ik}\\
            -\frac{g}{2ik} && 1 - \frac{g}{2ik}
        \end{pmatrix}.
    \end{equation*}
\end{lemma}
\begin{proof}
    Uma partícula livre com energia \(E = \frac{\hbar^2k^2}{2m}\) submetida a este potencial tem função de onda dada por
    \begin{equation*}
        \psi(x) = \begin{cases}
            A_1 e^{ikx} + A_2 e^{-ikx},&\text{se }x < 0\\
            B_1 e^{ikx} + B_2 e^{-ikx},&\text{se }x > 0
        \end{cases}.
    \end{equation*}
    Como a função de onda deve ser contínua, devemos ter \(A_1 + A_2 = B_1 + B_2\). Determinamos a descontinuidade da derivada em \(x = 0\) por integração em \([-\varepsilon, \varepsilon]\), obtendo
    \begin{equation*}
        \diff{\psi}{x}[x = \varepsilon] - \diff{\psi}{x}[x = -\varepsilon] = g \psi(0).
    \end{equation*}
    Assim, obtemos \(ik(B_1 - B_2) - ik(A_1 - A_2) = g (A_1 + A_2)\), portanto
    \begin{equation*}
        \begin{pmatrix}
            B_1\\
            B_2
        \end{pmatrix} =
        \begin{pmatrix}
            1 + \frac{g}{2ik} && \frac{g}{2ik}\\
            -\frac{g}{2ik} && 1 - \frac{g}{2ik}
        \end{pmatrix}
        \begin{pmatrix}
            A_1\\
            A_2
        \end{pmatrix}
    \end{equation*}
    é a matriz de transferência desejada.
\end{proof}
\begin{proof}[Resolução do item (b)]
    Pela periodicidade da função de onda, consideramos a função de onda de uma partícula com energia \(E\) entre os sítios \(x = na\) e \(x = (n+1)a\), que escreveremos como
    \begin{equation*}
        \varphi_n(x) = A_n e^{ikx} + B_n e^{-ikx}
    \end{equation*}
    onde \(\hbar k = \sqrt{2m E}\). Pelo espalhamento mostrado no \cref{lem:espalhamento_delta}, temos
    \begin{equation*}
        \begin{pmatrix}
            A_{n+1}\\
            B_{n+1}
        \end{pmatrix} =
        \begin{pmatrix}
            1 + \frac{g}{2ik} && \frac{g}{2ik}\\
            -\frac{g}{2ik} && 1-\frac{g}{2ik}
        \end{pmatrix}
        \begin{pmatrix}
            A_n\\
            B_n
        \end{pmatrix}
        = M(k)
        \begin{pmatrix}
            A_n\\
            B_n
        \end{pmatrix}
    \end{equation*}
    onde definimos \(g = \frac{2m V_0 b}{\hbar^2}\). Pelo teorema de Bloch, devemos ter \(\varphi_{n+1}(x+a) = e^{iqa}\varphi_{n}(x)\), portanto
    \begin{equation*}
        A_{n+1} e^{ika} e^{ikx} + B_{n+1} e^{-ika} e^{-ikx} = e^{iqa}(A_n  e^{ikx} + B_n e^{-ikx}),
    \end{equation*}
    donde segue que
    \begin{equation*}
        e^{iqa}
        \begin{pmatrix}
            A_{n}\\
            B_{n}
        \end{pmatrix} =
        \begin{pmatrix}
            e^{ika} && 0\\
            0 && e^{-ika}
        \end{pmatrix}
        \begin{pmatrix}
            A_{n+1}\\
            B_{n+1}
        \end{pmatrix}
        =
        T(k)
        \begin{pmatrix}
            A_{n+1}\\
            B_{n+1}
        \end{pmatrix}
    \end{equation*}
    pela independência linear de \(\set{e^{ikx}, e^{-ikx}}\). Determinamos assim que \(\begin{psmallmatrix} A_n \\ B_n\end{psmallmatrix}\) é autovetor de \(T(k) M(k)\) com autovalor \(e^{iqa}\). Temos
    \begin{equation*}
        T(k) M(k) = \begin{pmatrix}
            \left(1 + \frac{g}{2ik}\right)e^{ika} && \frac{g}{2ik} e^{ika}\\
            - \frac{g}{2ik}e^{-ika} && \left(1 - \frac{g}{2ik}\right) e^{-ika}
        \end{pmatrix},
    \end{equation*}
    portanto a equação secular é dada por
    \begin{equation*}
        \left[\left(1 + \frac{g}{2ik}\right)e^{ika} - \lambda\right]\left[\conj{\left(1 + \frac{g}{2ik}\right)}e^{-ika} - \lambda\right]- \frac{g^2}{4k^2} = 0,
    \end{equation*}
    isto é,
    \begin{equation*}
        \lambda^2 - 2 \alpha \lambda + 1 = 0,
    \end{equation*}
    onde \(\alpha = \Re\left[\left(1 + \frac{g}{2ik}\right)e^{ika}\right]\). A solução desta equação é
    \begin{equation*}
        \lambda_{\pm} = \alpha \pm \sqrt{\alpha^2 - 1},
    \end{equation*}
    portanto como \(e^{iqa}\) satisfaz esta equação, devemos ter \(\abs{\alpha} \leq 1\), isto é,
    \begin{equation*}
        \lambda_{\pm} = \alpha \pm i \sqrt{1 - \alpha^2},
    \end{equation*}
    e identificamos \(\cos(qa) = \alpha\). Disso concluímos que
    \begin{equation*}
        \cos(qa) = \cos(ka) + \frac{g}{2k} \sin(ka),
    \end{equation*}
    que é a equação que determina as bandas de energia do sistema.
\end{proof}
\begin{proof}[Resolução do item (d)]
    Consideramos a equação que determina as energias do sistema, dada por
    \begin{equation*}
        \cos(qa) = \cos(ka) + \frac{g}{2k} \sin(ka) = \sqrt{1 + \frac{g^2}{4k^2}} \left[\sin(\xi_k)\cos(ka) + \cos(\xi_k) \sin(ka)\right],
    \end{equation*}
    onde definimos \(\xi_k \in [0, 2\pi)\) por
    \begin{equation*}
        \sin(\xi_k) = \frac{1}{\sqrt{1 + \frac{g^2}{4k^2}}}\quad\text{e}\quad \cos(\xi_k) = \frac{g}{\sqrt{4k^2 + g^2}}.
    \end{equation*}
    Desta forma, temos
    \begin{equation*}
        \cos(qa) = \sqrt{1 + \frac{g^2}{4k^2}} \sin(ka + \xi_k)
    \end{equation*}
    e vemos que o lado direito possui valores em \([-1,1]\), mas que não é o caso para o lado direito, que pode exceder 1. Com isso, concluímos que nem todo valor de \(k\) satisfaz esta equação, portanto há valores de energia para os quais não existe auto-função. Ainda, pela continuidade da expressão, devemos ter na verdade intervalos de energia em que esta relação não é satisfeita.
\end{proof}
