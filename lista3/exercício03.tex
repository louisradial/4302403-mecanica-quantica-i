\begin{exercício}{}{exercício3}
    Nos propomos modelar o comportamento dos dois elétrons \(\pi\) da ligação dupla da molécula de formaldeído \ce{H2C=O}. O oxigênio sendo mais eletronegativo que o carbono, discuta por que o hamiltoniano de um elétron toma a forma
    \begin{equation*}
        \begin{pmatrix}
            E_C && -A\\
            -A &&E_O
        \end{pmatrix}
    \end{equation*}
    com \(E_O < E_C\), onde \(E_C\) e \(E_O\) representam a energia de um elétron localizado sobre o átomo de carbono e oxigênio, respectivamente.

    Definimos \(B = \frac12 (E_C - E_O) > 0\) e o ângulo \(\theta\) tal que \(B = \sqrt{A^2 + B^2}\cos\theta\) e \(A = \sqrt{A^2 + B^2}\sin\theta\). Calcule a probabilidade de encontrar um elétron \(\pi\) localizado sobre o átomo de carbono ou oxigênio em função de \(\theta\).
\end{exercício}
\begin{proof}[Resolução]

\end{proof}
