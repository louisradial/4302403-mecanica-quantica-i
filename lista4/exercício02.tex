\begin{lemma}{Igualdade na desigualdade de Cauchy-Schwarz}{cauchy_schwarz_equality}
    Seja \((V, \inner{\noarg}{\noarg})\) um espaço de produto interno. A igualdade \(\abs*{\inner{x}{y}}^2 = \inner{x}{x}\inner{y}{y}\) é satisfeita se e somente se \(\set{x,y} \subset V\) é linearmente dependente.
\end{lemma}
\begin{proof}
    Suponha \(\set{x,y} \subset V\) linearmente dependente. Se um dos vetores é o vetor nulo, então a igualdade é satisfeita trivialmente. Podemos assumir sem perda de generalidade que \(x, y \in V\setminus\set{0}\), então existe \(\alpha \in \mathbb{C}\setminus\set{0}\) tal que \(\alpha x = y\). Assim, temos
    \begin{equation*}
        \abs*{\inner{x}{y}}^2 = \inner{x}{y}\inner{y}{x} = \inner{x}{\alpha x}\inner*{y}{\frac1\alpha y} = \inner{x}{x} \inner{y}{y},
    \end{equation*}
    isto é, a igualdade é satisfeita.

    Suponha que \(\abs*{\inner{x}{y}} = \inner{x}{x} \inner{y}{y}\). Novamente, se algum dos dois vetores é o vetor nulo, então \(\set{x,y}\) é trivialmente linearmente dependente, portanto podemos assumir sem perda de generalidade que \(x, y \in V \setminus\set{0}\). Consideremos
    \begin{equation*}
        \inner*{y - \frac{\inner{x}{y}}{\inner{x}{x}}x}{y - \frac{\inner{x}{y}}{\inner{x}{x}}x} = \inner{y}{y} - \frac{\abs*{\inner{x}{y}}}{\inner{x}{x}}.
    \end{equation*}
    Por hipótese, o lado direito é nulo, portanto pelo produto interno ser positivo definido temos \(y - \frac{\inner{x}{y}}{\inner{x}{x}}x = 0\), logo \(\set{x,y}\) é linearmente dependente.
\end{proof}

\begin{exercício}{Condição necessária e suficiente para autovetor de um operador auto-adjunto}{exercício2}
    Mostre que uma condição necessária e suficiente para que \(\ket\varphi\) seja um autovetor de um operador auto-adjunto \(A\) é que o desvio padrão \(\Delta_\varphi A = 0\).
\end{exercício}
\begin{proof}[Resolução]
    Suponha que \(\ket{\varphi}\) é autovetor do operador auto-adjunto \(A\) associado ao autovalor \(a \in \mathbb{R}\), então
    \begin{align*}
        \Delta_\varphi A &= \bra{\varphi}A^2\ket{\varphi} - \left(\bra{\varphi}A\ket{\varphi}\right)^2\\
                         &= a\bra{\varphi}A\ket{\varphi} - \left(a\braket{\varphi}{\varphi}\right)^2\\
                         &= a^2\braket{\varphi}{\varphi} - a^2 = 0,
    \end{align*}
    dado que \(\braket{\varphi}{\varphi} = 1\).

    Suponha que o desvio padrão do operador auto-adjunto \(A\) no estado \(\ket{\varphi}\) é zero. Como \(A\) é auto-adjunto, temos
    \begin{equation*}
        \bra{\varphi}A^2\ket{\varphi} = \braket{A^\dagger \varphi}{A\varphi} = \braket{A\varphi}{A\varphi} = \norm{A\varphi}^2
    \end{equation*}
    então, por hipótese, obtemos
    \begin{equation*}
        \braket{\varphi}{A\varphi}^2 = \norm{A\varphi}^2 > 0
    \end{equation*}
    e podemos concluir que \(\braket{\varphi}{A \varphi} \in \mathbb{R}\). Suponhamos por contradição que \(\set{\ket{A\varphi}, \ket{\varphi}}\) é linearmente independente, portanto a desigualdade de Cauchy-Schwarz é uma inequação estrita,
    \begin{equation*}
        \norm{A \phi}^2 = \braket{\varphi}{A \varphi}^2 = \abs*{\braket{\varphi}{A \varphi}}^2 < \norm{\varphi}^2 \norm{A \phi}^2 = \norm{A \phi}^2,
    \end{equation*}
    o que é um absurdo. Esta contradição nos mostra que \(\set{\ket{A \varphi}, \ket{\varphi}}\) é linearmente dependente, isto é, existe \(a \in \mathbb{C}\) tal que \(A\ket{\varphi} = a \ket{\varphi}\), portanto \(\ket{\varphi}\) é autovetor de \(A\) associado ao autovalor \(a \in \mathbb{R}\).
\end{proof}
