\begin{exercício}{Modelo grosseiro para um elétron em uma molécula diatômica}{exercício3}
    Podemos modelar de forma bastante aproximada o potencial sentido por um elétron em uma molécula diatômica por
    \begin{equation*}
        V(x) = \frac{\hbar^2 g}{2m}\left[\delta(x + \ell) + \delta(x - \ell)\right]
    \end{equation*}
    com \(g < 0\). A linha que une os núcleos é tomada como eixo \(x\) e os dois núcleos se situam em \(x = -\ell\) e \(x = \ell\).
    \begin{enumerate}[label=(\alph*)]
        \item Mostre que podemos classificar as soluções da equação de Schrödinger em soluções pares e ímpares.
        \item Se a função de onda é par, mostre que existe apenas um único estado ligado para
            \begin{equation*}
                \kappa = \frac{\abs{g}}{2}(1 + e^{-2 \kappa \ell}),
            \end{equation*}
            onde \(\hbar \kappa = \sqrt{2 m \abs{E}}\). Faça um esboço qualitativo da função de onda desse estado ligado.
        \item Se a função de onda é ímpar, determine a equação que fornece as energias dos estados ligados
            \begin{equation*}
                \kappa = \frac{\abs{g}}{2}(1 - e^{-2 \kappa \ell}).
            \end{equation*}
            Sempre existe um estado ligado nesse caso? O que é necessário impor? Faça um esboço qualitativo da função de onda do estado ligado.
        \item Suponha agora que \(\kappa \ell \gg 1\). Mostre que os dois estados ligados constituem um sistema de dois níveis cujo hamiltoniano é dado por \(H = \left(\begin{smallmatrix}
                E_0 & -A\\
                -A & E_0
        \end{smallmatrix}\right)\).
    \end{enumerate}
\end{exercício}
\begin{proof}[Resolução do item (a)]
    Sabemos que para todo \(\alpha \in \mathbb{R}\setminus\set{0}\) vale \(\delta(\alpha z) = \frac{1}{\abs{\alpha}} \delta(z)\), então o potencial \(V\) é uma função par, pois temos
    \begin{equation*}
        V(-x) = \frac{\hbar^2 g}{2m }\left[\delta(\ell - x) + \delta(-x - \ell)\right] = \frac{\hbar^2 g}{2m}\left[\delta(x - \ell) + \delta(x + \ell)\right] = V(x)
    \end{equation*}
    para todo \(x \in \mathbb{R}\). Mostramos no Exercício 4 da Lista 5 que, se o potencial é par, as funções de onda são ou pares ou ímpares. Mostraremos explicitamente que é o caso para este sistema no que segue.
\end{proof}
\begin{proof}[Resolução dos itens (b) e (c)]
    Consideremos uma solução de estado estacionário ligado com energia \(E < 0\) e função de onda dada por \(\varphi(x,t) = \varphi(x) \exp(-i\frac{E}{\hbar}t)\). Assim, temos para todo \(x \in \mathbb{R}\)
    \begin{align*}
        -\frac{\hbar^2}{2m} \diff[2]{\varphi}{x} + V(x)\varphi(x) = E \varphi(x)
        &\implies \diff[2]{\varphi}{x} + \abs{g}\left[\delta(x + \ell) + \delta(x - \ell)\right]\varphi(x) = -\frac{2m}{\hbar^2}E \varphi(x)\\
        &\implies \diff[2]{\varphi}{x} + \left[\abs{g}\delta(x + \ell) + \abs{g}\delta(x - \ell) - \kappa^2\right]\varphi(x) = 0,
    \end{align*}
    onde definimos \(\hbar \kappa = \sqrt{2m \abs{E}}\). Integrando no intervalo \([\xi - \varepsilon, \xi + \varepsilon]\), obtemos
    \begin{align*}
        \diff{\varphi}{x}[x = \xi + \varepsilon] - \diff{\varphi}{x}[x = \xi - \varepsilon]
        &= \int_{\xi - \varepsilon}^{\xi + \varepsilon} \dli{x}\left[\kappa^2 - \abs{g}\delta(x + \ell) - \abs{g}\delta(x - \ell)\right]\varphi(x)\\
        &= \kappa^2 \int_{\xi - \varepsilon}^{\xi + \varepsilon}\dli{x} \varphi(x) - \abs{g}\varphi(\ell) \int_{\xi - \varepsilon}^{\xi + \varepsilon}\dli{x} \delta(x - \ell) - \abs{g}\varphi(-\ell) \int_{\xi - \varepsilon}^{\xi + \varepsilon} \dli{x} \delta(x + \ell).
    \end{align*}
    Assim, tomando o limite \(\varepsilon \to 0\) segue da continuidade de \(\varphi(x)\) que
    \begin{equation*}
        \diff{\varphi}{x}[x = \xi^+] - \diff{\varphi}{x}[x = \xi^-] = \begin{cases}
            0, & \text{se }\xi \notin \set{- \ell, \ell}\\
            -\abs{g}\varphi(\xi),& \text{se }\xi \in\set{- \ell, \ell}
        \end{cases}
    \end{equation*}
    é a descontinuidade da derivada de \(\varphi(x)\) no ponto \(x = \xi\).

    A solução geral da equação diferencial é dada por
    \begin{equation*}
        \varphi(x) = \begin{cases}
            A_1 e^{\kappa x} + A_2 e^{-\kappa x},&\text{se }x \in (-\infty, -\ell)\\
            B_1 e^{\kappa x} + B_2 e^{-\kappa x},&\text{se }x \in (-\ell, \ell)\\
            C_1 e^{\kappa x} + C_2 e^{-\kappa x},&\text{se }x \in (\ell, \infty)
        \end{cases}
    \end{equation*}
    já que em cada um destes intervalos a equação diferencial se torna \(\diff[2]{\varphi}{x} - \kappa^2 \varphi(x) = 0\). Da continuidade da função de onda em \(x = -\ell\) e em \(x = \ell\), temos
    \begin{equation*}
        \varphi(-\ell) = A_1 e^{-\kappa \ell} + A_2 e^{\kappa \ell} = B_1 e^{- \kappa \ell} + B_2 e^{\kappa \ell}
        \quad\text{e}\quad
        \varphi(\ell) = B_1 e^{\kappa \ell} + B_2 e^{-\kappa \ell} = C_1 e^{\kappa \ell} + C_2 e^{-\kappa \ell}.
    \end{equation*}
    Da descontinuidade da derivada da função de onda em \(x = - \ell\) e em \(x = \ell\), temos
    \begin{equation*}
        \frac{\abs{g}\varphi(-\ell)}{\kappa} = A_1 e^{- \kappa \ell} - A_2 e^{\kappa \ell} - B_1 e^{- \kappa \ell} + B_2 e^{\kappa \ell}
        \quad\text{e}\quad
        \frac{\abs{g}\varphi(\ell)}{\kappa} = B_1 e^{\kappa \ell} - B_2 e^{-\kappa \ell} - C_1 e^{\kappa \ell} + C_2 e^{-\kappa \ell}.
    \end{equation*}
    Definindo \(\gamma = e^{2 \kappa \ell}\), podemos reescrever as equações para as condições em \(x = -\ell\) e em \(x = \ell\) como
    \begin{equation*}
        \begin{cases}
            B_1 + \gamma B_2 = A_1 + \gamma A_2\\
            - \kappa B_1 + \kappa\gamma B_2 = (\abs{g} - \kappa)A_1 + (\abs{g} + \kappa)\gamma A_2
        \end{cases}
        \quad\text{e}\quad
        \begin{cases}
            \gamma B_1  + B_2 = \gamma C_1 + C_2\\
            \kappa \gamma B_1 - \kappa B_2 = (\abs{g} + \kappa)\gamma C_1 + (\abs{g} - \kappa)C_2
        \end{cases}
    \end{equation*}
    ou então em forma matricial
    \begin{align*}
        \begin{pmatrix}
            1 & \gamma\\
            -1 & \gamma
        \end{pmatrix}
        \begin{pmatrix}
            B_1\\
            B_2
        \end{pmatrix} &=
        \begin{pmatrix}
            1 & \gamma\\
            \frac{\abs{g} - \kappa}{\kappa} & \frac{\abs{g} + \kappa}{\kappa}\gamma
        \end{pmatrix}
        \begin{pmatrix}
            A_1\\
            A_2
        \end{pmatrix}&
        \begin{pmatrix}
            \gamma & 1\\
            \gamma & -1
        \end{pmatrix}
        \begin{pmatrix}
            B_1\\
            B_2
        \end{pmatrix} &=
        \begin{pmatrix}
            \gamma & 1\\
            \frac{\abs{g} + \kappa}{\kappa}\gamma & \frac{\abs{g} - \kappa}{\kappa}
        \end{pmatrix}
        \begin{pmatrix}
            C_1\\
            C_2
        \end{pmatrix}\\
              &= \left[
                  \begin{pmatrix}
                      1 & \gamma\\
                      -1 & \gamma
                  \end{pmatrix}
                  +
                  \begin{pmatrix}
                      0 & 0\\
                      \frac{\abs{g}}{\kappa} & \frac{\abs{g}}{\kappa}\gamma
              \end{pmatrix}\right]
              \begin{pmatrix}
                  A_1\\
                  A_2
              \end{pmatrix}&
              &= \left[
                  \begin{pmatrix}
                      \gamma & 1\\
                      \gamma & -1
                  \end{pmatrix}
                  +
                  \begin{pmatrix}
                      0 & 0\\
                      \frac{\abs{g}}{\kappa}\gamma & \frac{\abs{g}}{\kappa}
              \end{pmatrix}\right]
              \begin{pmatrix}
                  C_1\\
                  C_2
              \end{pmatrix}.
    \end{align*}
    Resolvendo para \(B_1\) e \(B_2\), obtemos
    \begin{align*}
        \begin{pmatrix}
            B_1\\
            B_2
        \end{pmatrix} &= \left[\unity +
            \frac{\abs{g}}{2 \gamma \kappa}
            \begin{pmatrix}
                \gamma & -\gamma\\
                1 & 1
            \end{pmatrix}
            \begin{pmatrix}
                0 & 0\\
                1 & \gamma
            \end{pmatrix}
        \right]
        \begin{pmatrix}
            A_1\\
            A_2
        \end{pmatrix}&
        \begin{pmatrix}
            B_1\\
            B_2
        \end{pmatrix} &= \left[\unity -
            \frac{\abs{g}}{2 \gamma \kappa}
            \begin{pmatrix}
                -1 & -1\\
                -\gamma & \gamma
            \end{pmatrix}
            \begin{pmatrix}
                0 & 0\\
                \gamma & 1
            \end{pmatrix}
        \right]
        \begin{pmatrix}
            C_1\\
            C_2
        \end{pmatrix}\\
        &= \left[\unity + \frac{\abs{g}}{2 \gamma \kappa}
            \begin{pmatrix}
                -\gamma & - \gamma^2\\
                1 & \gamma
            \end{pmatrix}
        \right]
        \begin{pmatrix}
            A_1\\
            A_2
        \end{pmatrix}&
        &= \left[\unity - \frac{\abs{g}}{2 \kappa \gamma}
            \begin{pmatrix}
                - \gamma & -1\\
                \gamma^2 & \gamma
            \end{pmatrix}
        \right]
        \begin{pmatrix}
            C_1\\
            C_2
        \end{pmatrix}\\
        &=
        \begin{pmatrix}
            1 - \frac{\abs{g}}{2 \kappa} & -\frac{\abs{g} \gamma}{2 \kappa}\\
            \frac{\abs{g}}{2 \kappa \gamma} & 1 + \frac{\abs{g}}{2 \kappa}
        \end{pmatrix}
        \begin{pmatrix}
            A_1\\
            A_2
        \end{pmatrix}&
        &=
        \begin{pmatrix}
            1 + \frac{\abs{g}}{2 \kappa} & \frac{\abs{g}}{2 \kappa \gamma}\\
            -\frac{\abs{g}\gamma}{2 \kappa} & 1 - \frac{\abs{g}}{2 \kappa}
        \end{pmatrix}
        \begin{pmatrix}
            C_1\\
            C_2
        \end{pmatrix}.
    \end{align*}
    A essa altura notamos que se considerássemos um estado de energia \(E > 0\) teríamos o mesmo sistema de equações trocando \(\kappa \to i \kappa\), mas como desejamos funções de quadrado integrável teríamos \(A_1 = A_2 = C_1 = C_2 = 0\), e seguiria que \(B_1 = B_2 = 0\), isto é, haveria apenas a solução trivial. Retornando ao caso \(E < 0\), podemos relacionar \(A_1\) e \(A_2\) com \(C_1\) e \(C_2\) por
    \begin{align*}
        \begin{pmatrix}
            A_1\\
            A_2
        \end{pmatrix}
        &=
        \begin{pmatrix}
            1 + \frac{\abs{g}}{2 \kappa} & \frac{\abs{g} \gamma}{2 \kappa}\\
            - \frac{\abs{g}}{2 \kappa \gamma} & 1 - \frac{\abs{g}}{2 \kappa}
        \end{pmatrix}
        \begin{pmatrix}
            1 + \frac{\abs{g}}{2 \kappa} & \frac{\abs{g}}{2 \kappa \gamma}\\
            -\frac{\abs{g}\gamma}{2 \kappa} & 1 - \frac{\abs{g}}{2 \kappa}
        \end{pmatrix}
        \begin{pmatrix}
            C_1\\
            C_2
        \end{pmatrix}\\
        &=
        \begin{pmatrix}
            \left(1 + \frac{\abs{g}}{2 \kappa}\right)^2 - \frac{\gamma^2 g^2}{4 \kappa^2} & \frac{\gamma \abs{g}}{2 \kappa}\left(1 - \frac{\abs{g}}{2 \kappa}\right) + \frac{\abs{g}}{2 \gamma \kappa}\left(1 + \frac{\abs{g}}{2 \kappa}\right)\\
            -\frac{\gamma \abs{g}}{2 \kappa}\left(1 + \frac{\abs{g}}{2 \kappa}\right) - \frac{\abs{g}}{2 \gamma \kappa}\left(1 - \frac{\abs{g}}{2 \kappa}\right) & \left(1 - \frac{\abs{g}}{2 \kappa}\right)^2 - \frac{g^2}{4 \gamma^2 \kappa^2}
        \end{pmatrix}
        \begin{pmatrix}
            C_1\\
            C_2
        \end{pmatrix}.
    \end{align*}
    Para que a função de onda seja quadrado integrável, devemos ter \(A_2 = C_1 = 0\), de forma que
    \begin{equation*}
        \begin{pmatrix}
            1\\
            0
        \end{pmatrix} A_1 =
        \begin{pmatrix}
            \frac{\gamma \abs{g}}{2 \kappa}\left(1 - \frac{\abs{g}}{2 \kappa}\right) + \frac{\abs{g}}{2 \gamma \kappa}\left(1 + \frac{\abs{g}}{2 \kappa}\right)\\
            \left(1 - \frac{\abs{g}}{2 \kappa}\right)^2 - \frac{g^2}{4 \gamma^2 \kappa^2}
        \end{pmatrix}
        C_2 =
        \begin{pmatrix}
            \frac{\abs{g}}{4 \kappa^2 \gamma}\left[(1+\gamma^2)(2 \kappa - \abs{g}) + 2 \abs{g} \right]\\
            \left(1 - \frac{\abs{g}}{2 \kappa}\right)^2 - \frac{g^2}{4 \gamma^2 \kappa^2}
        \end{pmatrix}C_2
    \end{equation*}
    Portanto,
    \begin{equation*}
        \left(1 - \frac{\abs{g}}{2 \kappa}\right)^2 - \frac{g^2}{4 \gamma^2 \kappa^2} = 0 \implies \left(2\kappa - \abs{g}\right)^2 = \frac{g^2}{\gamma^2} \implies 2 \kappa - \abs{g} = \pm \frac{\abs{g}}{\gamma} \implies \kappa = \frac{\abs{g}}{2}\left(1 \pm e^{-2 \kappa \ell}\right).
    \end{equation*}
    é a equação que determina a energia do estado ligado. Sendo \(\mu \in \set{-1,1}\), temos para estes estados que \(2\kappa - \abs{g} = \mu \frac{\abs{g}}{\gamma}\), então
    \begin{align*}
        A_1 &= \frac{\abs{g}}{4 \kappa^2 \gamma} \left[(1 + \gamma^2)(2 \kappa - \abs{g}) + 2\abs{g})\right]C_2\\
            &= \frac{\abs{g}}{4 \kappa^2 \gamma} \left[(1 + \gamma^2)\frac{\mu\abs{g}}{\gamma} + 2\abs{g}\right]C_2\\
            &= \mu\frac{g^2}{4\kappa^2 \gamma^2}(1 + \gamma^2 + 2 \gamma \mu)C_2\\
            &= \mu\left[\frac{\abs{g}}{2 \kappa \gamma}(\gamma + \mu)\right]^2C_2\\
            &= \mu\left[\frac{\abs{g}}{2\kappa} + \frac{1}{2\kappa}\left(\mu\frac{\abs{g}}{2 \kappa}\right)\right]^2 C_2\\
            &= \mu \left(\frac{\abs{g}}{2 \kappa} + \frac{2\kappa - \abs{g}}{2\kappa}\right)^2 C_2\\
            &= \mu C_2.
    \end{align*}
    Das expressões para \(B_1\) e \(B_2\), obtemos
    \begin{equation*}
        B_1 = \mu\left(1 - \frac{\abs{g}}{2\kappa}\right)C_2\quad\text{e}\quad B_2 = \mu\frac{\abs{g}}{2\kappa \gamma}C_2 = \left(1 - \frac{\abs{g}}{2\kappa}\right)C_2,
    \end{equation*}
    portanto a função de onda é dada por
    \begin{equation*}
        \varphi_\mu(x) = \begin{cases}
            \mu C_2 e^{\kappa x},&\text{se }x \in (-\infty, -\ell)\\
            C_2\left(1 - \frac{\abs{g}}{2\kappa}\right)\left(\mu e^{\kappa x} + e^{-\kappa x}\right),&\text{se }x \in (-\ell, \ell)\\
            C_2 e^{-\kappa x},&\text{se }x \in (\ell, \infty)
        \end{cases}
    \end{equation*}
    e vemos que o caso \(\mu = +1\) corresponde à função de onda par e o caso \(\mu = -1\) corresponde à função de onda ímpar. De fato, temos
    \begin{equation*}
        \varphi_+(x) = \begin{cases}
            C_+e^{\kappa_+ x},&\text{se }x \in (-\infty, -\ell)\\
            C_+\left(1 - \frac{\abs{g}}{2\kappa_+}\right)\left(e^{\kappa_+ x} + e^{-\kappa_+ x}\right),&\text{se }x \in (-\ell, \ell)\\
            C_+e^{-\kappa_+ x},&\text{se }x \in (\ell, \infty)
        \end{cases}
    \end{equation*}
    que verifica \(\varphi_+(-x) = \varphi_+(x)\) para todo \(x \in \mathbb{R}\) e
    \begin{equation*}
        \varphi_-(x) = \begin{cases}
            -C_-e^{\kappa_- x},&\text{se }x \in (-\infty, -\ell)\\
            C_-\left(1 - \frac{\abs{g}}{2\kappa_-}\right)\left(-e^{\kappa_- x} + e^{-\kappa_- x}\right),&\text{se }x \in (-\ell, \ell)\\
            C_-e^{-\kappa_- x},&\text{se }x \in (\ell, \infty)
        \end{cases}
    \end{equation*}
    que verifica \(\varphi_-(-x) = -\varphi_-(x)\) para todo \(x \in \mathbb{R}\). Mostramos, en passant, que há apenas estados ligados com funções de onda par ou ímpar e que as energias são tais que
    \begin{equation*}
        \kappa_+ = \frac{\abs{g}}{2} \left(1 + e^{-2\kappa_+ \ell}\right)
        \quad\text{e}\quad
        \kappa_- = \frac{\abs{g}}{2} \left(1 - e^{-2\kappa_- \ell}\right)
    \end{equation*}
    nos casos par e ímpar, respectivamente.

    \begin{figure}[!ht]
        \centering
        \begin{tikzpicture}
            \def\g{2}
            \def\l{2}
            \def\cimpar{5.129769549}
            \def\kimpar{0.98017259}
            \def\cpar{5.32017221}
            \def\kpar{1.01710443}
            \begin{axis}[
                width=0.8\linewidth,
                height=0.25\textheight,
                xmin=-5.15, xmax=5.15,
                ymin=-0.8,ymax=0.8,
                domain=-6:6,
                samples=500,
                axis lines=middle,
                xlabel={\(x\)},
                xlabel style = {anchor=north east},
                % ylabel near ticks,
                ylabel={\(\varphi(x)\)},
                legend pos=south east,
                ytick=\empty,
                xtick={-2,2},
                xticklabels={-\ell, \ell},
                % yticklabels={},
                ]
                \addplot[thick, Mauve] {x <= -2 ? \cpar*exp(\kpar*x) : (x <= 2 ? \cpar*(2 - 2/\kpar)*cosh(\kpar*x) : \cpar*exp(-\kpar*x))};
                \addlegendentry{\(\varphi_+(x)\)};
                \addplot[thick, Pink] {x <= -2 ? -\cimpar*exp(\kimpar*x) : (x <= 2 ? -\cimpar*(2 - 2/\kimpar)*sinh(\kimpar*x) : \cimpar*exp(-\kimpar*x))};
                \addlegendentry{\(\varphi_-(x)\)};
            \end{axis}
        \end{tikzpicture}
        \caption{Funções de onda dos estados ligados}
    \end{figure}

    Consideremos a função \(f_+ : \mathbb{R} \to \mathbb{R}\) definida por \(f_+(x) = x - \frac12\abs{g}(1 + e^{-2\ell x})\), com \(f_+(0) = - \abs{g} < 0\). Notemos que \(f_+(\abs{g}) = \frac12\abs{g}(1 - e^{-2\ell \abs{g}}) > 0\), portanto pela continuidade da função concluímos pelo teorema do valor intermediário que existe pelo menos um zero da função \(f_+\) no intervalo \((0, \abs{g})\). Como \(f_+'(x) = 1 + \abs{g} \ell e^{-2\ell x} > 1\) é positiva para todo \(x \in \mathbb{R}\), concluímos que \(f_+\) é estritamente crescente, logo injetora. Desse modo, existe um único zero de \(f_+\), isto é, há apenas um estado ligado para a função de onda par e sabemos que \(\kappa_+ \in (0, \abs{g})\).

    Consideremos a função \(f_- : \mathbb{R} \to \mathbb{R}\) definida por \(f_-(x) = x + \frac12\abs{g}(e^{-2\ell x}- 1)\), com \(f_-(0) = 0\). Como \(f'_-(x) = 1 - \abs{g} \ell e^{-2 \ell x}\) e \(f_-''(x) = \abs{g} \ell^2 e^{-2 \ell x} > 0\), sabemos que \(x_* = \frac{1}{2\ell}\ln(\abs{g} \ell)\) é um ponto de mínimo local de \(f_-\), e que \(f_-'\) é estritamente crescente, com \(f_-'\) positiva em \((x_*, \infty)\) e negativa em \((-\infty, x_*)\). Isto é, as restrições de \(f_-\) a esses intervalos,
    \begin{align*}
        h_- : (-\infty, x_*) &\to \mathbb{R} &
        h_+ : (x_*, \infty) &\to \mathbb{R}\\
                          x &\mapsto f_-(x)&
                          x &\mapsto f_-(x),
    \end{align*}
    são funções injetoras, com \(h_-\) estritamente decrescente e \(h_+\) estritamente crescente.

    O caso em que \(x_* = 0\) pode ser desconsiderado, isto é, devemos impor \(\abs{g} \ell \neq 1\). De fato, se \(\abs{g}\ell = 1\), então \(h_-\) e \(h_+\) são funções contínuas e monotônicas com
    \begin{equation*}
        \lim_{x \to 0^+}{h_+(x)} = 0,\quad
        \lim_{x \to 0^-}{h_-(x)} = 0,\quad
        \lim_{x \to +\infty}{h_+(x)} = +\infty,\quad\text{e}\quad
        \lim_{x \to -\infty}{h_-(x)} = +\infty,
    \end{equation*}
    portanto \(h_-\) e \(h_+\) não admitem zero, e concluímos que o único zero de \(f_-\) é 0. Ora, isso implica \(\kappa_- = 0\), e segue que \(E = 0\), contradizendo a hipótese de que \(E < 0\).

    Para \(x_* \neq 0\iff \abs{g} \ell \neq 1\), consideramos \(f_-(x_*) = \frac1{2\ell}[\ln(\abs{g}\ell) + 1 - \abs{g} \ell]\) e mostraremos que \(f_-(x_*) < 0\). Seja então a função auxiliar \(y : (0, \infty) \to \mathbb{R}\) definida por \(y(\xi) = \ln(\xi) + 1 - \xi\), com \(y(1) = 0\). Notemos que \(y'(\xi) = \frac{1}{\xi} - 1\), portanto \(\xi > 1 \implies y'(\xi) < 0\) e \(\xi \in (0,1) \implies y'(\xi) > 0\), isto é, \(y\) é estritamente crescente em \((0,1)\) e estritamente decrescente em \((1, \infty)\). Dessa forma, como \(y\) é contínua e
    \begin{equation*}
        \lim_{\xi \to 0^+}{y(\xi)} = -\infty\quad\text{e}\quad \lim_{\xi \to +\infty}{y(\xi)} = -\infty,
    \end{equation*}
    concluímos que \(\xi = 1\) é o único zero de \(y\), com \(y(\xi) < 0\) sempre que \(\xi \neq 1\). Está mostrado, assim, que \(f_-(x_*) < 0\) sempre que \(x_* \neq 0\). Dessa forma, no caso em que \(x_* > 0\), temos \(0 \in (-\infty, x_*)\), portanto o único zero de \(h_-\) é 0, e concluímos que \(h_+\) admite um único zero em \((x_*, \infty)\), por ser uma função injetora e contínua. Pelo mesmo argumento, concluímos que no caso em que \(x_* < 0\) a função injetora \(h_-\) admite um único zero em \((-\infty, x_*)\) e o único zero de \(h_+\) é 0. Em resumo, mostramos que se \(\abs{g}\ell \neq 1\), então os únicos zeros de \(f_-\) são \(0\) e \(\kappa_-\), com \(\abs{\kappa_-} > x_*\). Isto é, há apenas um estado ligado para a função de onda ímpar.
\end{proof}

\begin{proof}[Resolução do item (d)]
    No caso em que \(\kappa \ell \gg 1\), escrevamos \(e^{-2 \kappa \ell} = \epsilon\), então as energias dos estados ligados em ordem linear de \(\epsilon\) são dadas por
    \begin{equation*}
        \kappa_\pm = \frac{\abs{g}}{2}\left(1 \pm \epsilon\right) \implies -E_\pm =\frac{\hbar^2 \kappa_{\pm}^2}{2m} = \frac{\hbar^2g^2}{8m} \left[1 \pm \epsilon\right]^2 = \frac{\hbar^2 g^2(1 \pm 2 \epsilon)}{8m}.
    \end{equation*}
    Desse modo, temos
    \begin{align*}
        H &= E_+ \ketbra{\varphi_+}{\varphi_+} + E_- \ketbra{\varphi_-}{\varphi_-} \\
          &= -\frac{\hbar^2g^2}{8m} \left(\ketbra{\varphi_+}{\varphi_+} + \ketbra{\varphi_-}{\varphi_-}\right) - \frac{\hbar^2 g^2 \epsilon}{4m} \left[\ketbra{\varphi_+}{\varphi_+} - \ketbra{\varphi_-}{\varphi_-}\right]\\
          &= E_0 \unity + A \sigma_z,
    \end{align*}
    onde definimos \(E_0 = -\frac{\hbar^2 g^2}{8m}\), \(A = -\frac{\hbar^2 g^2}{4m}\epsilon\) e \(\sigma_z = \ketbra{\varphi_+}{\varphi_+} - \ketbra{\varphi_-}{\varphi_-}\).
    Consideremos agora a base \(\set{\ket{+}, \ket{-}}\) com
    \begin{equation*}
        \ket{+} = \frac{1}{\sqrt{2}} \ket{\varphi_+} + \frac{1}{\sqrt{2}}\ket{\varphi_-}
        \quad\text{e}\quad
        \ket{-} = -\frac{1}{\sqrt{2}} \ket{\varphi_+} + \frac{1}{\sqrt{2}}\ket{\varphi_-},
    \end{equation*}
    então \(\sigma_z \ket{\pm} = -\ket{\mp}\) e
    \begin{equation*}
        H\ket{\pm} = E_0 \ket{\pm} + A \sigma_z \ket{\pm} = E_0 \ket{\pm} - A \ket{\mp}.
    \end{equation*}
    Desse modo, temos
    \begin{equation*}
        \bra{\mp}H\ket{\pm} = -A, \quad\text{e}\quad
        \bra{\pm}H\ket{\pm} = E_0
    \end{equation*}
    isto é,
    \begin{equation*}
        H = \begin{pmatrix}
            E_0 & -A\\
            -A & E_0
        \end{pmatrix}
    \end{equation*}
    é a representação de \(H\) na base \(\set{\ket{+},\ket{-}}\).
\end{proof}
