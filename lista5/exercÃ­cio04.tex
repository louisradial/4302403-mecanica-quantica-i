\begin{exercício}{}{exercício4}
    Considere uma partícula ligada, em uma dimensão.
    \begin{enumerate}[label=(\alph*)]
        \item Mostre que
            \begin{equation*}
                \diff*{\int_{\mathbb{R}}\dli{x}\conj{\psi}(x,t)\psi(x,t)}{t} = 0,
            \end{equation*}
            onde \(\ket{\psi(t)}\) não precisa ser um estado estacionário.
        \item Mostre que, se uma partícula está num estado estacionário em um tempo qualquer \(t_0\), então ela sempre permanecerá num estado estacionário.
        \item Se em \(t = 0\) a função de onda é constante na região \([-a,a]\) e nula em seu complemento, expresse a função de onda completa para um tempo \(t\) qualquer em termos dos autoestados do sistema.
        \item Mostre que em problemas unidimensionais o espectro de energia do estado ligado é sempre não degenerado.
        \item Mostre que a primeira derivada de uma função de onda de um estado estacionário é contínua mesmo que o potencial \(V(x)\) tiver uma descontinuidade finita.
        \item Mostre que para um potencial par tal que \(V(x) = V(-x)\) as autofunões dos estados estacionários da equação de Schrödinger para os estados ligados são ou par ou ímpas sob a troca \(x \mapsto -x\).
    \end{enumerate}
\end{exercício}
\begin{proof}[Resolução]

\end{proof}
