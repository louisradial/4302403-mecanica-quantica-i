\begin{exercício}{Operações matriciais}{exercício2}
    Dadas as matrizes
    \begin{align*}
        \vetor{A} &= \begin{pmatrix}
            -1 && 1 && i\\
            2 && 0 && 3\\
            2i &&-2i&& 2
        \end{pmatrix}&
        \vetor{B} &= \begin{pmatrix}
            2 && 0 &&-i\\
            0 && 1 && 0\\
            i && 3 && 2
        \end{pmatrix},
    \end{align*}
    determine \(\commutator{\vetor{A}}{\vetor{B}}\), \(\vetor{A}^T\), \(\conj{\vetor{A}}\), \(\herm{\vetor{A}}\), \(\Tr(\vetor{B})\), e \(\det(\vetor{B})\). A matriz \(\vetor{B}^{-1}\vetor{A}\) admite inversa?
\end{exercício}
\begin{proof}[Resolução]
    Claramente, a transposta, a conjugada, e a conjugada hermitiana de \(A\) são dadas por
    \begin{align*}
        \vetor{A}^T &= \begin{pmatrix}
            -1 && 2 && 2i\\
            1 && 0 &&-2i\\
            i && 3 && 2
        \end{pmatrix}&
            \conj{\vetor{A}} &= \begin{pmatrix}
            -1 && 1 &&-i\\
            2 && 0 && 3\\
            -2i && 2i&& 2
       \end{pmatrix}&
            \herm{\vetor{A}} &= \begin{pmatrix}
            -1 && 2 &&-2i\\
            1 && 0 && 2i\\
            -i && 3 && 2
       \end{pmatrix}.
    \end{align*}

    O comutador de \(A\) e \(B\) é dado por
    \begin{align*}
        \commutator{\vetor{A}}{\vetor{B}} &= \vetor{A} \vetor{B} - \vetor{B} \vetor{A}\\
                                          &= \begin{pmatrix}
                                                -1 && 1 && i\\
                                                2 && 0 && 3\\
                                                2i &&-2i&& 2
                                            \end{pmatrix}
                                            \cdot
                                            \begin{pmatrix}
                                                2 && 0 &&-i\\
                                                0 && 1 && 0\\
                                                i && 3 && 2
                                            \end{pmatrix}-
                                            \begin{pmatrix}
                                                2 && 0 &&-i\\
                                                0 && 1 && 0\\
                                                i && 3 && 2
                                            \end{pmatrix}
                                            \cdot
                                            \begin{pmatrix}
                                                -1 && 1 && i\\
                                                2 && 0 && 3\\
                                                2i &&-2i&& 2
                                            \end{pmatrix}\\
                                          &= \begin{pmatrix}
                                              -3 && 1+3i && 3i\\
                                              4 + 3i && 9 && 6 - 2i\\
                                              6i && 6 - 2i && 6
                                          \end{pmatrix}-
                                          \begin{pmatrix}
                                              0 && 0 && 0\\
                                              2 && 0 && 3\\
                                              6+3i && -3i && 12
                                          \end{pmatrix}\\
                                          &= \begin{pmatrix}
                                              -3 && 1+3i && 3i\\
                                              2 + 3i && 9 && 3 - 2i\\
                                              -6+3i && 6 + i && -6
                                          \end{pmatrix}.
    \end{align*}

    O traço de \(\Tr(\vetor{B})\) é dado pela soma de seus elementos diagonais em qualquer base escolhida, portanto \(\Tr(\vetor{B}) = 5\). O determinante de \(\vetor{B}\) é
    \begin{equation*}
        \det(\vetor{B}) = \begin{vmatrix}
            2 && -i\\
            i && 2
        \end{vmatrix} = 3,
    \end{equation*}
    logo \(\vetor{B}\) admite inversa, com \(\det(\vetor{B}^{-1}) = \frac13\). O determinante de \(\vetor{A}\) é
    \begin{equation*}
        \det(\vetor{A}) = -2 \begin{vmatrix}
            1 && i\\
            -2i && 2
        \end{vmatrix} - 3 \begin{vmatrix}
            -1 && 1\\
            2i -2i
        \end{vmatrix} = 0,
    \end{equation*}
    logo \(\det(\vetor{B}^{-1}\vetor{A}) = \det(\vetor{B}^{-1})\det(\vetor{A}) = 0\), isto é, \(\vetor{B}^{-1}\vetor{A}\) não admite inversa.
\end{proof}
