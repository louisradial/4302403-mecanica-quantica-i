\begin{exercício}{Momento angular orbital de uma partícula}{exercício04}
    O momento angular orbital de uma partícula é descrito pelo estado
    \begin{equation*}
        \Psi(\theta,\varphi) = N\left[\frac{1}{3}Y_0^0(\theta,\varphi) + \frac16 Y_{1}^0 (\theta, \varphi)\right].
    \end{equation*}
    \begin{enumerate}[label=(\alph*)]
        \item Encontre \(N\) para que este estado esteja propriamente normalizado.
        \item Este estado é autovetor de \(\vetor{L}^2\)?
        \item Este estado é autovetor de \(L_z\)?
        \item Qual a probabilidade de uma medida obter \(\ell = 1\)?
        \item Qual a probabilidade de uma medida obter \(m = 0\)?
    \end{enumerate}
\end{exercício}
\begin{proof}[Resolução]

\end{proof}
