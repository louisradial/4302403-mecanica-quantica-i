\begin{exercício}{Evolução temporal de um estado do átomo de hidrogênio}{exercício01}
    Em \(t = 0\) o átomo de hidrogênio encontra-se no estado
    \begin{equation*}
        \ket{\psi(0)} = \frac1{\sqrt{10}} \left(2 \ket{100} + \ket{210} + \sqrt{2} \ket{211} + \sqrt{3}\ket{21 -1}\right)
    \end{equation*}
    na base \(\ket{n\ell m}\).
    \begin{enumerate}[label=(\alph*)]
        \item Qual o valor esperado da energia desse sistema?
        \item Qual a probabilidade de encontrar o sistema com \(\ell = 1\) e \(m = 1\) como função do tempo?
        \item Qual a probabilidade de encontrar o elétron dentro de um raio de \(\SI{1}{\pico\meter}\) do próton em \(t = 0\)? Uma boa aproximação é aceitável aqui.
        \item Como a função de onda correspondente evolui no tempo? Isto é, como é \(\psi(\vetor{r}, t)\)?
        \item Suponha que uma medida é feita que mostra que \(L = 1\) e \(L_z = 1\). Descreva a função de onda imediatamente após essa medida.
    \end{enumerate}
\end{exercício}
\begin{proof}[Resolução]

\end{proof}
