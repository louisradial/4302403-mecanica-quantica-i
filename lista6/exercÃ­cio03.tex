\begin{exercício}{Modelo grosseiro para um elétron em uma molécula diatômica}{exercício3}
    Podemos modelar de forma bastante aproximada o potencial sentido por um elétron em uma molécula diatômica por
    \begin{equation*}
        V(x) = \frac{\hbar^2 g}{2m}\left[\delta(x + \ell) + \delta(x - \ell)\right]
    \end{equation*}
    com \(g < 0\). A linha que une os núcleos é tomada como eixo \(x\) e os dois núcleos se situam em \(x = -\ell\) e \(x = \ell\).
    \begin{enumerate}[label=(\alph*)]
        \item Mostre que podemos classificar as soluções da equação de Schrödinger em soluções pares e ímpares.
        \item Se a função de onda é par, mostre que existe apenas um único estado ligado para
            \begin{equation*}
                \kappa = \frac{\abs{g}}{2}(1 + e^{-2 \kappa \ell}),
            \end{equation*}
            onde \(\hbar \kappa = \sqrt{2 m \abs{E}}\). Faça um esboço qualitativo da função de onda desse estado ligado.
        \item Se a função de onda é ímpar, determine a equação que fornece as energias dos estados ligados
            \begin{equation*}
                \kappa = \frac{\abs{g}}{2}(1 - e^{-2 \kappa \ell}).
            \end{equation*}
            Sempre existe um estado ligado nesse caso? O que é necessário impor? Faça um esboço qualitativo da função de onda do estado ligado.
        \item Suponha agora que \(\kappa \ell \gg 1\). Mostre que os dois estados ligados constituem um sistema de dois níveis cujo hamiltoniano é dado por \(H = E_0 \unity - A \sigma_1\).
    \end{enumerate}
\end{exercício}
\begin{proof}[Resolução do item (a)]
    Sabemos que para todo \(\alpha \in \mathbb{R}\setminus\set{0}\) vale \(\delta(\alpha z) = \frac{1}{\abs{\alpha}} \delta(z)\), então o potencial \(V\) é uma função par.
\end{proof}
\begin{proof}[Resolução dos itens (b) e (c)]
    Consideremos uma solução de estado estacionário ligado com energia \(E < 0\) e função de onda dada por \(\varphi(x,t) = \varphi(x) \exp(-i\frac{E}{\hbar}t)\). Assim, temos para todo \(x \in \mathbb{R}\)
    \begin{align*}
        -\frac{\hbar^2}{2m} \diff[2]{\varphi}{x} + V(x)\varphi(x) = E \varphi(x)
        &\implies \diff[2]{\varphi}{x} + \abs{g}\left[\delta(x + \ell) + \delta(x - \ell)\right]\varphi(x) = -\frac{2m}{\hbar^2}E \varphi(x)\\
        &\implies \diff[2]{\varphi}{x} + \left[\abs{g}\delta(x + \ell) + \abs{g}\delta(x - \ell) - \kappa^2\right]\varphi(x) = 0,
    \end{align*}
    onde definimos \(\hbar \kappa = \sqrt{2m \abs{E}}\). Integrando no intervalo \([\xi - \varepsilon, \xi + \varepsilon]\), obtemos
    \begin{align*}
        \diff{\varphi}{x}[x = \xi + \varepsilon] - \diff{\varphi}{x}[x = \xi - \varepsilon]
        &= \int_{\xi - \varepsilon}^{\xi + \varepsilon} \dli{x}\left[\kappa^2 - \abs{g}\delta(x + \ell) - \abs{g}\delta(x - \ell)\right]\varphi(x)\\
        &= \kappa^2 \int_{\xi - \varepsilon}^{\xi + \varepsilon}\dli{x} \varphi(x) - \abs{g}\varphi(\ell) \int_{\xi - \varepsilon}^{\xi + \varepsilon}\dli{x} \delta(x - \ell) - \abs{g}\varphi(-\ell) \int_{\xi - \varepsilon}^{\xi + \varepsilon} \dli{x} \delta(x + \ell).
    \end{align*}
    Assim, tomando o limite \(\varepsilon \to 0\) segue da continuidade de \(\varphi(x)\) que
    \begin{equation*}
        \diff{\varphi}{x}[x = \xi^+] - \diff{\varphi}{x}[x = \xi^-] = \begin{cases}
            0, & \text{se }\xi \notin \set{- \ell, \ell}\\
            -\abs{g}\varphi(\xi),& \text{se }\xi \in\set{- \ell, \ell}
        \end{cases}
    \end{equation*}
    é a descontinuidade da derivada de \(\varphi(x)\) no ponto \(x = \xi\).

    A solução geral da equação diferencial é dada por
    \begin{equation*}
        \varphi(x) = \begin{cases}
            A_1 e^{\kappa x} + A_2 e^{-\kappa x},&\text{se }x \in (-\infty, -\ell)\\
            B_1 e^{\kappa x} + B_2 e^{-\kappa x},&\text{se }x \in (-\ell, \ell)\\
            C_1 e^{\kappa x} + C_2 e^{-\kappa x},&\text{se }x \in (\ell, \infty)
        \end{cases}
    \end{equation*}
    já que em cada um destes intervalos a equação diferencial se torna \(\diff[2]{\varphi}{x} - \kappa^2 \varphi(x) = 0\). Da continuidade da função de onda em \(x = -\ell\) e em \(x = \ell\), temos
    \begin{equation*}
        \varphi(-\ell) = A_1 e^{-\kappa \ell} + A_2 e^{\kappa \ell} = B_1 e^{- \kappa \ell} + B_2 e^{\kappa \ell}
        \quad\text{e}\quad
        \varphi(\ell) = B_1 e^{\kappa \ell} + B_2 e^{-\kappa \ell} = C_1 e^{\kappa \ell} + C_2 e^{-\kappa \ell}.
    \end{equation*}
    Da descontinuidade da derivada da função de onda em \(x = - \ell\) e em \(x = \ell\), temos
    \begin{equation*}
        \frac{\abs{g}\varphi(-\ell)}{\kappa} = A_1 e^{- \kappa \ell} - A_2 e^{\kappa \ell} - B_1 e^{- \kappa \ell} + B_2 e^{\kappa \ell}
        \quad\text{e}\quad
        \frac{\abs{g}\varphi(\ell)}{\kappa} = B_1 e^{\kappa \ell} - B_2 e^{-\kappa \ell} - C_1 e^{\kappa \ell} + C_2 e^{-\kappa \ell}.
    \end{equation*}
    Definindo \(\gamma = e^{2 \kappa \ell}\), podemos reescrever as equações para as condições em \(x = -\ell\) e em \(x = \ell\) como
    \begin{equation*}
        \begin{cases}
            B_1 + \gamma B_2 = A_1 + \gamma A_2\\
            - \kappa B_1 + \kappa\gamma B_2 = (\abs{g} - \kappa)A_1 + (\abs{g} + \kappa)\gamma A_2
        \end{cases}
        \quad\text{e}\quad
        \begin{cases}
            \gamma B_1  + B_2 = \gamma C_1 + C_2\\
            \kappa \gamma B_1 - \kappa B_2 = (\abs{g} + \kappa)\gamma C_1 + (\abs{g} - \kappa)C_2
        \end{cases}
    \end{equation*}
    ou então em forma matricial
    \begin{align*}
        \begin{pmatrix}
            1 & \gamma\\
            -1 & \gamma
        \end{pmatrix}
        \begin{pmatrix}
            B_1\\
            B_2
        \end{pmatrix} &=
        \begin{pmatrix}
            1 & \gamma\\
            \frac{\abs{g} - \kappa}{\kappa} & \frac{\abs{g} + \kappa}{\kappa}\gamma
        \end{pmatrix}
        \begin{pmatrix}
            A_1\\
            A_2
        \end{pmatrix}&
        \begin{pmatrix}
            \gamma & 1\\
            \gamma & -1
        \end{pmatrix}
        \begin{pmatrix}
            B_1\\
            B_2
        \end{pmatrix} &=
        \begin{pmatrix}
            \gamma & 1\\
            \frac{\abs{g} + \kappa}{\kappa}\gamma & \frac{\abs{g} - \kappa}{\kappa}
        \end{pmatrix}
        \begin{pmatrix}
            C_1\\
            C_2
        \end{pmatrix}\\
              &= \left[
                  \begin{pmatrix}
                      1 & \gamma\\
                      -1 & \gamma
                  \end{pmatrix}
                  +
                  \begin{pmatrix}
                      0 & 0\\
                      \frac{\abs{g}}{\kappa} & \frac{\abs{g}}{\kappa}\gamma
              \end{pmatrix}\right]
              \begin{pmatrix}
                  A_1\\
                  A_2
              \end{pmatrix}&
              &= \left[
                  \begin{pmatrix}
                      \gamma & 1\\
                      \gamma & -1
                  \end{pmatrix}
                  +
                  \begin{pmatrix}
                      0 & 0\\
                      \frac{\abs{g}}{\kappa}\gamma & \frac{\abs{g}}{\kappa}
              \end{pmatrix}\right]
              \begin{pmatrix}
                  C_1\\
                  C_2
              \end{pmatrix}.
    \end{align*}
    Resolvendo para \(B_1\) e \(B_2\), obtemos
    \begin{align*}
        \begin{pmatrix}
            B_1\\
            B_2
        \end{pmatrix} &= \left[\unity +
            \frac{\abs{g}}{2 \gamma \kappa}
            \begin{pmatrix}
                \gamma & -\gamma\\
                1 & 1
            \end{pmatrix}
            \begin{pmatrix}
                0 & 0\\
                1 & \gamma
            \end{pmatrix}
        \right]
        \begin{pmatrix}
            A_1\\
            A_2
        \end{pmatrix}&
        \begin{pmatrix}
            B_1\\
            B_2
        \end{pmatrix} &= \left[\unity -
            \frac{\abs{g}}{2 \gamma \kappa}
            \begin{pmatrix}
                -1 & -1\\
                -\gamma & \gamma
            \end{pmatrix}
            \begin{pmatrix}
                0 & 0\\
                \gamma & 1
            \end{pmatrix}
        \right]
        \begin{pmatrix}
            C_1\\
            C_2
        \end{pmatrix}\\
        &= \left[\unity + \frac{\abs{g}}{2 \gamma \kappa}
            \begin{pmatrix}
                -\gamma & - \gamma^2\\
                1 & \gamma
            \end{pmatrix}
        \right]
        \begin{pmatrix}
            A_1\\
            A_2
        \end{pmatrix}&
        &= \left[\unity - \frac{\abs{g}}{2 \kappa \gamma}
            \begin{pmatrix}
                - \gamma & -1\\
                \gamma^2 & \gamma
            \end{pmatrix}
        \right]
        \begin{pmatrix}
            C_1\\
            C_2
        \end{pmatrix}\\
        &=
        \begin{pmatrix}
            1 - \frac{\abs{g}}{2 \kappa} & -\frac{\abs{g} \gamma}{2 \kappa}\\
            \frac{\abs{g}}{2 \kappa \gamma} & 1 + \frac{\abs{g}}{2 \kappa}
        \end{pmatrix}
        \begin{pmatrix}
            A_1\\
            A_2
        \end{pmatrix}&
        &=
        \begin{pmatrix}
            1 + \frac{\abs{g}}{2 \kappa} & \frac{\abs{g}}{2 \kappa \gamma}\\
            -\frac{\abs{g}\gamma}{2 \kappa} & 1 - \frac{\abs{g}}{2 \kappa}
        \end{pmatrix}
        \begin{pmatrix}
            C_1\\
            C_2
        \end{pmatrix}.
    \end{align*}
    Podemos assim relacionar \(A_1\) e \(A_2\) com \(C_1\) e \(C_2\) por
    \begin{align*}
        \begin{pmatrix}
            A_1\\
            A_2
        \end{pmatrix}
        &=
        \begin{pmatrix}
            1 + \frac{\abs{g}}{2 \kappa} & \frac{\abs{g} \gamma}{2 \kappa}\\
            - \frac{\abs{g}}{2 \kappa \gamma} & 1 - \frac{\abs{g}}{2 \kappa}
        \end{pmatrix}
        \begin{pmatrix}
            1 + \frac{\abs{g}}{2 \kappa} & \frac{\abs{g}}{2 \kappa \gamma}\\
            -\frac{\abs{g}\gamma}{2 \kappa} & 1 - \frac{\abs{g}}{2 \kappa}
        \end{pmatrix}
        \begin{pmatrix}
            C_1\\
            C_2
        \end{pmatrix}\\
        &=
        \begin{pmatrix}
            \left(1 + \frac{\abs{g}}{2 \kappa}\right)^2 - \frac{\gamma^2 g^2}{4 \kappa^2} & \frac{\gamma \abs{g}}{2 \kappa}\left(1 - \frac{\abs{g}}{2 \kappa}\right) + \frac{\abs{g}}{2 \gamma \kappa}\left(1 + \frac{\abs{g}}{2 \kappa}\right)\\
            -\frac{\gamma \abs{g}}{2 \kappa}\left(1 + \frac{\abs{g}}{2 \kappa}\right) - \frac{\abs{g}}{2 \gamma \kappa}\left(1 - \frac{\abs{g}}{2 \kappa}\right) & \left(1 - \frac{\abs{g}}{2 \kappa}\right)^2 - \frac{g^2}{4 \gamma^2 \kappa^2}
        \end{pmatrix}
        \begin{pmatrix}
            C_1\\
            C_2
        \end{pmatrix}
    \end{align*}
    Para que a função de onda seja quadrado integrável, devemos ter \(A_2 = C_1 = 0\), de forma que
    \begin{equation*}
        \begin{pmatrix}
            1\\
            0
        \end{pmatrix} A_1 =
        \begin{pmatrix}
             \frac{\gamma \abs{g}}{2 \kappa}\left(1 - \frac{\abs{g}}{2 \kappa}\right) + \frac{\abs{g}}{2 \gamma \kappa}\left(1 + \frac{\abs{g}}{2 \kappa}\right)\\
             \left(1 - \frac{\abs{g}}{2 \kappa}\right)^2 - \frac{g^2}{4 \gamma^2 \kappa^2}
        \end{pmatrix}
        C_2 =
        \begin{pmatrix}
            \frac{\abs{g}}{4 \kappa^2 \gamma}\left[(1+\gamma^2)(2 \kappa - \abs{g}) + 2 \abs{g} \right]\\
            \left(1 - \frac{\abs{g}}{2 \kappa}\right)^2 - \frac{g^2}{4 \gamma^2 \kappa^2}
        \end{pmatrix}C_2
    \end{equation*}
    Portanto,
    \begin{equation*}
        \left(1 - \frac{\abs{g}}{2 \kappa}\right)^2 - \frac{g^2}{4 \gamma^2 \kappa^2} = 0 \implies \left(2\kappa - \abs{g}\right)^2 = \frac{g^2}{\gamma^2} \implies 2 \kappa - \abs{g} = \pm \frac{\abs{g}}{\gamma} \implies \kappa = \frac{\abs{g}}{2}\left(1 \pm e^{-2 \kappa \ell}\right).
    \end{equation*}
    é a equação que determina a energia do estado ligado. Sendo \(\mu \in \set{-1,1}\), temos para estes estados que \(2\kappa - \abs{g} = \mu \frac{\abs{g}}{\gamma}\), então
    \begin{align*}
        A_1 &= \frac{\abs{g}}{4 \kappa^2 \gamma} \left[(1 + \gamma^2)(2 \kappa - \abs{g}) + 2\abs{g})\right]C_2\\
            &= \frac{\abs{g}}{4 \kappa^2 \gamma} \left[(1 + \gamma^2)\frac{\mu\abs{g}}{\gamma} + 2\abs{g}\right]C_2\\
            &= \mu\frac{g^2}{4\kappa^2 \gamma^2}(1 + \gamma^2 + 2 \gamma \mu)C_2\\
            &= \mu\left[\frac{\abs{g}}{2 \kappa \gamma}(\gamma + \mu)\right]^2C_2\\
            &= \mu\left[\frac{\abs{g}}{2\kappa} + \frac{1}{2\kappa}\left(\mu\frac{\abs{g}}{2 \kappa}\right)\right]^2 C_2\\
            &= \mu \left(\frac{\abs{g}}{2 \kappa} + \frac{2\kappa - \abs{g}}{2\kappa}\right)^2 C_2\\
            &= \mu C_2.
    \end{align*}
    Das expressões para \(B_1\) e \(B_2\), obtemos
    \begin{equation*}
        B_1 = \mu\left(1 - \frac{\abs{g}}{2\kappa}\right)C_2\quad\text{e}\quad B_2 = \mu\frac{\abs{g}}{2\kappa \gamma}C_2 = \left(1 - \frac{\abs{g}}{2\kappa}\right)C_2,
    \end{equation*}
    portanto a função de onda é dada por
    \begin{equation*}
        \varphi(x) = \begin{cases}
            \mu C_2 e^{\kappa x},&\text{se }x \in (-\infty, -\ell)\\
            C_2\left(1 - \frac{\abs{g}}{2\kappa}\right)\left(\mu e^{\kappa x} + e^{-\kappa x}\right),&\text{se }x \in (-\ell, \ell)\\
            C_2 e^{-\kappa x},&\text{se }x \in (\ell, \infty)
        \end{cases}
    \end{equation*}
    e vemos que o caso \(\mu = +1\) corresponde à função de onda par e o caso \(\mu = -1\) corresponde à função de onda ímpar. Mostramos, en passant, que há apenas estados ligados com funções de onda par ou ímpar e que as energias são tais que
    \begin{equation*}
        \kappa_p = \frac{\abs{g}}{2} \left(1 + e^{-2\kappa_p \ell}\right)
        \quad\text{e}\quad
        \kappa_i = \frac{\abs{g}}{2} \left(1 - e^{-2\kappa_i \ell}\right)
    \end{equation*}
    nos casos par e ímpar, respectivamente.
\end{proof}
