\begin{exercício}{Variância do momento angular em um estado}{exercício05}
    Considere os operadores \(L_\pm = L_x \pm i L_y\), onde \(L_x, L_y\) e \(L_z\) são as componentes do operador momento angular. Suponha que o estado do sistema é dado por
    \begin{equation*}
        \psi = \frac1{\sqrt{6}}\left(Y_2^0 - Y_2^{-1} + 2i Y_2^2\right).
    \end{equation*}
    Obtenha as variâncias \((\Delta L_z)^2\) e \((\Delta \vetor{L}^2)^2\).
\end{exercício}
\begin{proof}[Resolução]
    Na base \(\ket{\ell m}\) com \(\vetor{L}^2 \ket{\ell m} = \hbar^2 \ell (\ell + 1) \ket{\ell m}\) e \(L_z \ket{\ell m} \hbar m \ket{\ell m}\), temos
    \begin{equation*}
        \ket{\psi} = \frac1{\sqrt6} \ket{20} - \frac{1}{\sqrt6}\ket{2-1} + \frac{2i}{\sqrt{6}}\ket{22}
    \end{equation*}
    portanto
    \begin{align*}
        L_z \ket{\psi} &= \frac{\hbar}{\sqrt{6}}\ket{2-1} + \frac{4i\hbar}{\sqrt{6}} \ket{22} \implies \mean{L_z} = \frac{7\hbar}{6},\\
        L_z^2\ket{\psi} &= -\frac{\hbar^2}{\sqrt{6}} \ket{2-1} + \frac{8i\hbar^2}{\sqrt{6}}\ket{22} \implies \mean{L_z^2} = \frac{17\hbar^2}{6}\\
        \vetor{L}^2\ket{\psi} &= 6\hbar^2 \ket{\psi} \implies \Delta \vetor{L}^2 = 0,
    \end{align*}
    e temos \((\Delta L_z)^2 = \frac{53}{36}\hbar^2\).
\end{proof}
