\begin{exercício}{Autoestado simultâneo de \(L^2\) e \(L_z\)}{exercício7}
    Um certo estado \(\ket{\psi}\) é um autoestado de \(L^2\) e de \(L_z\), de forma que
    \begin{equation*}
        L^2\ket{\psi} = l(l+1)\hbar^2 \ket{\psi}\quad\text{e}\quad L_z\ket{\psi} = m\hbar \ket{\psi}.
    \end{equation*}
    Calcule \(\mean{L_x}, \mean{L_x^2}\) e \(\Delta L_x\).
\end{exercício}
\begin{proof}
    Consideramos os operadores \(L_\pm = L_x \pm i L_y\) com
    \begin{equation*}
        L_\pm \ket{lm} = \hbar a_\pm^{lm}\ket{l m\pm1},
    \end{equation*}
    onde \(a_\pm^{lm} = \sqrt{l(l+1) - m(m\pm1)},\)
    portanto
    \begin{equation*}
        L_x\ket{lm} = \frac12 \left(L_+ + L_-\right)\ket{lm} = \frac{\hbar}{2}\left[a_+^{lm}\ket{lm + 1} + a_-^{lm}\ket{lm -1}\right],
    \end{equation*}
    e concluímos que \(\mean{L_x} = 0\), já que \(\braket{l'm'}{lm} = \delta_{l'l}\delta_{m'm}\). Assim, temos
    \begin{equation*}
        L_x^2 \ket{lm} = \frac{\hbar^2}{4}\left[a_+^{lm}a_+^{lm+1}\ket{l m+2} + \left(a_-^{lm}a_+^{lm-1}+a_+^{lm}a_-^{lm+1}\right)\ket{lm} + a_-^{lm}a_-^{lm-1}\ket{lm-2}\right],
    \end{equation*}
    logo
    \begin{align*}
        \mean{L_x^2} &= \frac{\hbar^2}{4}\left(a_-^{lm}a_+^{lm-1}+a_+^{lm}a_-^{lm+1}\right)\\
                     &= \frac{\hbar^2}{4}\left[\sqrt{l(l+1) - m(m-1)}\sqrt{l(l+1) - m(m+1)} - \sqrt{l(l+1) - m(m+1)}\sqrt{l(l+1) - m(m+1)}\right]\\
                     &= \frac{\hbar^2}{4}\left[\sqrt{}\right]
    \end{align*}
\end{proof}
