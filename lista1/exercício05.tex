\section*{Exercício 5}
\begin{exercício}{Exponencial de matriz}{exercício5}
    A exponencial de uma matriz \(M\) é definida pela série
    \begin{equation*}
        \exp(M) = \mathds{1} + \sum_{k = 1}^{\infty} \frac{1}{k!} M^{k},
    \end{equation*}
    onde \(\mathds{1}\) é a matriz identidade.
    \begin{enumerate}[label=(\alph*)]
        \item Determine \(\exp(\theta \sigma_n),\) onde \(\theta \in \mathbb{R}\) e
            \begin{align*}
                \sigma_1 &= \begin{pmatrix}
                    0 & 1\\
                    1 & 0
                \end{pmatrix}&
                \sigma_2 &= \begin{pmatrix}
                    0 & -i\\
                    i & 0
                \end{pmatrix}&
                \sigma_3 &= \begin{pmatrix}
                    1 & 0\\
                    0 & -1
                \end{pmatrix}
            \end{align*}
            são as matrizes de Pauli.
        \item Mostre que se \(M\) é diagonalizável, então \(\det(\exp(M)) = \exp(\Tr(M))\).
        \item Mostre que se as matrizes \(M\) e \(N\) comutam, então \(\exp(M+N) = \exp(M)\exp(N)\).
        \item Mostre que se \(H\) é hermitiana, então \(\exp(iH)\) é unitária.
        \item Mostre que qualquer matriz \(T\) pode ser escrita como uma soma de uma matriz simétrica e uma matriz antissimétrica, como a soma de uma matriz real e uma matriz imaginária, e como uma matriz hermitiana e uma matriz anti-hermitiana.
    \end{enumerate}
\end{exercício}
