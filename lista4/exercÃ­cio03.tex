\begin{exercício}{Divisor de feixes}{exercício3}
    Considere um divisor de feixes -- por exemplo, um espelho semi-transparente para uma onda luminosa ou um cristal de Bragg para neutrons -- que vamos supor sem absorção. Ondas chegam com o mesmo ângulo de incidência do lado esquerdo e direito do separador de feixes com amplitudes respectivas \(A_E\) e \(A_D\). As amplitudes emergentes \(B_E\) e \(B_D\), provenientes tanto de reflexão como de transmissão, são ligadas linearmente com as amplitudes incidentes por
    \begin{equation*}
        \begin{pmatrix}
            B_D \\ B_E
        \end{pmatrix}
        = R' \begin{pmatrix}
            A_D \\ A_E
        \end{pmatrix},\quad\text{com}\quad
        R' = \begin{pmatrix}
            a & b\\c & d.
        \end{pmatrix}
    \end{equation*}
    \begin{enumerate}[label=(\alph*)]
        \item Mostre que quando não há absorção, \(R'\) é unitária. Deduza que \(\det(R') = \exp(i\theta)\).
        \item Como nos interessamos por experimentos que fazem interferir as ondas emergentes, um fator de fase global não tem consequência física e podemos substituir \(R = i \exp\left(-i\frac\theta2\right)R'\) com \(\det(R) = -1\). Vamos super que não há nenhuma absorção pelo divisor de feixes. Deduza que a forma geral de \(R\) é
            \begin{equation*}
                R = \begin{pmatrix}
                    r & \conj{t} \\
                    t & -\conj{r}
                \end{pmatrix},
            \end{equation*}
            com \(\abs{r}^2 + \abs{t}^2 = 1\).
        \item Mostre também que \(R\) pode ser escrito como
            \begin{equation*}
                R = \begin{pmatrix}
                    \abs{r}e^{i\chi} & \abs{t} e^{-i\phi}\\
                    \abs{t}e^{i\phi} & -\abs{r} e^{-i\chi}\\
                \end{pmatrix}.
            \end{equation*}
            Seja \(\delta_D\) diferença de fase entre a onda refletida e a onda transmitida por uma onda vindo da direita, e \(\delta_E\) a mesma diferença de fase para uma onda vindo da esquerda. Mostre que \(\delta_D + \delta_E = \pi + 2n\pi,\) com \(n \in \mathbb{Z}\). No caso em que o divisor de feixe é simétrico, \(\delta_E = \delta_D = \frac\pi2\), qual é a forma de \(R\)? Mostre que
            \begin{equation*}
                R = \frac1{\sqrt{2}}\begin{pmatrix}
                    i & 1\\
                    1 & i
                \end{pmatrix}
                \quad\text{ou}\quad
                R = \frac1{\sqrt{2}}\begin{pmatrix}
                    1 & 1\\
                    1 & -1
                \end{pmatrix}
            \end{equation*}
            no caso simétrico.
    \end{enumerate}
\end{exercício}
\begin{proof}[Resolução do item (a)]
    Sem absorção, a potência incidente, proporcional a \(\abs{A_D}^2 + \abs{A_E}^2\), deve ser igual à potência emergente, proporcional a \(\abs{B_D}^2 + \abs{B_E}^2\), isto é, \(R'\) é uma isometria, portanto unitária. Desse modo, temos \(R'^\dag R' = \unity\). Assim, como \(\det{(A^\dag)} = \conj{(\det{A})}\) para qualquer matriz \(A\), segue que \(\abs{\det{R'}}^2 = 1\). Isto é, existe \(\theta \in [0, 2\pi)\) tal que \(\det{R'} = \exp(i\theta)\).
\end{proof}
\begin{proof}[Resolução do item (b)]
    Tomando \(R = i \exp\left(-i\frac{\theta}{2}\right) R'\), temos \(\det{R} = \left[i \exp\left(-i\frac{\theta}{2}\right)\right]^2\det{R'} = -1\). Escrevamos
    \begin{equation*}
        R = \begin{pmatrix}
            r && z\\
            t && w
        \end{pmatrix}
    \end{equation*}
    então,
    \begin{equation*}
        \herm{R} = \begin{pmatrix}
            \conj{r} && \conj{t}\\
            \conj{z} && \conj{w}
        \end{pmatrix}
    \end{equation*}
    com \(rw - zt = -1\). Como \(R\) é unitária, temos \(\herm{R} = R^{-1}\), isto é,
    \begin{equation*}
        \begin{pmatrix}
            \conj{r} && \conj{t}\\
            \conj{z} && \conj{w}
        \end{pmatrix} =
        \frac{1}{\det{R}}\begin{pmatrix}
            w && -z\\
            -t && r
        \end{pmatrix} \implies
        \begin{cases}
            \conj{r} = -w\\
            \conj{t} = z
        \end{cases}
    \end{equation*}
    portanto
    \begin{equation*}
        R = \begin{pmatrix}
            r && \conj{t}\\
            t && - \conj{r}
        \end{pmatrix}
    \end{equation*}
    com \(-r\conj{r} - t\conj{t} = -1,\) isto é, \(\abs{r}^2 + \abs{t}^2 = 1\).
\end{proof}
\begin{proof}[Resolução do item (c)]
    Como \(r, t \in \mathbb{C}\), existem \(\phi, \chi \in [0, 2\pi)\) tais que \(r = \abs{r}e^{i\chi}\) e \(t = \abs{t}e^{i\phi}\), portanto
    \begin{equation*}
        R = \begin{pmatrix}
            \abs{r}e^{i\chi} && \abs{t}e^{-i\phi}\\
            \abs{t}e^{i\phi} && -\abs{r}e^{-i\chi}
        \end{pmatrix}.
    \end{equation*}
    Consideremos uma onda incidente pela direita com amplitudes emergentes dadas por \(R_D\) à direita e \(T_D\) à esquerda, então
    \begin{equation*}
        \begin{pmatrix}
            R_D\\
            T_D
        \end{pmatrix} =
        R \begin{pmatrix}
            A_D\\0
        \end{pmatrix} =
        \begin{pmatrix}
            \abs{r}e^{i\chi}\\\abs{t} e^{i\phi}
        \end{pmatrix}A_D,
    \end{equation*}
    logo a diferença de fase \(\delta_D\) entre a onda refletida \(R_D\) e a transmitida \(T_D\) é \(\delta_D = \chi - \phi\). De forma análoga, temos para uma onda incidente pela esquerda que
    \begin{equation*}
        \begin{pmatrix}
            T_E\\
            R_E
        \end{pmatrix} =
        R \begin{pmatrix}
            0\\A_E
        \end{pmatrix} =
        \begin{pmatrix}
            \abs{t}e^{-i\phi}\\\abs{r} e^{-i\chi}
        \end{pmatrix}A_E,
    \end{equation*}
    portanto a diferença de fase é \(\delta_E = (2n+1)\pi - \chi + \phi\), para algum \(n \in \mathbb{Z}\). Assim, concluímos que \(\delta_E + \delta_D = (2n + 1)\pi\) com \(n \in \mathbb{Z}\).

    No caso simétrico temos \(\abs{t} = \abs{r}\) e \(\delta_E = \delta_D = \frac{\pi}{2}\), portanto \(\chi = \frac{\pi}{2} + \phi\) e \(\abs{t} = \abs{r} = \frac1{\sqrt{2}}\). Com isso, temos
    \begin{equation*}
        R = \frac{1}{\sqrt{2}}\begin{pmatrix}
            ie^{i\phi} && e^{-i\phi}\\
            e^{i\phi} && ie^{-i\phi}
        \end{pmatrix},
    \end{equation*}
    portanto tomando \(\phi = 0\), obtemos
    \begin{equation*}
        R = \frac{1}{\sqrt{2}}\begin{pmatrix}
            i && 1\\
            1 && i
        \end{pmatrix},
    \end{equation*}
    como desejado.
\end{proof}
