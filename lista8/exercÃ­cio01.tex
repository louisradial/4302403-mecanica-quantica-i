\begin{exercício}{Gerador infinitesimal de translações}{exercício1}
    Mostre que
    \begin{equation*}
        \psi(x + a) = \exp\left(\frac{i a p}{\hbar}\right)\psi(x),
    \end{equation*}
    em que \(p = -i\hbar \diff*{}{x}\) é o operador momento linear e \(a\) é a uma constante real.
\end{exercício}
\begin{proof}[Resolução]
    Seja \(U(\alpha)\) a representação de \((\mathbb{R},+)\) dada pela ação
    \begin{equation*}
        U(\alpha)\psi(x) = \psi(x + \alpha)
    \end{equation*}
    para qualquer \(\alpha \in \mathbb{R}\). Uma translação infinitesimal é dada por
    \begin{equation*}
        U(\epsilon) = \unity + i \epsilon T,
    \end{equation*}
    com \(T\) auto-adjunto, já que \(U\) é unitário. Expandindo \(\psi(x + \epsilon)\), temos
    \begin{equation*}
        \psi(x + \epsilon) = \psi(x) + \epsilon \diff{\psi}{x} = \left(\unity + \epsilon\diff{\psi}{x}\right)\psi(x),
    \end{equation*}
    portanto \(T = -i \diff{}{x}\). Isto é, o gerador infinitesimal de translações é \(\frac1{\hbar}p\) e temos pelo teorema de Stone que
    \begin{equation*}
        U(\alpha) = \exp\left(\frac{i \alpha p}{\hbar}\right)
    \end{equation*}
    para todo \(\alpha \in \mathbb{R}\).
\end{proof}
