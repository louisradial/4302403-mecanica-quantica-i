\begin{exercício}{Valor esperado do momento angular}{exercício08}
    Mostre que a expressão
    \begin{equation*}
        \mean{L^2} = \hbar^2 \ell(\ell + 1)
    \end{equation*}
    pode ser obtida diretamente das condições
    \begin{enumerate}[label=(\alph*)]
        \item os únicos valores possíveis que o memento angular orbital pode assumir em qualquer eixo são \(- \ell \hbar, (-\ell + 1)\hbar, \dots, \ell\hbar;\)
        \item todas as componentes do momento angular orbital são igualmente prováveis.
    \end{enumerate}
\end{exercício}
\begin{proof}[Resolução]

\end{proof}
