\begin{exercício}{Gerador infinitesimal de \(\mathrm{SU}(2)\)}{exercício3}
    Mostre que
    \begin{equation*}
        \exp\left(-\frac12i\theta \vetor{n} \cdot \vetor{\sigma}\right) = \cos\left(\frac{\theta}{2}\right) \unity - i \sin\left(\frac{\theta}{2}\right)\vetor{n}\cdot\vetor{\sigma},
    \end{equation*}
    com \(\theta \geq 0\) e \(\vetor{n}\) um vetor unitário.
\end{exercício}
\begin{proof}[Resolução]
    Seja \(\vetor{\eta} \in \setc{\vetor{x} \in \mathbb{R}^3}{\norm{\vetor{x}}^2 = 1}\), então
    \begin{align*}
        (\vetor \eta \cdot \vetor \sigma)^2 &= \eta_k \eta_\ell \sigma_k \sigma_\ell\\
                                        &= \eta_k \eta_\ell \left(\delta_{k\ell}\unity + i \epsilon_{k\ell m}\sigma_m\right)\\
                                        &= \eta_k \eta_\ell \delta_{k \ell}\unity + i \epsilon_{k\ell m} \eta_k \eta_\ell \sigma_m\\
                                        &= \unity + \frac12 \eta_k \eta_\ell [\sigma_k, \sigma_\ell]\\
                                        &= \unity + \frac12[\vetor{\eta}\cdot \sigma, \vetor{\eta}\cdot\vetor{\sigma}]\\
                                        &= \unity,
    \end{align*}
    onde usamos a bilinearidade e anticomutatividade do comutador.

    Mostremos por indução em \(m \in \mathbb{N}\) que
    \begin{equation*}
        (i{z}\vetor \eta \cdot \vetor \sigma)^{2m} = (-1)^m{z}^{2m}\unity\quad\text{e}\quad(i{z}\vetor \eta \cdot \vetor \sigma)^{2m+1} = (-1)^{m}i{z}^{2m+1}\vetor \eta \cdot \vetor \sigma,
    \end{equation*}
    para todo \({z} \in \mathbb{C}\). Essas igualdades seguem trivialmente para \(m = 0\) e para \(m = 1\) temos
    \begin{equation*}
        (i{z} \vetor \eta\cdot \vetor \sigma)^2 = - {z}^2 \unity\quad\text{e}\quad(i{z}\vetor\eta\cdot\vetor\sigma)^3 = -i{z}^3 \vetor \eta \cdot \vetor \sigma,
    \end{equation*}
    como proposto. Suponhamos que as igualdades sejam satisfeitas para algum \(k \in \mathbb{N}\), então
    \begin{align*}
        (i{z}\vetor \eta \cdot \vetor \sigma)^{2k+2} &= (i{z}\vetor \eta \cdot \vetor \sigma)(i{z}\vetor \eta \cdot \vetor \sigma)^{2k+1}&
        (i{z}\vetor \eta \cdot \vetor \sigma)^{2k+3} &= (i{z}\vetor \eta \cdot \vetor \sigma)(i{z}\vetor \eta \cdot \vetor \sigma)^{2k+2}\\
                                                     &= (i{z}\vetor \eta \cdot \vetor \sigma)\left[(-1)^{k+1}{z}^{2k+2} \unity\right]&
                                                    &= (i{z}\vetor \eta \cdot \vetor \sigma)\left[(-1)^ki{z}^{2k+1}\vetor \eta\cdot \vetor \sigma\right]\\
                                                    &= (-1)^{k+1}{z}^{2k+2} \unity&
                                                    &= (-1)^{k+1}i{z}^{2k+3}\vetor \eta \cdot \vetor \sigma,
    \end{align*}
    isto é, as igualdades são satisfeitas por \(k + 1\). Pelo princípio da indução finita, são válidas para todo \(m \in \mathbb{N}\).

    Assim, para todo \({z} \in \mathbb{C}\), temos
    \begin{align*}
        \exp(i{z} \vetor \eta \cdot \vetor \sigma) &= \sum_{\ell = 0}^{\infty} \frac{(i {z} \vetor \eta \cdot \vetor \sigma)^n}{n!}\\
                                                  &= \left[\sum_{m = 0}^{\infty} \frac{(-1)^{m}{z}^{2m}}{(2m)!}\right] \unity + \left[\sum_{m=0}^\infty\frac{(-1)^m {z}^{2m+1}}{(2m+1)!}\right] i \vetor \eta \cdot \vetor \sigma\\
                                                  &= (\cos{z})\unity + (i\sin{z})(\vetor \eta \cdot \vetor \sigma),
    \end{align*}
    como desejado.
\end{proof}
