\begin{exercício}{Partícula em trajetória circular}{exercício4}
    Considere uma partícula de massa \(\mu\) sujeita a uma trajetória circular de raio constante \(r\). A energia potencial será considerada nula em todos os pontos da trajetória.
    \begin{enumerate}[label=(\alph*)]
        \item Calcule a expressão clássica para a energia cinética da partícula em função de seu momento angular orbital \(\vetor{L}\).
        \item Determine a expressão das componentes de \(\vetor{L}\) em coordenadas cartesianas.
        \item Escreva o operador de momento angular \(L_z\) em coordenadas cartesianas.
        \item Calcule a expressão de \(L_z\) em coordenadas polares do plano de rotação e determine suas autofunções.
        \item Determine a expressão do hamiltoniano \(H\) da partícula em rotação.
        \item Mostre que \(L_z\) comuta com \(H\). O que se pode então deduzir das autofunções de \(H\)?
    \end{enumerate}
\end{exercício}
\begin{proof}[Resolução]
    Classicamente, a energia cinética de uma partícula de massa \(\mu\) é dada por
    \begin{equation*}
        T = \frac{\mu \dot{r}^2}{2} + \frac{L^2}{2\mu r^2},
    \end{equation*}
    onde \(L\) é o seu momento angular orbital. Em coordenadas cartesianas, temos
    \begin{equation*}
        \vetor{L} = \vetor{r} \times \vetor{p} = x_i \vetor{e}_i \times p_j \vetor{e}_j = x_i  p_j \epsilon_{ijk}\vetor{e}_k,
    \end{equation*}
    com \(L_z = x p_y - y p_x\).

    Se a trajetória é planar, temos \(x = r\cos\varphi\) e \(y = r\sin\varphi\), logo \(r = \sqrt{x^2 + y^2}\) e \(\tan\varphi = \frac{y}{x}\). Assim,
    \begin{align*}
        \frac1{i\hbar}L_z = y \diffp{}{x} - x \diffp{}{y} &= y \left(\diffp{r}{x} \diffp{}{r} + \diffp{\varphi}{x}\diffp{}{\varphi}\right) - x \left(\diffp{r}{y} \diffp{}{r} + \diffp{\varphi}{y}\diffp{}{\varphi}\right)\\
                                                          &= y \left(\frac{x}{r}\diffp{}{r} - \frac{y}{x^2\sec^2\varphi} \diffp{}{\varphi}\right) - x \left(\frac{y}{r} \diffp{}{r} + \frac{1}{x\sec^2\varphi}\diffp{}{\varphi}\right)\\
                                                          &= -\left(1 + \frac{y^2}{x^2}\right)\cos^2\varphi \diffp{}{\varphi}\\
                                                          &= - \diffp{}{\varphi},
    \end{align*}
    isto é, \(L_z = -i \hbar \diffp{}{\varphi}\) é a expressão para a componente \(z\) do momento angular para uma trajetória planar. Consideremos uma autofunção \(\Phi_m(\varphi)\) de \(L_z\) associada ao autovalor \(\hbar m\), então
    \begin{equation*}
        \diff{\Phi_m}{\varphi} + i m \Phi_m(\varphi) = 0,
    \end{equation*}
    logo \(\Phi_m(\varphi) = \frac{1}{\sqrt{2\pi}} e^{-i m \varphi}\) e devemos ter \(m\in \mathbb{Z}\) para que \(\Phi_m\) seja \(2\pi\)-periódica.

    As demais componentes do momento angular são nulas pois não há variação em \(z\), e o plano de rotação pode ser tomado como \(z = 0\). Desta forma, o hamiltoniano de uma partícula em rotação no plano é dado por
    \begin{equation*}
        H = \frac{1}{2\mu r^2} L^2 = -\frac{\hbar^2}{2\mu r^2} \diffp[2]{}{\varphi},
    \end{equation*}
    logo \([L_z,H] = 0\), já que \(L^2 = L_z^2\). Assim, as autofunções de \(H\) são dadas por combinações lineares \(\alpha \Phi_{-\ell} + \beta \Phi_{\ell}\) e são associadas ao autovalor \(\frac{\hbar^2 \ell^2}{2\mu r^2}\).
\end{proof}
