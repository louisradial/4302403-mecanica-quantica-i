\begin{exercício}{Relação de anticomutação}{exercício4}
    Suponha que os operadores \(a\) e \(\herm{a}\) em
    \begin{equation*}
        H = \hbar \omega \left(\herm{a}a + \frac12\unity\right)
    \end{equation*}
    obedecem a relação de anticomutação \(\anticommutator{a}{\herm{a}} = \unity\).
    \begin{enumerate}[label=(\alph*)]
        \item Quais os valores de \(a\ket{n}\) e \(\herm{a}\ket{n}\) que seguem dessa relação de anticomutação?
        \item Como \(\mean{H} \geq 0\), podemos tomar por consistência \(a\ket{0} = 0\). Combinando esse fato com os resultados de (a), quais os únicos estados \(\ket{n}\) não nulos?
        \item Se além da relação de anticomutação, \(a\) e \(\herm{a}\) também obedecem \(\anticommutator{a}{a} = \anticommutator{\herm{a}}{\herm{a}} = 0\), mostre que \(N^2 = N\).
    \end{enumerate}
\end{exercício}
\begin{proof}[Resolução]
    Consideremos o operador auto-adjunto \(N = \herm{a}a\), então o espectro de \(N\) é não-negativo. De fato, se \(\lambda\) é um autovalor associado ao autovetor \(\ket{\lambda}\), temos
    \begin{equation*}
        0 \leq \norm{a \lambda}^2 = \bra{\lambda}\herm{a}a\ket{\lambda} = \bra{\lambda}N\ket{\lambda} = \lambda,
    \end{equation*}
    e concluímos ainda que, se \(0\) é autovalor de \(N\), então \(a \ket{0} = 0\), e para todo \(\lambda > 0\), \(a\ket{0} \neq 0\). Notemos ainda que
    \begin{equation*}
        \anticommutator{N}{a} = \herm{a}aa - a\herm{a}a = \anticommutator{\herm{a}}{a}a = a,
    \end{equation*}
    portanto para \(\lambda > 0\), temos
    \begin{equation*}
        N a\ket{\lambda} = \anticommutator{N}{a} \ket{\lambda} - a N \ket{\lambda} = (1 - \lambda)a \ket{\lambda},
    \end{equation*}
    logo \(a \ket{\lambda}\) é autovetor de \(N\) associado ao autovalor \(1 - \lambda \geq 0\), e concluímos que o espectro de \(N\) é contido em \([0,1]\). Semelhantemente, temos
    \begin{equation*}
        0 \leq \norm{\herm{a}\lambda}^2 = \bra{\lambda}a\herm{a}\ket{\lambda} = \bra{\lambda}\unity - \herm{a}a\ket{\lambda} = 1 - \lambda,
    \end{equation*}
    portanto se \(1\) é autovalor de \(N\), então \(\herm{a}\ket{1} = 0\). Para \(\lambda\neq 1\), temos
    \begin{equation*}
        N\herm{a}\ket{\lambda} = \anticommutator{N}{\herm{a}}\ket{\lambda} - \herm{a} N \ket{\lambda} = (1 - \lambda)\herm{a}\ket{\lambda},
    \end{equation*}
    isto é, \(\herm{a}\ket{\lambda}\) é autovalor de \(N\) associado ao autovalor \(1 - \lambda \geq 0\). Em resumo, encontramos que
    \begin{equation*}
        a\ket{\lambda} = \sqrt{\lambda} \ket{1-\lambda}\quad\text{e}\quad \herm{a}\ket{\lambda} = \sqrt{1 - \lambda} \ket{1 - \lambda}
    \end{equation*}
    para todo \(\lambda\) no espectro de \(N\).

    Se um operador \(u\) satisfaz \(\anticommutator{u}{u} = 0\), segue que \(u\) é nilpotente com \(u^2 = 0\). Desse modo, temos
    \begin{equation*}
        N^2 = (\herm{a}a)^2 = \herm{a}a\herm{a}a = \herm{a}\left(\unity - \herm{a}a\right)a = \herm{a}a - (\herm{a})^2a^2 = \herm{a}a = N,
    \end{equation*}
    como desejado. Agora os únicos autovalores possíveis são \(0\) e \(1\), portanto temos
    \begin{equation*}
        a\ket{0} = 0,\quad a\ket{1} = \ket{0}, \quad \herm{a}\ket{0} = \ket{1},\quad\text{e}\quad \herm{a}\ket{1} = 0
    \end{equation*}
    pela idempotência de \(N\).
\end{proof}
