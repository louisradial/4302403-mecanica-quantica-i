% \section*{Exercício 3}
\begin{lemma}{Regra de Leibniz para comutadores}{leibniz}
    Sejam \(X, Y\) operadores lineares que dependem suavemente de um parâmetro \(t \in \mathbb{R}\). Então
    \begin{equation*}
        \diff*{\commutator{X(t)}{Y(t)}}{t} = \commutator*{\diff{X}{t}}{Y(t)} + \commutator*{X(t)}{\diff{Y}{t}}
    \end{equation*}
    para todo \(t\).
\end{lemma}
\begin{proof}
    Temos
    \begin{align*}
        \diff*{\commutator{X(t)}{Y(t)}}{t} &= \diff*{\left(X(t)Y(t) - Y(t)X(t)\right)}{t}\\
                                           % &= \diff{X}{t} Y(t) + X(t)\diff{Y}{t} - \diff{Y}{t} X(t) - Y(t)\diff{X}{t}\\
                                           &= \left(\diff{X}{t} Y(t) - Y(t) \diff{X}{t}\right) + \left(X(t) \diff{Y}{t} - \diff{Y}{t} X(t)\right)\\
                                           &= \commutator*{\diff{X}{t}}{Y(t)} + \commutator*{X(t)}{\diff{Y}{t}}
    \end{align*}
    como desejado.
\end{proof}

\begin{lemma}{Derivada de um comutador iterado}{iterado}
    Seja \(X\) um operador linear e seja \(Y(t)\) um operador linear que depende suavemente de um parâmetro real. Denotemos o comutador iterado por
    \begin{equation*}
        \commutator{X}{Y(t)}^{[1]} = \commutator{X}{Y(t)}
        \quad\text{e}\quad
        \commutator{X}{Y(t)}^{[n]} = \commutator*{X}{\commutator{X}{Y(t)}^{[n-1]}},
    \end{equation*}
    onde \(n \in \mathbb{N} \setminus \set{1}\). Então
    \begin{equation*}
        \diff*{\commutator{X}{Y(t)}^{[n]}}{t} = \commutator*{X}{\diff{Y}{t}}^{[n]}
    \end{equation*}
    para todo \(n \in \mathbb{N}\).
\end{lemma}
\begin{proof}
    Pelo \cref{lem:leibniz}, segue que \(\diff*{\commutator{X}{Y(t)}}{t} = \commutator*{X}{\diff{Y}{t}}\), visto que \(X\) não depende de \(t\). Suponhamos que \(\diff*{\commutator{X}{Y(t)}^{[k]}}{t} = \commutator*{X}{\diff{Y}{t}}^{[k]}\) para algum \(k \in \mathbb{N}\), então
    \begin{equation*}
        \diff*{\commutator{X}{Y(t)}^{[k+1]}}{t} = \commutator*{X}{\diff*{\commutator{X}{Y(t)}^{[k]}}{t}} = \commutator*{X}{\commutator*{X}{\diff{Y}{t}}^{[k]}} = \commutator*{X}{\diff{Y}{t}}^{[k+1]}.
    \end{equation*}
    Concluímos a demonstração ao invocar o princípio da indução finita.
\end{proof}

\begin{lemma}{Operador comuta com a sua exponencial}{comuta}
    Seja \(X\) um operador linear, então
    \begin{equation*}
        [X, \exp(\lambda X)] = 0
    \end{equation*}
    para todo \(\lambda \in \mathbb{C}\).
\end{lemma}
\begin{proof}
    É claro que \(X\) comuta com suas potências pela associatividade do produto, logo
    \begin{equation*}
        X \exp(\lambda X) = X \left(\sum_{k=0}^\infty \frac{\lambda^k}{k!}X^k\right) = \sum_{k=0}^\infty \frac{\lambda^k}{k!}X^{k+1} = \left(\sum_{k=0}^\infty \frac{\lambda^k}{k!}X^k\right)X = \exp(\lambda X) X,
    \end{equation*}
    como desejado.
\end{proof}
\setcounter{tcb@cnt@exercício}{2}
\begin{exercício}{Expansão em série de um operador}{exercício3}
    Seja \(f(t)\) um operador dado por
    \begin{equation*}
        f(t) = \exp(tA) B\exp(-tA),
    \end{equation*}
    onde \(A\) e \(B\) são operadores em um espaço linear de dimensão finita. Mostre que
    \begin{equation*}
        \diff[n]{f}{t} = \commutator{A}{f(t)}^{[n]}
    \end{equation*}
    para todo \(n \in \mathbb{N}\) e conclua que
    \begin{equation*}
        \exp(tA)B\exp(-tA) = B + \sum_{n = 1}^\infty \frac{t^n}{n!}\commutator{A}{B}^{[n]}.
    \end{equation*}
\end{exercício}
\begin{proof}[Resolução]
    Pelo \cref{lem:comuta}, obtemos
    \begin{equation*}
        \diff{f}{t} = A \exp(tA) B\exp(-tA) - \exp(tA) B\exp(-tA) A = \commutator{A}{f(t)}
    \end{equation*}
    para todo \(t\). Suponhamos que \(\diff[k]{f}{t} = \commutator{A}{f(t)}^{[k]}\) para algum natural \(k > 1\), então
    \begin{equation*}
        \diff[k+1]{f}{t} = \diff*{\commutator{A}{f(t)}^{[k]}}{t} = \commutator*{A}{\diff{f}{t}}^{[k]} = \commutator*{A}{\commutator{A}{f(t)}}^{[k]} = \commutator{A}{f(t)}^{[k+1]},
    \end{equation*}
    onde utilizamos o \cref{lem:iterado}. Pelo princípio da indução finita, segue que
    \begin{equation*}
        \diff[n]{f}{t} = \commutator{A}{f(t)}^{[n]}
    \end{equation*}
    para todo \(n \in \mathbb{N}\). Desse modo, a série de Taylor em torno de \(t = 0\) para \(f(t)\) é
    \begin{equation*}
        \exp(tA)B\exp(-tA) = B + \sum_{n = 1}^{\infty} \frac{t^n}{n!}\diff[n]{f}{t}[t=0] = B + \sum_{n = 1}^{\infty} \frac{t^n}{n!}\commutator{A}{B}^{[n]},
    \end{equation*}
    como desejado.
\end{proof}
