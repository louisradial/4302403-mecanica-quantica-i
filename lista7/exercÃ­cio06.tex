\begin{exercício}{Potencial periódico}{exercício6}
    Uma partícula de massa \(m\) move-se num potencial unidimensional periódico de extensão infinita. O potencial é zero em quase todos os lugares, mas em regiões estreitas de comprimento \(b\) separadas por espaços de comprimento \(a \gg b\) o potencial é \(V_0 > 0\). Podemos escrever
    \begin{equation*}
        V(x) = \sum_{n \in \mathbb{Z}} V_0 b \delta(x - na).
    \end{equation*}
    \begin{enumerate}[label=(\alph*)]
        \item Quais as condições de contorno apropriadas que devemos aplicar às funções de onda? Por quê?
        \item Suponha que a energia mais baixa da onda que pode propagar-se através do potencial seja \(E_0 = \frac{\hbar^2 k_0^2}{2m}\). Escreva a equação transcendental que pode ser resolvida para obter \(k_0\) e, logo, \(E_0\).
        \item Escreva a função de onda para a energia \(E_0\) válida na região \([0,a]\). Escolhemos normalização e fase tal que \(\psi(0) = 1\). O que acontece com a função de onda entre \(x = a\) e \(x = a+b\)?
        \item Mostre que existem intervalos de energia para os quais não existe auto-função. Encontre o valor da energia do início do primeiro desses intervalos.
    \end{enumerate}
\end{exercício}
\begin{proof}[Resolução]

\end{proof}
