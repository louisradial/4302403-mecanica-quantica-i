\todo[Considerar o resto do enunciado.]
\begin{exercício}{Poços quânticos com semicondutores}{exercício5}
    Considere um elétron localizado em um poço de potencial de largura \(2a\) e altura \(V_0 > 0\). Estudaremos os estados ligados de energia \(E\), tal que \(E \in [0, V_0].\)
    \begin{enumerate}[label=(\alph*)]
        \item Determine a função de onda \(\Psi(x)\) do elétron.
        \item Utilizando as condições de continuidade da função de onda, determine as equações que permitem calcular os valores das energias.
        \item Como podemos resolver essas equações?
        \item Para que valor máximo de \(V_0\) existe apenas um estado possível?
        \item Para um poço de largura igual a \SI{5}{\nano\meter}, calcule, em elétron-volts, o valor máximo de \(V_0\) que permite um único estado. Determine a separação máxima entre os dois primeiros níveis. Compare esse valor ao salto de banda de potencial entre \ce{GaAs} e \(\ce{AlAs}\).
        \item Sabendo que o \textit{gap} do \ce{Al_yGa_{1-y}As} varia entre o do \ce{GaAs}, \(y = 0\), e o do \ce{AlAs}, \(y=1\), proporcionalmente à concentração \(y\) em alumínio, calcule \(y\) para que tenhamos ao menos um estado ligado possível dentro do poço de \SI{5}{\nano\meter} de largura.
    \end{enumerate}
\end{exercício}
\begin{proof}[Resolução]

\end{proof}
