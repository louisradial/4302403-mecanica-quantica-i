\begin{exercício}{Autovalores e projetores ortogonais de operadores}{exercício5}
    Considere os operadores cujas matrizes em uma base ortonormal dada se escrevem
    \begin{equation*}
        \sigma_y = \begin{pmatrix}
            0 && -i\\
            i && 0
        \end{pmatrix},
        \quad
        M = \begin{pmatrix}
            2 && i \sqrt{2}\\
            -i \sqrt{2} && 3
        \end{pmatrix},
        \quad\text{e}\quad
        L_y = \frac{\hbar}{i\sqrt{2}}\begin{pmatrix}
            0 && 1 && 0\\
            -1&& 0 && 1\\
            0 && -1&& 0
        \end{pmatrix}.
    \end{equation*}
    Determine os autovalores e autovetores e os projetores ortogonais sobre seus autoespaços.
\end{exercício}
\begin{proof}[Resolução]
    Notemos que \(\sigma_y\ket{1} = i \ket{2}\) e \(\sigma_y \ket{2} = -i \ket{1},\) então
    \begin{equation*}
        \sigma_y \left(\ket{1} + i\ket{2}\right) = \ket{1} + i\ket{2}\quad\text{e}\quad \sigma_y \left(\ket{1} - i \ket{2}\right) = -\left(\ket{1} - i\ket{2}\right).
    \end{equation*}
    Isto é, os autovalores de \(\sigma_y\) são \(+1\) e \(-1\), com autovetores associados dados por
    \begin{equation*}
        \ket{+} = \frac1{\sqrt{2}}\left(\ket{1} + i\ket{2}\right)
        \quad\text{e}\quad
        \ket{-} = \frac1{\sqrt{2}}\left(\ket{1} - i\ket{2}\right),
    \end{equation*}
    respectivamente. Na base \(\set{\ket{1}, \ket{2}}\) os seus projetores ortogonais são dados por
    \begin{equation*}
        \ketbra{+}{+} = \frac12\begin{pmatrix}
            1 && -i\\
            i && 1
        \end{pmatrix}
        \quad\text{e}\quad
        \ketbra{-}{-} = \frac12 \begin{pmatrix}
            1 && i\\
            -i&& 1
        \end{pmatrix}.
    \end{equation*}

    Verifica-se facilmente que
    \begin{equation*}
        M\left(\sqrt{2}\ket{1} + i \ket{2}\right) = \sqrt{2}\ket{1} + i \ket{2}
        \quad\text{e}\quad
        M\left(\ket{1} - i \sqrt{2} \ket{2}\right) = 4\left(\ket{1} - i\sqrt{2}\ket{2}\right),
    \end{equation*}
    portanto os autovalores de \(M\) são \(1\) e \(4\), associados aos autovetores dados por
    \begin{equation*}
        \ket{M_1} = \frac{1}{\sqrt{3}}\left(\sqrt{2} \ket{1} + i \ket{2}\right)
        \quad\text{e}\quad
        \ket{M_4} = \frac{1}{\sqrt{3}}\left(\ket{1} - i \sqrt{2}\ket{2}\right).
    \end{equation*}
    Os projetores ortogonais dos autoespaços de \(M\) são dados por
    \begin{equation*}
        \ketbra{M_1}{M_1} = \frac13 \begin{pmatrix}
            2 && -i\sqrt{2}\\
            i\sqrt{2} && 1
        \end{pmatrix}
        \quad\text{e}\quad
        \ketbra{M_4}{M_4} = \frac13 \begin{pmatrix}
            1 && i\sqrt{2}\\
            -i\sqrt{2} && 2
        \end{pmatrix}
    \end{equation*}
    na base \(\set{\ket{1}, \ket{2}}\).

    A equação secular para o operador \(L_y\) é \(\lambda(\lambda - \hbar)(\lambda+\hbar) = 0\), portanto seus autovalores são \(\lambda_1 = 0\), \(\lambda_2 = \hbar\) e \(\lambda_3 = -\hbar\). Constata-se que \(\ket{\lambda_1} = \frac{1}{\sqrt{2}} \left(\ket{1} + \ket{3}\right)\) pertence ao núcleo de \(L_y\), portanto é autovetor associado a \(\lambda_1\). Observemos que
    \begin{align*}
        L_y\left(\ket{1} \pm i\sqrt{2} \ket{2} - \ket{3}\right)
        &= \left(\frac{\hbar}{i\sqrt{2}}\right)\left[-\ket{2} \pm i\sqrt{2} \left(\ket{1} - \ket{3}\right) - \ket{2}\right]\\
        &= \hbar \left[\pm \left(\ket{1} - \ket{3}\right) + i\sqrt{2}\ket{2}\right]\\
        &= \pm\hbar \left(\ket{1} \pm i\sqrt{2}\ket{2} - \ket{3}\right),
    \end{align*}
    portanto \(\ket{\lambda_2} = \frac{1}{2}\left(\ket{1} + i\sqrt{2}\ket{2} - \ket{3}\right)\) e \(\ket{\lambda_3} = \frac12 \left(\ket{1} - i\sqrt{2} \ket{2} - \ket{3}\right)\) são autovetores associados aos autovalores \(\lambda_2\) e \(\lambda_3\), respectivamente. Os projetores ortogonal dos autoespaços de \(L_y\) são dados por
    \begin{align*}
        \ketbra{\lambda_1}{\lambda_1} &= \frac12 \left(\begin{smallmatrix}
            1 && 0 && 1\\
            0 && 0 && 0\\
            1 && 0 && 1
        \end{smallmatrix}\right),&
        \ketbra{\lambda_2}{\lambda_2} &= \frac14 \left(\begin{smallmatrix}
            1 && -i\sqrt{2} && -1\\
            i\sqrt{2} && 2 && -i\sqrt{2}\\
            -1 && i\sqrt{2} && 1
        \end{smallmatrix}\right),&
        \ketbra{\lambda_3}{\lambda_3} &= \frac14 \left(\begin{smallmatrix}
            1 && i\sqrt{2} && -1\\
            -i\sqrt{2} && 2 && i\sqrt{2}\\
            -1 && -i\sqrt{2} && 1
        \end{smallmatrix}\right)
    \end{align*}
    na base \(\set{\ket{1}, \ket{2}, \ket{3}}.\)
\end{proof}
