\begin{exercício}{O grupo \(\mathrm{SU}(2)\)}{exercício2}
    O grupo \(\mathrm{SU}(2)\) é o grupo de matrizes \(2\times 2\) unitárias e de determinante unitário.
    \begin{enumerate}[label=(\alph*)]
        \item Mostre que se \(U \in \mathrm{SU}(2)\), então
            \begin{equation*}
                U = \begin{pmatrix}
                    a && b\\
                    -\conj{b} && \conj{a}
                \end{pmatrix},
            \end{equation*}
            para \(a,b \in \mathbb{C}\) tais que \(\abs{a}^2 + \abs{b}^2 = 1\).
        \item Mostre que se \(U \in \mathrm{SU}(2)\) pertence a uma vizinhança da identidade podemos escrever \(U = \unity - i \tau\), com \(\tau\) auto-adjunto, e que \(\tau\) é combinação linear real das matrizes de Pauli, dada por \(\tau = \frac12 \theta_i \sigma_i\), com \(\theta_i \to 0\).
        \item Seja \(\theta^2 = \theta_i \theta_j \delta_{ij}\) e \(\theta_i = \theta n_i\), onde \(\vetor{n}\) é um vetor unitário. Supondo agora que os \(\theta_i\) sejam finitos, definimos
            \begin{equation*}
                U_{\vetor{n}}(\theta) = \lim_{N\to \infty} \left[U\left(\frac{\theta}{N}\right)\right]^N.
            \end{equation*}
            Mostre que \(U_{\vetor{n}}(\theta) = \exp\left(-\frac12i\theta \vetor{n} \cdot \vetor{\sigma}\right)\) e mostre que toda matriz de \(\mathrm{SU}(2)\) é dessa forma.
    \end{enumerate}
\end{exercício}
\begin{proof}[Resolução]
    \todo[Pelo item (a)], consideramos a parametrização  de \(\mathrm{SU}(2)\) dada por
    \begin{equation*}
        U(a,b) = \begin{pmatrix}
            a && b\\
            -\conj{b}&&\conj{a}
        \end{pmatrix},
    \end{equation*}
    onde \(a,b \in \mathbb{C}\) tais que \(\abs{a}^2 + \abs{b}^2 = 1\). Notemos que podemos trocar os parâmetros complexos \(a,b\) pelos parâmetros reais \(\phi \in [0,\pi)\) e \(\vetor{\eta} \in \mathbb{R}^3\) com \(\norm{\vetor{\eta}} = 1\) com
    \begin{equation*}
        a = \cos\left(\frac{\phi}{2}\right) - i \eta_3 \sin\left(\frac{\phi}{2}\right)
        \quad\text{e}\quad
        b = -\sin\left(\frac{\phi}{2}\right)\left(\eta_2 + i\eta_1\right).
    \end{equation*}
    Assim, para \(\phi \ll 1\), temos
\end{proof}
