\begin{exercício}{Sistema de dois níveis com acoplamento}{exercício1}
    Considere um sistema de dois níveis cujo hamiltoniano é representado pela matriz
    \begin{equation*}
        H = \hbar \begin{pmatrix}
            A && B\\
            B && -A
        \end{pmatrix}
    \end{equation*}
    na base \(\set{\ket{+}, \ket{-}}\).
    \begin{enumerate}[label=(\alph*)]
        \item Mostre que
            \begin{equation*}
                \ket{\chi_+} = \cos\left(\frac\theta2\right)\ket{+} + \sin\left(\frac\theta2\right)\ket{-}
            \end{equation*}
            é autovetor de \(H\) associado ao autovalor \(E_+ = \hbar\sqrt{A^2 + B^2}\) e que
            \begin{equation*}
                \ket{\chi_-} = -\sin\left(\frac\theta2\right)\ket{+} + \cos\left(\frac\theta2\right)\ket{-}
            \end{equation*}
            é autovetor de \(H\) associado ao autovalor \(E_- = -\hbar\sqrt{A^2 + B^2}\).
        \item Um vetor de estado \(\ket{\varphi(t)}\) no tempo \(t\) pode ser decomposto na base \(\set{\ket{+}, \ket{-}}\) como
            \begin{equation*}
                \ket{\varphi(t)} = c_+(t) \ket{+} + c_-(t) \ket{-}.
            \end{equation*}
            Escreva o sistema de equações diferenciais acopladas que deve ser obedecido pelas componentes \(c_+(t)\) e \(c_-(t)\).
        \item Podemos agora decompor \(\ket{\varphi(0)}\) na base \(\set{\ket{+}, \ket{-}}\) como
            \begin{equation*}
                \ket{\varphi(0)} = \lambda\ket{\chi_+} + \mu\ket{\chi_-},
            \end{equation*}
            com \(\abs{\lambda}^2 + \abs{\mu}^2 = 1\). Mostre que \(c_+(t)\) se escreve como
            \begin{equation*}
                c_+(t) = \lambda \exp\left(\frac{\Omega t}{2i}\right)\cos\left(\frac{\theta}{2}\right) -\mu \exp\left(-\frac{\Omega t}{2i}\right)\sin\left(\frac{\theta}{2}\right),
            \end{equation*}
            onde \(\hbar \Omega\) é a diferença entre os dois níveis de energia.
        \item Mostre que
            \begin{equation*}
                c_\pm''(t) + \left(\frac{\Omega}{2}\right)^2 c_{\pm}(t) = 0
            \end{equation*}
            é satisfeita pelas componentes \(c_+(t)\) e \(c_-(t)\).
        \item Suponha que \(c_+(0) = 0\). Calcule \(\mu, \lambda\) e \(c_+(t)\).
        \item Mostre que
            \begin{equation*}
                P_+(t) = \sin^2\theta \sin^2\left(\frac{\Omega t}{2}\right) = \frac{B^2}{A^2 + B^2}\sin^2\left(\frac{\Omega t}{2}\right)
            \end{equation*}
            é a probabilidade de encontrar o sistema no tempo \(t\) no estado \(\ket{+}\).
        \item Mostre que se \(c_+(0) = 1\) então
            \begin{equation*}
                c_+(t) = \cos\left(\frac{\Omega t}{2}\right) - i \cos\theta \sin\left(\frac{\Omega t}{2}\right).
            \end{equation*}
            Deduza \(P_+(t)\) e \(P_-(t)\), verificando a compatibilidade de seu resultado com o do item anterior.
    \end{enumerate}
\end{exercício}
\begin{proof}[Resolução do item (a)]
    Definamos \(\theta \in [0,2\pi)\) por \(A = \sqrt{A^2 + B^2}\cos\theta\) e \(B = \sqrt{A^2 + B^2}\sin\theta\), de modo que
    \begin{equation*}
        H = E_+ \begin{pmatrix}
            \cos\theta & \sin\theta\\
            \sin\theta & -\cos\theta
        \end{pmatrix}
    \end{equation*}
    é a representação do hamiltoniano na base \(\set{\ket{+},\ket{-}}\), com \(E_+ = \hbar\sqrt{A^2 +B^2}\). Consideremos os vetores \(\ket{\chi_+} = \cos\left(\frac\theta2\right)\ket{+} + \sin\left(\frac\theta2\right)\ket{-}\) e \(\ket{\chi_-} = -\sin\left(\frac\theta2\right)\ket{+} + \cos\left(\frac\theta2\right)\ket{-}\), então
    \begin{align*}
        H\ket{\chi_+} &= E_+\left\{\left[\cos\theta\cos\left(\frac\theta2\right) + \sin\theta\sin\left(\frac{\theta}2\right)\right]\ket{+} + \left[\sin\theta\cos\left(\frac\theta2\right) - \cos\theta\sin\left(\frac{\theta}2\right)\right]\right\}\\
                      &= E_+\left[\cos\left(\theta - \frac\theta2\right)\ket{+} + \sin\left(\theta - \frac{\theta}{2}\right)\ket{-}\right]\\
                      &= E_+\ket{\chi_+}
    \end{align*}
    e
    \begin{align*}
        H\ket{\chi_-} &= E_+\left\{\left[-\cos\theta\sin\left(\frac\theta2\right) + \sin\theta\cos\left(\frac{\theta}2\right)\right]\ket{+} + \left[-\sin\theta\sin\left(\frac\theta2\right) - \cos\theta\cos\left(\frac{\theta}2\right)\right]\right\}\\
                      &= -E_+\left[\sin\left(\theta - \frac\theta2\right)\ket{+} + \cos\left(\theta - \frac{\theta}{2}\right)\ket{-}\right]\\
                      &= -E_+\ket{\chi_-}
    \end{align*}
    isto é, \(\ket{\chi_+}\) é o autovetor de \(H\) associado ao autovalor \(E_+\) e \(\ket{\chi_-}\) é o autovetor de \(H\) associado ao autovalor \(E_- = - E_+\).
\end{proof}

\begin{proof}[Resolução do item (b)]
    Da evolução temporal para o estado \(\ket{\varphi(t)}\),
    \begin{equation*}
        i \hbar \diff*{\ket{\varphi(t)}}{t} = H \ket{\phi(t)},
    \end{equation*}
    temos o sistema de equações diferenciais
    \begin{equation*}
        \left\{
        \begin{aligned}
            i\hbar \diff{c_+}{t} = E_+ \cos\theta c_+(t) + E_+ \sin\theta c_-(t)\\
            i\hbar \diff{c_-}{t} = E_+ \sin\theta c_+(t) - E_+ \cos\theta c_-(t)
        \end{aligned}\right.
    \end{equation*}
    que as componentes devem satisfazer.
\end{proof}
\begin{proof}[Resolução do item (c)]
    Decompondo na base de autoestados do hamiltoniano, segue que
    \begin{align*}
        \ket{\varphi(t)} &= \exp\left(\frac{Ht}{i\hbar}\right)\ket{\varphi(0)}\\
                         &= \lambda\exp\left(\frac{Ht}{i\hbar}\right)\ket{\chi_+} + \mu\exp\left(\frac{Ht}{i\hbar}\right)\ket{\chi_-}\\
                         &= \lambda\exp\left(\frac{E_+t}{i\hbar}\right)\ket{\chi_+} + \mu\exp\left(\frac{E_-t}{i\hbar}\right)\ket{\chi_-}.
    \end{align*}
    Definindo \(\Omega = \frac{E_+ - E_-}{\hbar} = \frac{2E_+}{\hbar}\), temos
    \begin{equation*}
        \ket{\varphi(t)} = \lambda\exp\left(\frac{\Omega t}{2i}\right)\ket{\chi_+} + \mu\exp\left(\frac{\Omega t}{2i}\right)\ket{\chi_-}.
    \end{equation*}
    Desse modo, obtemos
    \begin{equation*}
        c_\pm(t) = \braket{\pm}{\varphi(t)} = \lambda\exp\left(\frac{\Omega t}{2i}\right)\braket{\pm}{\chi_+} + \mu\exp\left(\frac{\Omega t}{2i}\right)\braket{\pm}{\chi_-},
    \end{equation*}
    donde segue que
    \begin{equation*}
        \left\{
        \begin{aligned}
            c_+(t) = \lambda\exp\left(\frac{\Omega t}{2i}\right)\cos\left(\frac\theta2\right) - \mu\exp\left(\frac{\Omega t}{2i}\right)\sin\left(\frac\theta2\right)\\
            c_-(t) = \lambda\exp\left(\frac{\Omega t}{2i}\right)\sin\left(\frac\theta2\right) + \mu\exp\left(\frac{\Omega t}{2i}\right)\cos\left(\frac\theta2\right)
        \end{aligned}\right.
    \end{equation*}
    são as componentes \(c_+(t)\) e \(c_-(t)\).
\end{proof}
\begin{proof}[Resolução do item (d)]

\end{proof}
\begin{proof}[Resolução do item (e)]

\end{proof}
\begin{proof}[Resolução do item (f)]

\end{proof}
\begin{proof}[Resolução do item (g)]

\end{proof}
