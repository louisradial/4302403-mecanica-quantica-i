\begin{exercício}{Autoestado simultâneo de \(L^2\) e \(L_z\)}{exercício7}
    Um certo estado \(\ket{\psi}\) é um autoestado de \(L^2\) e de \(L_z\), de forma que
    \begin{equation*}
        L^2\ket{\psi} = l(l+1)\hbar^2 \ket{\psi}\quad\text{e}\quad L_z\ket{\psi} = m\hbar \ket{\psi}.
    \end{equation*}
    Calcule \(\mean{L_x}, \mean{L_x^2}\) e \(\Delta L_x\).
\end{exercício}
