\begin{exercício}{}{exercício3}
    Nos propomos modelar o comportamento dos dois elétrons \(\pi\) da ligação dupla da molécula de formaldeído \ce{H2C=O}. O oxigênio sendo mais eletronegativo que o carbono, discuta por que o hamiltoniano de um elétron toma a forma
    \begin{equation*}
        \begin{pmatrix}
            E_C && -A\\
            -A &&E_O
        \end{pmatrix}
    \end{equation*}
    com \(E_O < E_C\), onde \(E_C\) e \(E_O\) representam a energia de um elétron localizado sobre o átomo de carbono e oxigênio, respectivamente.

    Definimos \(B = \frac12 (E_C - E_O) > 0\) e o ângulo \(\theta\) tal que \(B = \sqrt{A^2 + B^2}\cos\theta\) e \(A = \sqrt{A^2 + B^2}\sin\theta\). Calcule a probabilidade de encontrar um elétron \(\pi\) localizado sobre o átomo de carbono ou oxigênio em função de \(\theta\).
\end{exercício}
\begin{proof}[Resolução]
    Consideremos primeiro o caso em que
    Com as definições dadas, sendo \(\alpha = \frac{E_C + E_O}{2}\) e \(\beta = \sqrt{A^2 + B^2}\), temos \(H = \alpha \mathds{1} + \beta T\), onde \(T\) é o operador dado por \(T\ket{1} = \cos\theta \ket{1} - \sin\theta \ket{2}\) e \(T\ket{2} = -\sin\theta\ket{1} - \cos\theta\ket{2}\). Assim, segue que um vetor não nulo é autovetor de \(T\) se e somente se é autovetor de \(H\), visto que \(\beta \neq 0\). Sejam os vetores não nulos
    \begin{equation*}
        \ket{+} = \cos\left(\frac{\theta}{2}\right)\ket{1} - \sin\left(\frac{\theta}{2}\right)\ket{2}
        \quad\text{e}\quad
        \ket{-} = \sin\left(\frac{\theta}{2}\right)\ket{1} + \cos\left(\frac{\theta}{2}\right)\ket{2},
    \end{equation*}
    então
    \begin{align*}
        T\ket{+} &= \left[\cos\left(\frac{\theta}{2}\right)\cos\theta + \sin\left(\frac{\theta}{2}\right)\sin\theta\right]\ket{1} + \left[\sin\left(\frac{\theta}{2}\right)\cos\theta - \cos\left(\frac{\theta}{2}\right)\sin\theta\right]\ket{2}\\
                 &= \cos\left(\frac{\theta}{2}\right)\ket{1} - \sin\left(\frac{\theta}{2}\right) \ket{2} = \ket{+}
    \end{align*}
    e
    \begin{align*}
        T\ket{-} &= \left[\sin\left(\frac{\theta}{2}\right)\cos\theta - \cos\left(\frac{\theta}{2}\right)\sin\theta\right]\ket{1} - \left[\sin\left(\frac{\theta}{2}\right)\sin\theta - \cos\left(\frac{\theta}{2}\right)\cos\theta\right]\ket{2}\\
                 &= -\sin\left(\frac{\theta}{2}\right)\ket{1} - \cos\left(\frac{\theta}{2}\right) \ket{2} = -\ket{-}
    \end{align*}
    portanto concluímos que \(\ket{+}\) e \(\ket{-}\) são autoestados de \(H\) associados às energias \(\alpha + \beta\) e \(\alpha - \beta\), respectivamente.
\end{proof}
