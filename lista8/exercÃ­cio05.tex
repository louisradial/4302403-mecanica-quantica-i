\begin{exercício}{}{exercício5}
    O \cref{ex:exercício4} é uma representação da massa reduzida \(\mu\) de uma molécula diatômica, cujos átomos estão a uma distância fixa \(r\) um do outro, e que efetua movimentos de rotação.
    \begin{enumerate}[label=(\alph*)]
        \item Determine os níveis de energia de rotação da molécula diatômica utilizando a expressão geral para os autovalores de um momento angular \(\vetor{L}\).
        \item O espectro de absorção de rotação da molécula de monóxido de carbono apresenta um pico de absorção para um comprimento de onda \(\lambda = \SI{1.3}{\milli\meter}\), correspondente à uma transição entre os níveis \(\ell = 1\) e \(\ell = 2\). Calcule o momento de inércia da molécula a partir dos dados experimentais.
        \item Deduza a distância entre os átomos constituintes da molécula. \todo[Para \(N_A\) átomos, C = \SI{12}{g}, O = \SI{16}{g}.]
    \end{enumerate}
\end{exercício}
\begin{proof}[Resolução]

\end{proof}
