\begin{exercício}{Projetores ortonormais de estados}{exercício6}
    O espaço de estados de um certo sistema físico tem 3 dimensões. Sejam \(\set{\ket{u_1}, \ket{u_2}, \ket{u_3}}\) uma base ortonormal desse espaço. Definimos os kets \(\ket{\psi_0}\) e \(\ket{\psi_1}\) por
    \begin{equation*}
        \ket{\psi_0} = \frac{1}{\sqrt{2}} \ket{u_1} +\frac{i}{2} \ket{u_2} + \frac{i}{2}\ket{u_3}
        \quad\text{e}\quad
        \ket{\psi_1} = \frac{1}{\sqrt{2}} \ket{u_1} + \frac{i}{\sqrt{2}}\ket{u_3}.
    \end{equation*}
    Determine na base \(\set{\ket{u_1}, \ket{u_2}, \ket{u_3}}\) as representações dos projetores ortonormais \(\rho_0\) e \(\rho_1\) associados aos estados \(\ket{\psi_0}\) e \(\ket{\psi_1}\), respectivamente.
\end{exercício}
\begin{proof}[Resolução]
    Na base \(\set{\ket{u_1}, \ket{u_2}, \ket{u_3}}\),
    \begin{equation*}
        \rho_0 = \ketbra{\psi_0}{\psi_0} = \frac14\left(\begin{smallmatrix}
                2 && -i\sqrt{2} && -i\sqrt{2}\\
                i\sqrt{2} && 1 && 1\\
                i\sqrt{2} && 1 && 1
        \end{smallmatrix}\right)
        \quad\text{e}\quad
        \rho_1 = \ketbra{\psi_1}{\psi_1} = \frac12\left(\begin{smallmatrix}
                1 && 0 && -i\\
                0 && 0 && 0\\
                i && 0 && 1
        \end{smallmatrix}\right)
    \end{equation*}
    são as representações dos projetores ortogonais \(\rho_0\) e \(\rho_1\).
\end{proof}
