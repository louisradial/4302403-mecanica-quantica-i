\begin{lemma}{Matriz de espalhamento a partir da matriz de transmissão}{matriz_espalhamento}
    Para um potencial que se anula na região \((-\infty,\ell_1) \cup (\ell_2, \infty)\) com \(\ell_1 < \ell_2\), seja
    \begin{equation*}
        \varphi_k(x) = \begin{cases}
            A_1 e^{ikx} + A_2 e^{-ikx}, &\text{se }x < \ell_1\\
            B_1 e^{ikx} + B_2 e^{-ikx}, &\text{se }x > \ell_2
        \end{cases}
    \end{equation*}
    a função de onda de uma partícula livre com energia \(\frac{(\hbar k)^2}{2m}\). Definamos as matrizes de transmissão \(M\) e de espalhamento \(S\) de forma que
    \begin{equation*}
        \begin{pmatrix}
            B_1\\
            B_2
        \end{pmatrix} =
        M
        \begin{pmatrix}
            A_1\\
            A_2
        \end{pmatrix}\quad\text{e}\quad
        \begin{pmatrix}
            B_1\\
            A_2
        \end{pmatrix} =
        S
        \begin{pmatrix}
            A_1\\
            B_2
        \end{pmatrix}.
    \end{equation*}
    Para uma matriz de transmissão dada por
    \begin{equation*}
        M = \begin{pmatrix}
            F && G^*\\
            G && F^*
        \end{pmatrix}
    \end{equation*}
    com \(\abs{F}^2 - \abs{G}^2 = 1\), então
    \begin{equation*}
        S = \frac{1}{F^*}\begin{pmatrix}
            1 && G^*\\
            -G&& 1
        \end{pmatrix}
    \end{equation*}
    é a matriz de espalhamento.
\end{lemma}
\begin{proof}
    Da definição da matriz de transmissão, temos
    \begin{equation*}
        \begin{cases}
            B_1 = F A_1 + G^* A_2\\
            B_2 = G A_1 + F^* A_2
        \end{cases}
    \end{equation*}
    então
    \begin{equation*}
        A_2 = \frac{B_2 - G A_1}{F^*}
    \end{equation*}
    e obtemos
    \begin{align*}
        B_1 &= F A_1 + \frac{G^* B_2 - \abs{G}^2A_1}{F^*}\\
            &= \frac{(\abs{F}^2 - \abs{G}^2)A_1 + G^* B_2}{F^*}\\
            &= \frac{A_1 + G^* B_2}{F^*}.
    \end{align*}
    Isto é, temos
    \begin{equation*}
        \begin{pmatrix}
            B_1\\A_2
        \end{pmatrix} = \frac{1}{F^*}
        \begin{pmatrix}
            1 && G^*\\
            -G&& 1
        \end{pmatrix}
        \begin{pmatrix}
            A_1\\B_2
        \end{pmatrix},
    \end{equation*}
    como desejado.
\end{proof}
\begin{exercício}{Microscópio a efeito túnel}{exercício4}
    G. Binning e H. Rohrer receberam o prêmio Nobel de Física de 1986 \enquote{for their design of the scanning tunneling microscope}. Esse microscópio é constituído por uma ponta metálica que se desloca muito perto da superfície de uma amostra de material condutor. A aplicação de uma pequena diferença de potencial entre a ponta e a amostra possibilita que elétrons escapem da amostra para a ponta por \emph{efeito túnel}. A corrente elétrica obtida pelo deslocamento da ponta sobre a amostra permite reconstruir uma imagem da superfície do material em escala atômica. O exercício a seguir mostra como as variações da intensidade da corrente em função da distância entre a ponta e a amostra possibilita visualizar átomos. Se a diferença de potencial aplicada é bem pequena, aparece uma barreira de potencial entre a ponta exploradora e a amostra, que tem a forma indicada pela linha pontilhada na figura abaixo.

    \begin{center}
        \begin{tikzpicture}
            \begin{axis}[
                width=0.45\linewidth,
                height=0.15\textheight,
                xmin=-0.5, xmax=1.5,
                ymin=0,ymax=1.5,
                domain=-0.5:1.5,
                samples=2001,
                axis lines=middle,
                xlabel={\(x\)},
                xlabel style = {anchor=north east},
                % ylabel near ticks,
                ylabel={\(E\)},
                legend pos=south east,
                ytick={0,1},
                xtick={0,1},
                xticklabels={\(0\),\(a\)},
                yticklabels={\(0\),\(V_0\)},
                ]

                \addplot[Mauve, thick] {(x >= 0) ? ((x <= 1) ? 1 : 0) : 0};

                \addplot[Pink, dotted, thick] {(x >= 0.025) ? ((x <= 0.975) ? 1 - 0.12*x : 0.05) : 0.15};
            \end{axis}
        \end{tikzpicture}
    \end{center}

    A barreira pode então ser descrita em boa aproximação por
    \begin{equation*}
        V(x) = \begin{cases}
            V_0,&\text{se }x \in [0,a]\\
            0,&\text{se }x \notin[0,a].
        \end{cases}
    \end{equation*}
    \begin{enumerate}[label=(\alph*)]
        \item Determine as funções de onda que representam um fluxo de elétrons de momento \(p\) e energia \(E \leq V_0\) através da barreira de potencial.
        \item Calcule a expressão para o fluxo de elétrons transmitidos através da barreira de potencial.
        \item O coeficiente de transmissão \(T\) é definido como sendo a razão entre as correntes incidentes e transmitidas. Determine \(T\).
        \item Calcule a variação de corrente de um microscópio a efeito túnel quando o espaço entre a ponta e a amostra sofre uma variação de \SI{0.1}{\nano\meter} com \(V_0 = \SI{6}{\eV}\), \(a = \SI{1}{\nano\meter}\), e \(E = \SI{1}{\eV}\).
    \end{enumerate}
\end{exercício}
\begin{proof}[Resolução]
    A função de onda de uma partícula livre com energia \(E = V_0 \bar{E} < V_0\) e momento \(p = \hbar k\) pode ser escrita como
    \begin{equation*}
        \varphi(x) = \begin{cases}
            A_1 e^{i k x} + A_2 e^{-i k x},&\text{se }x < 0\\
            B_1 e^{\rho x} + B_2 e^{-\rho x},&\text{se }x \in [0,a]\\
            C_1 e^{i k x} + C_2 e^{-i k x},&\text{se }x > a,
        \end{cases}
    \end{equation*}
    onde
    \begin{equation*}
        \hbar k = \sqrt{2m V_0 \bar{E}} \quad\text{e}\quad \hbar \rho = \sqrt{2 m (1 - \bar{E})V_0}.
    \end{equation*}
    Da continuidade da função de onda temos
    \begin{equation*}
        A_1 + A_2 = B_1 + B_2
        \quad\text{e}\quad
        C_1 e^{i k a} + C_2 e^{-ik a} = B_1 e^{\rho a} + B_2 e^{-\rho a}
    \end{equation*}
    e da continuidade de sua primeira derivada,
    \begin{equation*}
        ik(A_1 - A_2 ) = \rho(B_1  - B_2 )
        \quad\text{e}\quad
        ik(C_1 e^{i k a} - C_2 e^{-ik a}) = \rho(B_1 e^{\rho a} - B_2 e^{-\rho a}).
    \end{equation*}
    Podemos reescrever matricialmente como
    \begin{equation*}
        \begin{pmatrix}
            B_1 \\
            B_2
        \end{pmatrix} = \frac12
        \begin{pmatrix}
            1 + \frac{ik}{\rho} &&
            1 - \frac{ik}{\rho} \\
            1 - \frac{ik}{\rho} &&
            1 + \frac{ik}{\rho}
        \end{pmatrix}
        \begin{pmatrix}
            A_1\\
            A_2
        \end{pmatrix}
    \end{equation*}
    e
    \begin{equation*}
        \begin{pmatrix}
            C_1\\
            C_2
        \end{pmatrix} = \frac12
        \begin{pmatrix}
            \left(1 + \frac{\rho}{ik}\right)e^{(\rho - ik)a} &&
            \left(1 - \frac{\rho}{ik}\right)e^{-(\rho + ik)a} \\
            \left(1 - \frac{\rho}{ik}\right)e^{(\rho + ik)a} &&
            \left(1 + \frac{\rho}{ik}\right)e^{-(\rho - ik)a}
        \end{pmatrix}
        \begin{pmatrix}
            B_1\\
            B_2
        \end{pmatrix},
    \end{equation*}
    isto é,
    \begin{align*}
        \begin{pmatrix}
            C_1\\
            C_2
        \end{pmatrix} &= \frac1{4 ik \rho}
        \begin{pmatrix}
            \left(ik + \rho\right)e^{(\rho - ik)a} &&
            \left(ik - \rho\right)e^{-(\rho + ik)a} \\
            \left(ik - \rho\right)e^{(\rho + ik)a} &&
            \left(ik + \rho\right)e^{-(\rho - ik)a}
        \end{pmatrix}
        \begin{pmatrix}
            \rho + ik &&
            \rho - ik \\
            \rho - ik &&
            \rho + ik
        \end{pmatrix}
        \begin{pmatrix}
            A_1\\
            A_2
        \end{pmatrix}\\
        &= \frac{1}{4ik\rho}
        \begin{pmatrix}
            (\rho + ik)^2 e^{(\rho - ik)a} - (\rho -ik)^2 e^{-(\rho + ik)a} && (\rho^2 + k^2) \left[e^{(\rho - ik)a} - e^{-(\rho + ik)a}\right]\\
            (\rho^2 + k^2) \left[e^{-(\rho - ik)a} - e^{(\rho + ik)a}\right] && (\rho + ik)^2e^{-{(\rho -ik)a}} - (\rho - ik)^2e^{(\rho + ik)a}
        \end{pmatrix}
        \begin{pmatrix}
            A_1\\
            A_2
        \end{pmatrix}\\
        &= \begin{pmatrix}
            e^{-ika}\left[\frac{\rho^2 - k^2}{2i k \rho}\sinh(\rho a) + \cosh(\rho a)\right] && \frac{\rho^2 + k^2}{2ik\rho}e^{-ika}\sinh(\rho a)\\
            -\frac{\rho^2 + k^2}{2ik\rho} e^{ika} \sinh(\rho a) &&e^{ika}\left[\cosh(\rho a)-\frac{\rho^2 - k^2}{2i k \rho}\sinh(\rho a)\right]
        \end{pmatrix}
        \begin{pmatrix}
            A_1\\
            A_2
        \end{pmatrix}.
    \end{align*}
    Dessa forma, determinamos a matriz de transmissão e podemos escrever a matriz de espalhamento pela equação
    \begin{equation*}
        \begin{pmatrix}
            C_1 \\
            A_2
        \end{pmatrix} =
        e^{-ika}\left[\cosh(\rho a)-\frac{\rho^2 - k^2}{2i k \rho}\sinh(\rho a)\right]^{-1}
        \begin{pmatrix}
            1 && \frac{\rho^2 + k^2}{2ik\rho}e^{-ika}\sinh(\rho a)\\
            \frac{\rho^2 + k^2}{2ik\rho} e^{ika} \sinh(\rho a) && 1
        \end{pmatrix}
        \begin{pmatrix}
            A_1 \\
            C_2
        \end{pmatrix}.
    \end{equation*}
    Se a partícula é proveniente da região \(x < 0\), temos \(C_2 = 0\), logo o coeficiente de transmissão é dado por
    \begin{equation*}
        T^{-1} = \abs*{\cosh(\rho a) - \frac{\rho^2 - k^2}{2ik\rho}\sinh(\rho a)}^2 = 1 + \left(\frac{\rho^2 + k^2}{2k\rho}\right)^2 \sinh^2(\rho a) = 1 + \left(\frac{2m V_0}{4mV_0 \sqrt{\bar{E} - \bar{E}^2}}\right)^2\sinh^2(\rho a),
    \end{equation*}
    isto é,
    \begin{equation*}
        T = \frac{4\bar{E}(1 - \bar{E})}{4\bar{E} (1 - \bar{E}) + \sinh^2\left[\frac{a}{\hbar}\sqrt{2mV_0(1 - \bar{E})}\right]}.
    \end{equation*}
\end{proof}
