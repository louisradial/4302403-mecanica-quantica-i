\begin{exercício}{Oscilador harmônico isotrópico}{exercício08}
    Encontre os autoestados e respectivas autofunções de um oscilador harmônico tridimensional isotrópico, isto é, cujo potencial é \(V(r) = \frac12 M \omega^2 r^2\).
    \begin{enumerate}[label=(\alph*)]
        \item Resolva o problema em coordenadas cartesianas. Qual a degenerescência dos autoestados de \(H\)?
        \item Resolva novamente o problema, mas desta vez imponha que os autoestados também são autovetores de \(L_z\) e de \(\vetor{L}^2\).
    \end{enumerate}
\end{exercício}
\begin{proof}[Resolução]
    Escrevendo \(\vetor{r} = x_j \vetor{e}_j\) e \(\vetor{p} = p_j \vetor{e}_j\), temos
    \begin{equation*}
        H = \frac1{2M} p^2 + \frac12 M \omega^2 r^2 = \sum_{j= 1}^3\left(\frac1{2M} p_j^2 + \frac12 M \omega^2 x_j^2\right) = \frac{\hbar \omega}{2} \sum_{j= 1}^3 \left[\left(\frac1{\sqrt{M  \hbar\omega}}p_j\right)^2 + \left(\sqrt{\frac{M \omega}{\hbar}}x_j\right)^2\right].
    \end{equation*}
    Definindo os operadores de aniquilação e de aniquilação
    \begin{equation*}
        a_j = \frac{1}{\sqrt{2}}\left(\sqrt{\frac{M\omega}{\hbar}}x_j + \frac{i}{\sqrt{M \hbar \omega}}p_j\right)
        \quad\text{e}\quad
        \herm{a_j} = \frac{1}{\sqrt{2}}\left(\sqrt{\frac{M\omega}{\hbar}}x_j - \frac{i}{\sqrt{M \hbar \omega}}p_j\right)
    \end{equation*}
    temos
    \begin{equation*}
        \sqrt{\frac{M\omega}\hbar}x_j = \frac1{\sqrt{2}}\left(a_j + \herm{a_j}\right) \quad\text{e}\quad \frac{1}{\sqrt{M \hbar \omega}} p_j = \frac{i}{\sqrt{2}}\left(\herm{a_j} - a_j\right),
    \end{equation*}
    portanto
    \begin{equation*}
        H = \frac{\hbar \omega}{2} \sum_{j=1}^3 \anticommutator{a_j}{\herm{a_j}}.
    \end{equation*}
    Notemos que os operadores de criação e aniquilação satisfazem
    \begin{align*}
        \commutator{a_j}{\herm{a_k}} &= \frac12\commutator*{\sqrt{\frac{M\omega}{\hbar}}x_j + \frac{i}{\sqrt{M \hbar \omega}}p_j}{\sqrt{\frac{M\omega}{\hbar}}x_k - \frac{i}{\sqrt{M \hbar \omega}}p_k}\\
                                     &= \frac{i}{2\hbar} \left(\commutator{p_j}{x_k} - \commutator{x_j}{p_k}\right)\\
                                     &= \frac{1}{i\hbar} \commutator{x_j}{p_j} \delta_{jk}\\
                                     &= \unity \delta_{jk},
    \end{align*}
    portanto
    \begin{equation*}
        H = \hbar \omega \sum_{j= 1}^3 \left(\herm{a_j}a_j + \frac12 \unity\right) = \frac{3 \hbar \omega}{2} \unity + \hbar \omega \sum_{j= 1}^3 \herm{a_j}a_j = \frac{3\hbar \omega}{2} \unity + \hbar\omega \sum_{j= 1}^3 N_j,
    \end{equation*}
    onde o operador autoadjunto \(N_j = \herm{a_j}a_j\) é o operador de número da direção \(j\). Consideramos o autoestado \(\ket{\set{n_j}}\) do hamiltoniano com
    \begin{equation*}
        N_k\ket{\set{n_j}} = \delta_{jk} n_j \ket{\set{n_j}}
    \end{equation*} de modo que
    \begin{equation*}
        H \ket{\set{n_j}} = \hbar \omega \left(\frac32 + \sum_{k = 1}^3 n_k\right)\ket{\set{n_j}}.
    \end{equation*}
    Assim, temos
    \begin{equation*}
        \norm{a_k \ket{\set{n_j}}}^2 = \bra{\set{n_j}}N_k\ket{\set{n_j}} = n_k,
    \end{equation*}
    portanto segue que \(n_k \geq 0\) e que se \(n_k = 0\), então \(a_k \ket{\set{n_j}} = 0\). De forma semelhante temos
    \begin{equation*}
        \norm{\herm{a_k} \ket{\set{n_j}}}^2 = \bra{\set{n_j}}a_k\herm{a_k}\ket{\set{n_j}} = \bra{\set{n_j}}\left(\unity + N_k\right)\ket{\set{n_j}} = 1 + n_k,
    \end{equation*}
    de modo que \(\herm{a_k} \ket{\set{n_j}} \neq 0\). Notemos que
    \begin{equation*}
        \commutator{N_k}{a_j} = \herm{a_k}a_ka_j - a_j\herm{a_k}a_k = \delta_{jk}\commutator{\herm{a_k}}{a_k}a_k = -\delta_{jk} a_k,
    \end{equation*}
    então \(\commutator{N_k}{\herm{a_j}} = \delta_{jk}\herm{a_k}\) já que \(N_k\) é autoadjunto. Destas relações de comutação, temos
    \begin{equation*}
        N_i\herm{a_j}\ket{\set{n_k}} = \left(\commutator{N_i}{\herm{a_j}}  + \herm{a_j}N_i\right)\ket{\set{n_k}} = \delta_{ij}\herm{a_i}\ket{\set{n_k}} + n_i \herm{a_j}\ket{\set{n_k}} = (n_i + \delta_{ij}) \herm{a_j} \ket{\set{n_k}},
    \end{equation*}
    portanto \(\herm{a_j}\ket{\set{n_k}}\) é autovetor de \(N_i\) com autovalor \(n_i + \delta_{ij}\) e, se \(n_j > 0\), temos
    \begin{equation*}
        N_ia_j\ket{\set{n_k}} = \left(\commutator{N_i}{a_j}  + a_jN_i\right)\ket{\set{n_k}} = -\delta_{ij}a_i\ket{\set{n_k}} + n_i a_j\ket{\set{n_k}} = (n_i - \delta_{ij}) a_j \ket{\set{n_k}},
    \end{equation*}
    isto é, \(a_j\ket{\set{n_k}}\) é autovetor de \(N_i\) com autovalor \(n_i - \delta_{ij}\). Destas relações, concluímos que \(n_i \in \mathbb{N}_0\), caso contrário existiria um estado com energia negativa, mas o espectro do hamiltoniano deve ser não negativo já que o potencial é não negativo. Com a liberdade de fase, encontramos
    \begin{equation*}
        a_k\ket{\set{n_j}} = \sqrt{n_k}\ket{\set{n_j - \delta_{jk}}}\quad\text{e}\quad\herm{a_k}\ket{\set{n_j}} = \sqrt{1 + n_k}\ket{\set{n_k + \delta_{jk}}}
    \end{equation*}
    e o espectro de energia é dado por
    \begin{equation*}
        E_n = \hbar \omega \left(\frac32 + n\right),
    \end{equation*}
    onde \(n \in \setc{n_1 + n_2 + n_3}{n_1, n_2, n_3 \in \mathbb{N}_0}\).

    É fácil ver que \(n \in \mathbb{N}_0\), já que podemos tomar \(n_2 = n_3 = 0\), portanto podemos indexar os níveis de energia por \(n \in \mathbb{N}_0\), cada um com degenerescência \(g_n\). Seja \(n \in \mathbb{N}_0\), então se fixarmos \(n_1\), temos \(n_2 + n_3 = n - n_1\), logo há \(n - n_1 + 1\) possíveis valores de \(n_2, n_3\) que satisfazem esta relação, já que para cada \(n_2\) há apenas um possível valor de \(n_3\) e podemos tomar \(n_2 \in \set{0, 1,\dots, n - n_1}\). Com isso, a degenerescência é dada por
    \begin{equation*}
        g_n = \sum_{n_1 = 0}^{n} (n - n_1 + 1) = (n + 1)^2 - \sum_{n_1 = 0}^n n_1 = (n+1)^2 - \frac{n(n+1)}{2} = \frac{(n+1)(n + 2)}{2}
    \end{equation*}
    para todo \(n \in \mathbb{N}_0\).

    Consideramos o problema novamente, com os autovetores \(\ket{n \ell m}\) simultâneos de \(H,\) \(\vetor{L}^2\), e \(L_z\),
    \begin{equation*}
        H\ket{n \ell m} = \hbar \omega \left(\frac32 + n\right) \ket{n \ell m},\quad
        \vetor{L}^2\ket{n \ell m} = \hbar^2 \ell(\ell + 1) \ket{n \ell m},\quad\text{e}\quad
        L_z\ket{n \ell m} = \hbar \omega \ket{n \ell m}.
    \end{equation*}
    % Como \(L_i = \epsilon_{ijk}x_jp_k\), temos
    % \begin{align*}
    %     L_i &= \hbar \epsilon_{ijk} \left(\sqrt{\frac{M \omega}{\hbar}} x_j\right)\left(\frac{1}{\sqrt{M \hbar \omega}} p_k\right)\\
    %         &= \frac{i\hbar}{2} \epsilon_{ijk}\left(a_j + \herm{a_j}\right)\left(\herm{a_k} - a_k\right)\\
    %         &= \frac{i\hbar}{2} \epsilon_{ijk}\left(a_j \herm{a_k} - a_j a_k + \herm{a_j}\herm{a_k} - \herm{a_j}a_k\right)\\
    %         &= \frac{i\hbar}{2} \epsilon_{ijk} \left(a_j \herm{a_k} - \herm{a_j}a_k\right),
    % \end{align*}
    % já que \([a_j,a_k] = 0\) e \([\herm{a_j},\herm{a_k}] = 0\). Da relação de comutação, obtemos
    % \begin{align*}
    %     L_i &= \frac{i\hbar}{2} \epsilon_{ijk} \left(a_j \herm{a_k} - a_k \herm{a_j} + \commutator{a_k}{\herm{a_j}}\right)\\
    %         &= \frac{i\hbar}{2}\epsilon_{ijk} \left(a_j \herm{a_k} - a_k \herm{a_j} + \delta_{jk}\unity\right)\\
    %         &= \frac{i\hbar}{2} \left(\epsilon_{ijk}a_j \herm{a_k} + \epsilon_{ikj}a_k \herm{a_j} \right)\\
    %         &= i\hbar \epsilon_{ijk}a_j \herm{a_k},
    % \end{align*}
    % e então
    % \begin{align*}
    %     L^2 &= -\hbar^2 \epsilon_{ijk} a_j \herm{a_k} \epsilon_{uvw} a_v \herm{a_w} \delta_{iu}\\
    %         &= -\hbar^2 \epsilon_{ijk} \epsilon_{ivw} a_j \herm{a_k} a_v \herm{a_w}\\
    %         &= \hbar^2 (\delta_{jw}\delta_{kv}-\delta_{jv}\delta_{kw}) a_j \herm{a_k} a_v \herm{a_w}\\
    %         &= \hbar^2 a_j \herm{a_k} a_k \herm{a_j} - \hbar^2 a_j\herm{a_k} a_j \herm{a_k}\\
    %         &= \hbar^2 a_j N \herm{a_j} - \hbar^2 \left(\commutator{a_j}{\herm{a_k}} + \herm{a_k}a_j\right)\left(\commutator{a_j}{\herm{a_k}} + \herm{a_k}a_j\right)\\
    %         &= \hbar^2 a_j N \herm{a_j} - \hbar^2 \left(\delta_{jk}\unity + \herm{a_k}a_j\right)\left(\delta_{jk}\unity + \herm{a_k}a_j\right)\\
    %         &= \hbar^2 \left(a_j N \herm{a_j} - \delta_{jk}\delta_{jk} \unity - 2N - \herm{a_k}a_j \herm{a_k}a_j\right)\\
    %         &= \hbar^2 \left(a_j N \herm{a_j} - 3 \unity - 2N - \herm{a_k}\herm{a_k}a_ja_j - \herm{a_k}\commutator{a_j}{\herm{a_k}}a_j\right)\\
    %         &= \hbar^2\left( a_j N \herm{a_j} -3 \unity - 3N - \herm{a_k}\herm{a_k}a_ja_j\right),
    % \end{align*}
    % onde \(N = \herm{a_k}a_k = \sum_{j=1}^3 N_j\). Notemos que \(\commutator{N}{a_j} = -a_j\), então
    % \begin{align*}
    %     L^2 &= \hbar^2\left(N a_j \herm{a_j} + \commutator{a_j}{N}\herm{a_j} - 3 \unity - 3N - \herm{a_k}\herm{a_k}a_ja_j\right)\\
    %         &= \hbar^2 \left[(N + \unity) \left(a_j\herm{a_j} - 3\unity\right) - \herm{a_k}\herm{a_k}a_ja_j\right]\\
    %         &= \hbar^2\left[(N + \unity)\left(N - 2\unity\right) - \herm{a_k}\herm{a_k}a_ja_j\right]
    % \end{align*}
    Consideramos a função de onda \(\psi_{n\ell m}(\vetor{r}) = \braket{\vetor{r}}{n \ell m} = \frac{u_{n\ell}(r)}{r} Y_\ell^m(\theta, \phi)\), então
    \begin{equation*}
        -\frac{\hbar^2}{2M}\diff[2]{u_{n\ell}}{r} + \left[\frac{\hbar^2\ell(\ell + 1)}{2M r^2} + \frac12 M \omega^2 r^2 - \hbar\omega \left(\frac{3}{2} + n\right)\right]u_{n \ell}(r) = 0
    \end{equation*}
    pelo \cref{lem:equação_radial}. Definindo \(\gamma = \sqrt{\frac{M \omega}{\hbar}}\), temos
    \begin{equation*}
        \diff[2]{u_{n\ell}}{r} + \gamma^2\left[(3 + 2n) - \gamma^2r^2 - \frac{\ell(\ell + 1)}{\gamma^2r^2}\right]u_{n \ell}(r) = 0
    \end{equation*}
    portanto escrevendo \(\rho = \gamma r\) e \(u_{n\ell}(r) = w_{n \ell}(\rho)\), temos
    \begin{equation*}
        \diff[2]{w_{n\ell}}{\rho} + \left[(3 + 2n)- \rho^2 - \frac{\ell(\ell + 1)}{\rho^2}\right]w_{n \ell}(\rho) = 0.
    \end{equation*}
    No limite \(\rho^2 \gg (3+2n) - \frac{\ell(\ell+1)}{\rho^2}\), vemos que \(e^{-\frac{\rho^2}{2}}\) é solução assintótica. Consideremos uma solução do tipo \(w_{n\ell}(\rho) = e^{-\frac{\rho^2}{2}}f_{n \ell}(\rho)\), então
    \begin{equation*}
        \diff[2]{w_{n\ell}}{\rho} = e^{-\frac{\rho^2}{2}} \left[\diff[2]{f_{n\ell}}{\rho} - 2\rho \diff{f_{n\ell}}{\rho} + (\rho^2 - 1)f_{n\ell}(\rho)\right],
    \end{equation*}
    portanto a equação diferencial se torna
    \begin{equation*}
        \diff[2]{f_{n\ell}}{\rho} - 2\rho \diff{f_{n\ell}}{\rho} + \left[2(n+1) - \frac{\ell (\ell + 1)}{\rho^2}\right]f_{n\ell}(\rho) = 0.
    \end{equation*}
    No limite \(\rho \ll 1\), vemos que \(\rho^{\ell + 1}\) é solução assintótica, portanto consideramos \(f_{n\ell}(\rho) = \rho^{\ell+1} h_{n\ell}(\rho)\), obtendo
    \begin{equation*}
        \diff{f_{n\ell}}{\rho} = (\ell+1) \rho^\ell h_{n\ell}(\rho) + \rho^{\ell + 1}\diff{h_{n\ell}}{\rho}\quad\text{e}\quad
        \diff[2]{f_{n\ell}}{\rho} = \ell(\ell+1) \rho^{\ell-1} h_{n\ell}(\rho) + 2(\ell + 1)\rho^{\ell}\diff{h_{n\ell}}{\rho} + \rho^{\ell + 1}\diff[2]{h_{n\ell}}{\rho},
    \end{equation*}
    e então
    \begin{equation*}
        \rho\diff[2]{h_{n\ell}}{\rho} + 2\left[(\ell + 1) - \rho^{2}\right]\diff{h_{n\ell}}{\rho} + 2\left(n  - \ell\right)\rho h_{n\ell}(\rho) = 0.
    \end{equation*}
    Notemos que podemos tentar uma solução do tipo \(h_{n\ell}(\rho) = \sum_{k = 0}^\infty \alpha_k \rho^{k}\), donde segue que \(\alpha_1 = 0\) e
    \begin{equation*}
        \alpha_{k+2} = \frac{2(\ell + k - n)}{(k + 2\ell + 3)(k + 2)}\alpha_{k}
    \end{equation*}
    para todo \(k \in \mathbb{N}_0\), portanto \(\alpha_{2k + 1} = 0.\) Para \(\rho \gg 1\), o comportamento para \(k\) grande é o dominante, e vemos que \(\alpha_{k+2} \sim \frac{2\alpha_k}{k}\), obtendo o comportamento assintótico
    \begin{equation*}
        h_{n\ell}(\rho) \propto \sum_{j = 0} \frac{\rho^{2j}}{j!} = \exp(\rho^2),
    \end{equation*}
    de forma que \(R_{n\ell}\) não seria normalizável. Devemos ter então \(2j + \ell - n = 0\) com \(j\in \mathbb{N}_0\) de forma que a série se reduza a um polinômio de grau \(2j\) e os autovalores de energia sejam dados por \(\hbar \omega \left(2j + \ell + \frac32\right)\). Assim,
    dado \(n = 2j + \ell\), a paridade de \(\ell\) é igual a de \(n\), portanto o valor máximo de \(j\) é \(\frac{n}{2}\) se \(n\) é par e \(\frac{n - 1}{2}\) se \(n\) é ímpar. Para cada \(\ell\) correspondem \(2\ell + 1\) possíveis valores de \(m\), portanto a degenerescência do nível de energia \(n\) é
    \begin{equation*}
        g_{n} = \sum_{j = 0}^{\frac{n}{2}} [2(n - 2j) + 1] = \frac{(2n + 1)(n + 2)}{2} - 4\sum_{j=1}^{\frac{n}{2}} j = \frac{(2n + 1)(n+2)}{2} - \frac{n(n+2)}{2} = \frac{(n+1)(n+2)}{2},
    \end{equation*}
    se \(n\) é par e
    \begin{equation*}
        g_{n} = \sum_{j = 0}^{\frac{n - 1}{2}} [2(n-2j) + 1] = \frac{(n+1)(2n+1)}{2} - 4 \sum_{j=1}^{\frac{n-1}{2}} j = \frac{(n+1)(2n+1)}{2} - \frac{(n-1)(n+1)}{2} = \frac{(n+1)(n+2)}{2},
    \end{equation*}
    se \(n\) é ímpar. Isto é, \(g_n = \frac12 (n+1)(n+2)\) para todo \(n\), como havíamos encontrado antes.
\end{proof}
