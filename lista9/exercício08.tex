\begin{exercício}{Oscilador harmônico isotrópico}{exercício08}
    Encontre os autoestados e respectivas autofunções de um oscilador harmônico tridimensional isotrópico, isto é, cujo potencial é \(V(r) = \frac12 m \omega^2 r^2\).
    \begin{enumerate}[label=(\alph*)]
        \item Resolva o problema em coordenadas cartesianas. Qual a degenerescência dos autoestados de \(H\)?
        \item Resolva novamente o problema, mas desta vez imponha que os autoestados também são autovetores de \(L_z\) e de \(\vetor{L}^2\).
    \end{enumerate}
\end{exercício}
\begin{proof}[Resolução]

\end{proof}
