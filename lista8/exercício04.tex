\begin{exercício}{Partícula em trajetória circular}{exercício4}
    Considere uma partícula de massa \(\mu\) sujeita a uma trajetória circular de raio constante \(r\). A energia potencial será considerada nula em todos os pontos da trajetória.
    \begin{enumerate}[label=(\alph*)]
        \item Calcule a expressão clássica para a energia cinética da partícula em função de seu momento angular orbital \(\vetor{L}\).
        \item Determine a expressão das componentes de \(\vetor{L}\) em coordenadas cartesianas.
        \item Escreva o operador de momento angular \(L_z\) em coordenadas cartesianas.
        \item Calcule a expressão de \(L_z\) em coordenadas polares do plano de rotação e determine suas autofunções.
        \item Determine a expressão do hamiltoniano \(H\) da partícula em rotação.
        \item Mostre que \(L_z\) comuta com \(H\). O que se pode então deduzir das autofunções de \(H\)?
    \end{enumerate}
\end{exercício}
\begin{proof}[Resolução]

\end{proof}
