\begin{exercício}{Propriedades de sistemas unidimensionais}{exercício4}
    Considere uma partícula ligada, em uma dimensão.
    \begin{enumerate}[label=(\alph*)]
        \item Mostre que
            \begin{equation*}
                \diff*{\int_{\mathbb{R}}\dli{x}\conj{\psi}(x,t)\psi(x,t)}{t} = 0,
            \end{equation*}
            onde \(\ket{\psi(t)}\) não precisa ser um estado estacionário.
        \item Mostre que, se uma partícula está num estado estacionário em um tempo qualquer \(t_0\), então ela sempre permanecerá num estado estacionário.
        \item Se em \(t = 0\) a função de onda é constante na região \([-a,a]\) e nula em seu complemento, expresse a função de onda completa para um tempo \(t\) qualquer em termos dos autoestados do sistema.
        \item Mostre que em problemas unidimensionais o espectro de energia do estado ligado é sempre não degenerado.
        \item Mostre que a primeira derivada de uma função de onda de um estado estacionário é contínua mesmo que o potencial \(V(x)\) tiver uma descontinuidade finita.
        \item Mostre que para um potencial par tal que \(V(x) = V(-x)\) as autofunções dos estados estacionários da equação de evolução temporal para os estados ligados são ou par ou ímpar sob a troca \(x \mapsto -x\).
\end{enumerate}
\end{exercício}
\begin{proof}[Resolução do item (a)]
    Temos
    \begin{equation*}
        \diff*{\int_{\mathbb{R}}\dli{x}\conj{\psi}(x,t)\psi(x,t)}{t} = \diff*{\int_{\mathbb{R}}\dli{x} \braket{\psi(t)}{x} \braket{x}{\psi(t)}}{t} = \diff*{\braket{\psi(t)}{\psi(t)}}{t} = 0,
    \end{equation*}
    já que
    \begin{equation*}
        \diff*{\braket{\psi(t)}{\psi(t)}}{t} = \frac{1}{i\hbar}\bra{\psi(t)}H\ket{\psi(t)} - \frac{1}{i\hbar}\bra{\psi(t)}\herm{H}\ket{\psi(t)} = 0
    \end{equation*}
    segue do operador Hamiltoniano ser auto-adjunto.
\end{proof}

\begin{proof}[Resolução do item (b)]
    Seja \(\ket{\psi(t)}\) um estado que no instante \(t=0\) é um estado estacionário de energia \(\hbar \omega\). Sua evolução temporal é descrita por
    \begin{equation*}
        \ket{\psi(t)} = \exp\left(\frac{t}{i\hbar}H\right)\ket{\psi(0)} = \int_{\sigma(H)} \dli{\lambda} e^{-i\frac{\lambda}{\hbar} t}P_{\lambda} \ket{\psi(0)} = e^{-i\omega t} \ket{\psi(0)},
    \end{equation*}
    portanto há apenas uma diferença de fase em relação ao estado inicial.
\end{proof}

\begin{proof}[Resolução do item (c)]
    Consideramos o potencial como \(V(x) = 0\) em \([-a,a]\) e \(V(x) = \infty\) caso contrário. Sabemos que as autofunções do hamiltoniano para o potencial \(\tilde{V}(\tilde{x}) = 0\) para \(\tilde{x} \in [0,2a]\) são
    \begin{equation*}
        \psi_n(\tilde{x}) = \sqrt{\frac{2}{2a}} \sin\left(\frac{n\pi \tilde{x}}{2a}\right),
    \end{equation*}
    portanto as autofunções do hamiltoniano para o potencial considerado são dadas por
    \begin{equation*}
        \psi_n(x) = a^{-\frac12} \sin\left(\frac{n \pi x}{2a} + \frac{n \pi}{2}\right) = a^{-\frac12} \left[\sin\left(\frac{n \pi x}{2a}\right)\cos\left(\frac{n\pi}{2}\right) + \cos\left(\frac{n \pi x}{2a}\right)\sin\left(\frac{n\pi}{2}\right)\right],
    \end{equation*}
    isto é,
    \begin{equation*}
        \psi_n(x) = \begin{cases}
            a^{-\frac12} \sin\left(\frac{n \pi x}{2a}\right),&\text{se }n\text{ é par}\\
            a^{-\frac12} \cos\left(\frac{n \pi x}{2a}\right),&\text{se }n\text{ é ímpar}
        \end{cases}
    \end{equation*}
    para todo \(n \in \mathbb{N}\), onde utilizamos a liberdade de escolha de fase.

    Assim, no instante \(t = 0\) se a função de onda de uma partícula é dada por \(\varphi(x, 0) = \frac{1}{\sqrt{2a}}\) para \(x \in [-a,a]\), temos
    \begin{equation*}
        \varphi(x, t) = \sum_{n = 1}^\infty c_n e^{-i \omega_n t}\psi_n(x),
    \end{equation*}
    onde \(\hbar \omega_n = \frac{\hbar^2 n^2 \pi^2}{2m(2a)^2}\) e \(c_n = \frac{1}{\sqrt{2a}}\int_{-a}^a \dli{x} \conj{\psi_n}(x)\). Como \(\psi_n\) é real e é ímpar para todo \(n\) par, segue que \(c_{2n} = 0\), e temos
    \begin{equation*}
        c_{2n - 1} = \frac{2}{\sqrt{2a}} \int_{0}^a \dli{x} a^{-\frac12} \cos\left(\frac{(2n-1)\pi x}{2a}\right) = \frac{4}{\sqrt{2}(2n - 1)\pi} \sin\left(\frac{(2n - 1)\pi}{2}\right) = -\frac{(-1)^n\sqrt{8}}{(2n - 1) \pi},
    \end{equation*}
    isto é,
    \begin{equation*}
        \varphi(x, t) = -\frac{\sqrt{8}}{\pi} \sum_{n = 1}^\infty \frac{(-1)^n}{2n - 1}\exp\left(-i\frac{\hbar n^2 \pi^2}{2m(2a)^2}t\right)\cos\left(\frac{(2n-1)\pi x}{2a}\right)
    \end{equation*}
    é a função de onda desta partícula em qualquer instante \(t\).
\end{proof}

\begin{proof}[Resolução do item (d)]
    Sejam \(\ket{\psi_1}\) e \(\ket{\psi_2}\) autoestados de energia \(E\), então
    \begin{equation*}
        -\frac{\hbar^2}{2m}\diff[2]{\psi_1}{x} + V(x)\psi_1(x) = E\psi_1(x) \quad\text{e}\quad
        -\frac{\hbar^2}{2m}\diff[2]{\psi_2}{x} + V(x)\psi_2(x) = E\psi_2(x).
    \end{equation*}
    Multiplicando a primeira equação por \(\psi_2(x)\) e a segunda por \(\psi_1(x)\) e subtraindo os resultados, obtemos
    \begin{equation*}
        -\frac{\hbar^2}{2m} \left(\psi_2(x) \diff[2]{\psi_1}{x} - \psi_1(x) \diff[2]{\psi_2}{x}\right) = 0 \implies \diff*{\left(\psi_2(x)\diff{\psi_1}{x}-\psi_1(x) \diff{\psi_2}{x}\right)}{x} = 0,
    \end{equation*}
    isto é, existe uma constante \(\alpha\) tal que
    \begin{equation*}
        \psi_2(x) \diff{\psi_1}{x} - \psi_1(x) \diff{\psi_2}{x} = \alpha.
    \end{equation*}
    Como as funções de onda devem ser normalizáveis, temos \(\psi \to 0\) conforme \(x \to \infty\) e temos as suas derivadas limitadas, portanto concluímos que \(\alpha = 0\). Disso, segue que
    \begin{equation*}
        \frac{1}{\psi_1(x)}\diff{\psi_1}{x} = \frac{1}{\psi_2(x)}\diff{\psi_2}{x} \implies \psi_2(x) = \beta \psi_1(x),
    \end{equation*}
    para alguma constante \(\beta\). Isto é, \(\set{\ket{\psi_1},\ket{\psi_2}}\) é linearmente dependente, portanto o espectro de energia é não degenerado.
\end{proof}

\begin{proof}[Resolução do item (e)]
    Seja \(\ket{\psi(t)}\) um estado estacionário de energia \(E = \hbar \omega\), então sua função de onda \(\psi(x,t) = \braket{x}{\psi(t)} = \psi(x)\exp(-i\omega t)\) satisfaz
    \begin{equation*}
        -\frac{\hbar^2}{2m} \diff[2]{\psi}{x} + V(x)\psi(x) = E \psi(x).
    \end{equation*}
    Seja \(x_0 \in \mathbb{R}\), então integrando no intervalo \([x_0 - \varepsilon, x_0 + \varepsilon]\) contido no domínio relevante, segue que
    \begin{equation*}
        \diff{\psi}{x}[x = x_0 + \varepsilon] - \diff{\psi}{x}[x = x_0 - \varepsilon] = \frac{2m}{\hbar^2}\int_{x_0 - \varepsilon}^{x_0 + \varepsilon} \dli{x} \left[V(x) - E\right]\psi(x)
    \end{equation*}
    donde inferimos que se o potencial é contínuo em quase toda parte, isto é, descontínuo em no máximo um conjunto de medida nula, então \(\diff{\psi}{x}\) é contínuo em \(x_0\). Como \(x_0\) é arbitrário, segue que a primeira derivada da função de onda é contínua.
\end{proof}

\begin{proof}[Resolução do item (f)]
    Seja \(\ket{\psi(t)}\) um estado estacionário de energia \(E = \hbar \omega\), então sua função de onda \(\psi(x,t) = \braket{x}{\psi(t)} = \psi(x)\exp(-i\omega t)\) satisfaz
    \begin{equation*}
        -\frac{\hbar^2}{2m} \diff*[2]{\psi(x)}{x} + V(x)\psi(x) = E \psi(x).
    \end{equation*}
    Com a troca de variáveis \(x \mapsto -x\), temos
    \begin{equation*}
        -\frac{\hbar^2}{2m} \diff*[2]{\psi(-x)}{x} + V(x)\psi(-x) = E\psi(-x),
    \end{equation*}
    pela paridade do potencial. Isto é, para a energia \(E\) tanto \(\psi(x)\) quanto \(\psi(-x)\) satisfaz a equação de evolução temporal, donde segue da normalização do estado e da não degenerescência do espectro que \(\psi(x) = \pm \psi(-x)\), isto é, ou \(\psi(x)\) é par ou é ímpar.
\end{proof}
