\begin{exercício}{Propriedades do comutador e do traço}{exercício3}
    Sejam \(A, B,\) e \(C\) operadores lineares. Mostre que
    \begin{enumerate}[label=(\alph*)]
        \item \(\commutator{A}{BC} = B\commutator{A}{C} + \commutator{A}{B}C\);
        \item \(\Tr(AB) = \Tr(BA)\);
        \item \(\Tr(S A S^{-1}) = \Tr(A)\) para qualquer operador inversível \(S\); e
        \item \(\Tr(ABC) = \Tr(BCA) = \Tr(CAB)\).
    \end{enumerate}
\end{exercício}
\begin{proof}[Resolução]
    Temos
    \begin{align*}
        B \commutator{A}{C} + \commutator{A}{B}C &= B(AC - CA) + (AB - BA)C\\
                                                 &= BAC - BCA + ABC - BAC\\
                                                 &= ABC - BCA\\
                                                 &= \commutator{A}{BC},
    \end{align*}
    portanto concluímos (a).

    Tomemos uma base ortonormal e enumeremo-la por \(\ket{n}\) como o seu \(n\)-ésimo vetor. Seja \(\ket{n'}\) uma outra base ortonormal, então existe um operador unitário \(U\) tal que \(U\ket{n} = \ket{n'}\). Para um operador linear \(X\) qualquer, segue que
    \begin{equation*}
        \sum_{n = 1} \bra{n} X \ket{n} = \sum_{n = 1} \bra{n} \herm{U}U X \herm{U}U \ket{n} = \sum_{n' = 1} \bra{n'} U X \herm{U} \ket{n'}.
    \end{equation*}
    Temos ainda para um par de operadores lineares \(Y\) e \(Z\) que
    \begin{align*}
        \sum_{n' = 1}^{N} \bra{n'} YZ \ket{n'} &= \sum_{n'=1}^{N} \sum_{m' = 1}^{N} \bra{n'} Y \ketbra{m'}{m'}Z\ket{n'}\\
                                            &= \sum_{n'=1}^{N}\sum_{m'=1}^{N} \bra{m'} Z \ketbra{n'}{n'} Y\ket{m'}\\
                                            &= \sum_{n'=1}^{N}\sum_{m'=1}^{N} \bra{n'} Z \ketbra{m'}{m'} Y\ket{n'}\\
                                            &= \sum_{n'=1}^{N} \bra{n'} ZY \ket{n'}.
    \end{align*}
    Fixando \(Y = U\) e \(Z = X\herm{U}\), obtemos
    \begin{equation*}
        \sum_{n=1} \bra{n} X \ket{n} = \sum_{n' = 1} \bra{n'} X\herm{U} U \ket{n'} = \sum_{n' = 1} \bra{n'} X \ket{n'},
    \end{equation*}
    isto é, o traço \(\Tr(X) = \sum_{n=1} \bra{n} X \ket{n}\) é bem definido e independe da escolha de base ortonormal utilizada. Fixando \(Y = A\) e \(Z = B\) no cômputo acima, mostramos (b).

    Seja \(S\) um operador inversível, então
    \begin{equation*}
        \Tr(SAS^{-1}) = \Tr(S(AS^{-1})) = \Tr(AS^{-1}S) = \Tr(A)
    \end{equation*}
    segue de (b), mostrando (c).

    Um outro corolário de (b) é
    \begin{equation*}
        \Tr(ABC) = \Tr(A(BC)) = \Tr(BCA).
    \end{equation*}
    Repetindo estas permutações cíclicas por conta da associatividade da composição de operadores lineares e utilizando (b), obtemos (d).
\end{proof}
